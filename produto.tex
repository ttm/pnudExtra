\documentclass[12pt]{report}
\usepackage[usenames,dvipsnames]{color}
\usepackage{listings}
\usepackage{graphicx}
\usepackage{fancyhdr}
\usepackage{framed}
\usepackage[T1]{fontenc}
\usepackage[toc,page]{appendix}
\usepackage[utf8]{inputenc}
\usepackage[brazil]{babel}
\usepackage{fancyvrb}
\usepackage[hmargin=2cm,vmargin=2cm]{geometry}
\usepackage{lastpage}
\usepackage{pdfpages}
\usepackage{makeidx}
\usepackage{hyperref}
\pagestyle{fancy}
\usepackage{enumitem}
% cabecalho e rodapé
\setlength{\headheight}{120pt}
\setlength{\textheight}{550pt}
\renewcommand{\headrulewidth}{0pt}
\lhead{\includegraphics[scale=0.03]{brasao.png}}
\chead{\includegraphics[scale=0.5]{logo-brasil-sem-pobreza2.png}}
\rhead{\includegraphics[scale=0.5]{logo-pnud.png}}
\cfoot{\textbf{\ProjectCode\ - Inovando a democracia participativa}}
\rfoot{\thepage}

\hyphenation{par-ti-ci-pa-ção}
\bibliographystyle{ieeetr}

% definições sobre o autor e o produto
\newcommand{\MyName}{Renato Fabbri}
\newcommand{\MySurnameForename}{Fabbri, Renato}
\newcommand{\SupervisorName}{Gabriella Vieira Oliveira Gonçalves}
\newcommand{\MyEmail}{renato.fabbri@gmail.com}
\newcommand{\ContractNumber}{2013/000566}
\newcommand{\ContractYear}{2014}
\newcommand{\ProjectCode}{Projeto BRA/12/018}
\newcommand{\NomeSecretaria}{Secretaria-Geral da Presidência da República}
%Q\newcommand{\SiglaSecretaria}{SG/PR}
\newcommand{\SiglaSecretaria}{Secretaria: SNAS }
\newcommand{\ProductNumber}{Extra}
\newcommand{\ProductTitle}{Anotações sobre a leitura de produtos dos outros consultores do Projeto BRA/12/018}
\newcommand{\ProductSubtitle}{confluências, aplicações, dúvidas, sugestões, correções}
\newcommand{\ProductDescription}{"Ferramentas assistidas de categorização de conteúdo: Com Processamento de Linguagem Natural e de Redes Complexas, adaptadas para o ambiente do portal de participação."
}

\newcommand{\ProductValue}{R\$ 0,00 (zero reais e zero centavos)}
\newcommand{\ObjetoContratacao}{
Aporte de conhecimentos e tecnologias para especificação de vocabulário e ferramentas assistidas que utilizam processamento de linguagem natural e análise de redes complexas para o conteúdo do portal da participação social.
}
\newcommand{\DataEntrega}{Agosto de 2014}
\newcommand{\PalavrasChave}{reconhecimento de padrões, redes complexas, processamento de linguagem natural, participação social}
\newcommand{\pp}[1]{

\textbf{Parágrafo #1:}

}
% lista de abreviações
\makeindex

\begin{document}

\input{folhaderosto.tex}
\input{folhadeaprovacao.tex}
\input{folhadeidentificacao.tex}
\tableofcontents
\newpage


\begin{abstract}
Este documento registra a reflexão sobre produtos dos outros consultores do mesmo projeto (BRA/12/018).
Cada consultor entrega alguns ``produtos'' para os órgãos interessados. No caso, são documentos escritos,
que relatam atividades, pesquisas, propostas, enfim, o que for pertinente para o trabalho.
Na leitura destes documentos, são registradas anotações pertinentes à consultoria de contrato 2013/000566, sob responsabilidade
do autor deste produto extra. Cada parte do documento corresponde a um consultor, cada capítulo corresponde
a um documento/produto, as seções e subseções correspondem à estrutura original de cada documento considerado.\\

{\bf Palavras-chave:} \PalavrasChave.
\end{abstract}
\newpage
\part{Produtos de Fabrício Solagna}
\chapter{Análise de experiências nacionais e internacionais de participação mediada por Internet}
Este documento contempla uma análise de experiências nacionais e internacionais de
participação mediada por Internet considerando os aspectos de inclusão, inovação e
deliberação, com a finalidade de gerar uma matriz de características para integração ao portal
Participa.br. As iniciativas analisadas foram agrupadas em 5 capítulos temáticos, somando 18
iniciativas e mais de 30 ferramentas de participação digital. O capítulo final apresenta
recomendações metodológicas à luz das experiências estudadas.

\section{Introdução}

Página 10

\pp{1} ok.

\pp{2} ok. Unipresente deve ser Omnipresente ou Ubíqua.

\pp{3} ok.

\pp{4} Legal da definição do Macintosh.

\pp{5} Porquê esta visão de participação social se opõe a esta capacidade do indivíduo influenciar os processos? É quanto à visão do Macintosh?

\pp{6} {\bf \color{red} Motivação para o SM essa inspiração através do empírico nas redes sociais}.

\pp{7} Legal a definição de ``gift economics'' como processos nos quais atores cooperam em projetos maiores que as corporações.

\pp{8} Ok.

\noindent Página 11

\pp{9} Dada a função da OGP, é pertinente observar quais os posicionamentos e recursos que exibem quanto à web semântica.

\pp{10} Legal a necessidade da ferramenta de participação de delimitar \emph{input} e \emph{output}.

\pp{11} Muito interessante. Esquematicamente: políticas públicas equilibra: necessidades da comunidade, recursos disponíveis e capacidade de execução. Participação pela internet: há hiato entre capacidade de coletar opinião e sistematização para ação. Congregando colaboração massiva, qualidade e resultado prático, cita: Wikipédia, o Software Livre e o Crowdfunding.

\pp{12} Interessante que nos gov abertos essas iniciativas tiveram pouco impacto.

\noindent Página 12

\pp{13} ok.

\pp{14} Legal da suavização da separação digital e presencial.

\pp{15} ok.

\pp{16} Interessante: inovação, inclusão e protagonismo; deliberação, promoção, construção de políticas públicas.

\pp{17} foco no executivo e na fiscalização de serviço público. Legal do longo histórico com o Legislativo.

\pp{18} ok. Cap 2 -> Democracia digital ligada aa PR nos EUA, Chile, Bolívia e País de Gales.

\noindent Página 13

\pp{19} Cap 3 -> visualização e deliberação de Orçamento Público.
Legal da simulação e visualização pela popularização dos datasets. Seria bom ver quais as iniciativas do OKF caso não esteja neste cap.

\pp{20} Cap4 -> projs/entidades com relevância na democracia digital.

\pp{21} Cap5 -> monitoramento e fiscalização de obras e ações do governo.

\pp{22} Cap6 -> Gabinete Digital do RS. Três maiores consultas nacionais ou internacionais?

\pp{23} Cap7 -> Recomendações de metodologias e ferramentas.

\pp{24} Legal dos três vértices: elaboração, fiscalização e monitoramento.

\noindent Página 14 

\pp{25} Ok.

\section{Experiências de democracia digital ligadas
diretamente a presidência da república}
Página 15
\subsection{Chile: Gobierno Abierto}
Página 16

Figura 1: ok.

\pp{1} ok.

\pp{2} ok.

\pp{3} ok.

\noindent Página 17

Figura 2: ok

\noindent Página 18

\pp{4} ok

\pp{5} ok

\pp{6} ok. Interessante essas consultas que já passaram estarem acessíveis junto a conteúdo jornalístico e multimídia.

\noindent Página 19

Figura 4: ok.

\pp{7} ok.

\noindent Página 20

\subsubsection{Resultados alcançados}

\pp{8} ok.

\subsection{Bolívia: Urna de Cristal}

\indent Figura 5: ok.

\noindent Página 21

\pp{1} ok.

\subsubsection{Metodologia e arquitetura de escolha}

\pp{2} Legal da proposta ou pergunta e então divulgação nas redes sociais.

\pp{3} ok.

\subsubsection{Resultados alcançados}

\pp{4} Interessante o envolvimento das equipes nas redes sociais. {\bf \color{red} Talvez valha eu procurar melhor como fazem isso.}

\noindent Página 22

Figura 6: ok.

\subsection{EUA: Iniciativas de diálogo com o Governo}

\pp{1} ok.

\pp{2} ok.

\pp{3} ok.

\noindent Página 23

\subsubsection{Open for Question}

\pp{4} Legal do envio de questões. Interessante a ambiguidade do texto, que deixou a sugestão de fazer uma relação de pessoas cadastradas para receberem a pergunta.

Figura 7: ok.

Metodologia

\pp{5} bacana a ideia de resposta em video pelo presidente.

\subsubsection{AskObama on Twitter}

\pp{6} super da equipe do Twitter junto.

\noindent Página 24

Figura 8: ok.

\subsubsection{Reddit}

\pp{7} ok.

\noindent Página 25

Figura 9: ok.

\subsubsection{We the People}

\pp{8} Interessante de petição estar integrado ao site da casa branca. {\bf \color{red} Talvez pensar em algo assim para o Participa.br, fornecendo ao usuário meios sofisticados de ativar suas redes sociais.}

Metodologia

\pp{9} ok.

\pp{10} ok.

\pp{11} Legal das respostas quando a petição tem mais de 100 mil assinaturas.

\noindent Página 26

Resultados

\pp{12} ok.

Figura 10: ok.

\subsection{País de Gales: E-petitions}

\pp{13} ok.

\noindent Página 27

Figura 11: ok.

\noindent Página 28

\section{Experiências de visualização e deliberação de Orçamento Público}

\noindent Página 29

\subsection{Orçamento Participativo (OP)}

\pp{1} ok dos três períodos do OP.

\pp{2} Dúbia a construção? Assumindo aqui que a segunda fase é situada entre 1989 e 1992 enquando a primeira fase vai de 1960 até 1998. (?)

\pp{3} OP é criação do PT? (creio que não, mas o texto está me dando a entender isso)

\pp{4} Interessante a mescla de OP com outras iniciativas de democracia digital.

\pp{5} ok.

\pp{6} Bem curiosa essa ausência de OP federal e pouca estadual.

\pp{7} ok.

\noindent Página 30

\subsection{Orçamento Participativo Digital – Belo Horizonte}

\pp{8} ok.

Figura 12: ok.

\pp{9} ok.

\subsubsection{Metodologia do OP Digital}

\pp{10} ok.

\pp{11} ok, bem legal da votação via ligações telefonicas e via app de celular.

\noindent Página 31

\pp{12} ok

Figura 13: ok.

\pp{13} ok.

\subsubsection{Resultados}

\pp{14} ok.

\noindent Página 32

\pp{15} ok.

\subsection{Nova Iorque: Participatory Budgeting in New York City}

\pp{16} ok.

\subsubsection{Metodologia e arquitetura de escolha}

\pp{17} ok.

\noindent Página 33

Figura 14: ok.
Figura 15: ok.

\noindent Página 34

\subsubsection{Orçamento Participativo em outras cidades norte-americanas}

\pp{18} ok.

\pp{19} ok.

\subsection{Liverpool: Budget Simulator}

\pp{20} que tendência?

Figura 16: ok.

\subsection{Stabilize the U.S. Debt}
\noindent Página 35
\pp{21} ok.
\pp{22} ok.
\pp{23} {\bf \color{red} Gamificação}.
\pp{24} ok.

Figura 17: ok.

\noindent Página 36

\subsection{Where Does My Money Go}
\pp{25} ok.
\pp{26} ok.
\pp{27} ok.
\pp{28} ok.
\pp{29} ok.

Figura 18: ok.

\noindent Página 37

\subsubsection{Metodologia utilizada}
\pp{30} ótimo das techs livres.

Figura 19: ok.

\subsection{Aplicações no Brasil}
\subsubsection{Para onde foi meu dinheiro}
ok tudo.
\subsubsection{Orçamento ao seu alcance}
Bem legal sobre o orçamento federal, até porque não há OP federal.

\section{Projetos de democracia digital oriundos da sociedade civil}

\noindent Página 42

\subsection{PortoAlegre.cc}
\pp{1} ok.
\pp{2} ok. Quais conceitos de inteligência social?
\pp{3} ok.
\subsubsection{Metodologia e arquitetura de escolha}
\pp{4} ok.
\pp{5} ok.
\pp{6} Super legal dos pilares: cultura de cidadania, ética do cuidado, corresponsabilização cidadã e engajamento cívico.
\pp{7} 

\noindent Página 43

Figura 23: ok.

\pp{8} ok.
\pp{9} Interessante a confluência de pautas levantadas pela sociedade com agendas públicas.
\pp{10} 

\noindent Página 44

Figura 24: ok.

\subsubsection{Fases de uma causa}
\pp{11} ok.
\pp{12} ok.
\pp{13} ok.
\pp{14} ok.

\subsubsection{Resultados}
\pp{15} ok.

\subsection{MySociet}
\pp{16} ok da UKCOD.
\pp{17} só de transportes?








\part{Notas aleatórias}

\chapter{Simulação de processos}

Difusão, contágio, acessibilidade e fofoca nas redes reais do Participa.br e de outras instâncias.




\newpage
\bibliography{bibliografia}
\newpage
%\listoffigures
\input{listadeabreviaturas.tex}
\newpage
\printindex
\newpage
%\input{listadeanexos.tex}
\appendix
\section*{Anexos}
\addcontentsline{toc}{section}{Anexos}  
A seguir estão trechos de código Python que executam funções básicas de classificação de conteúdo do Participa.br ou úteis para isso.
Estes códigos estão executáveis online, acessíceis nos browsers web usuais (firefox, chrome(um)), através de um IPython Notebook. Mais informações no Anexo~\ref{subsec:online}.
%\appendixpage
%\section*{Anexos}
%\addcontentsline{toc}{section}{Anexos}  
\newcommand\invisiblesection[1]{%
  \refstepcounter{section}%
  \addcontentsline{toc}{section}{\protect\numberline{\thesection}#1}%
  \sectionmark{#1}}
\newcommand\invisiblesubsection[1]{%
  \refstepcounter{subsection}%
  \addcontentsline{toc}{subsection}{\protect\numberline{\thesubsection}#1}%
  \subsectionmark{#1}}

%\input{appendix.tex}


\section{Instâncias online}\label{subsec:online}
Alguns recursos desenvolvidos neste trabalho foram disponibilizados online, via http, pela natureza do trabalho e por conveniências. Consistem em ferramentas de streaming, ferramentas de análise, experimentos artísticos e repositórios. Uma seleção pertinente a este escrito é:
\begin{itemize}
    \item IPython Notebook, com os códigos dos Anexos~\ref{subsec:etp}-\ref{subsec:csparql}: \url{http://200.144.255.210:8003}
    \item Endpoint Fuseki/Jena, para acesso aos dados do Participa.br via critérios semânticos: \url{http://200.144.255.210:8082}.
    \item Telões para streaming de estruturas sociais (cria 3 redes de relacionamento (d3js) entre os agentes e algumas seleções de palavras e contextos): \url{http://ocupagov.meteor.com}
    \item BrServers, com o resumo ``redes+filtros multidimensionais'': \url{http://gentle-mesa-5082.herokuapp.com/twitter/arenaNETmundial/}
    \item MyNSA, pra coleta de dados do Facebook e Twitter: \url{http://mynsa.meteor.com}
    \item MMISSA, disposições livres, iniciais e artísticas sobre dados das redes sociais: \url{http://mmissa.meteor.com}
    \item MM, sonificação de estruturas sociais, com propósitos artísticos e para acessibilidade: \url{http://mm.meteor.com}
    \item Respositório de códigos e com este documento: \url{https://github.com/ttm/pnud3}
\end{itemize}
\end{document}
