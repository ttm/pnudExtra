\documentclass[12pt]{report}
\usepackage[usenames,dvipsnames]{color}
\usepackage{listings}
\usepackage{graphicx}
\usepackage{fancyhdr}
\usepackage{framed}
\usepackage[T1]{fontenc}
\usepackage[toc,page]{appendix}
\usepackage[utf8]{inputenc}
\usepackage[brazil]{babel}
\usepackage{fancyvrb}
\usepackage[hmargin=2cm,vmargin=2cm]{geometry}
\usepackage{lastpage}
\usepackage{pdfpages}
\usepackage{makeidx}
\usepackage{hyperref}
\pagestyle{fancy}
\usepackage{enumitem}
% cabecalho e rodapé
\setlength{\headheight}{120pt}
\setlength{\textheight}{550pt}
\renewcommand{\headrulewidth}{0pt}
\lhead{\includegraphics[scale=0.03]{brasao.png}}
\chead{\includegraphics[scale=0.5]{logo-brasil-sem-pobreza2.png}}
\rhead{\includegraphics[scale=0.5]{logo-pnud.png}}
\cfoot{\textbf{\ProjectCode\ - Inovando a democracia participativa}}
\rfoot{\thepage}

\hyphenation{par-ti-ci-pa-ção}
\bibliographystyle{ieeetr}

% definições sobre o autor e o produto
\newcommand{\MyName}{Renato Fabbri}
\newcommand{\MySurnameForename}{Fabbri, Renato}
\newcommand{\SupervisorName}{Gabriella Vieira Oliveira Gonçalves}
\newcommand{\MyEmail}{renato.fabbri@gmail.com}
\newcommand{\ContractNumber}{2013/000566}
\newcommand{\ContractYear}{2014}
\newcommand{\ProjectCode}{Projeto BRA/12/018}
\newcommand{\NomeSecretaria}{Secretaria-Geral da Presidência da República}
%Q\newcommand{\SiglaSecretaria}{SG/PR}
\newcommand{\SiglaSecretaria}{Secretaria: SNAS }
\newcommand{\ProductNumber}{Extra}
\newcommand{\ProductTitle}{Anotações sobre a leitura de produtos dos outros consultores do Projeto BRA/12/018}
\newcommand{\ProductSubtitle}{confluências, aplicações, dúvidas, sugestões, correções}
\newcommand{\ProductDescription}{"Ferramentas assistidas de categorização de conteúdo: Com Processamento de Linguagem Natural e de Redes Complexas, adaptadas para o ambiente do portal de participação."
}

\newcommand{\ProductValue}{R\$ 0,00 (zero reais e zero centavos)}
\newcommand{\ObjetoContratacao}{
Aporte de conhecimentos e tecnologias para especificação de vocabulário e ferramentas assistidas que utilizam processamento de linguagem natural e análise de redes complexas para o conteúdo do portal da participação social.
}
\newcommand{\DataEntrega}{Agosto de 2014}
\newcommand{\PalavrasChave}{reconhecimento de padrões, redes complexas, processamento de linguagem natural, participação social}
\newcommand{\pp}[1]{

\textbf{Parágrafo #1:}
}
\newcommand{\PP}[1]{

\noindent Página #1

}
\newcommand{\VV}[1]{{\bf \color{red} #1}}

% lista de abreviações
\makeindex

\begin{document}

\input{folhaderosto.tex}
\input{folhadeaprovacao.tex}
\input{folhadeidentificacao.tex}
\tableofcontents
\newpage


\begin{abstract}
Este documento registra a reflexão sobre produtos dos outros consultores do mesmo projeto (BRA/12/018).
Cada consultor entrega alguns ``produtos'' para os órgãos interessados. No caso, são documentos escritos,
que relatam atividades, pesquisas, propostas, enfim, o que for pertinente para o trabalho.
Na leitura destes documentos, são registradas anotações pertinentes à consultoria de contrato 2013/000566, sob responsabilidade
do autor deste produto extra. Cada parte do documento corresponde a um consultor, cada capítulo corresponde
a um documento/produto, as seções e subseções correspondem à estrutura original de cada documento considerado.\\

{\bf Palavras-chave:} \PalavrasChave.
\end{abstract}
\newpage
\part{Produtos de Fabricio Solagna}
\chapter{Análise de experiências nacionais e internacionais de participação mediada por Internet}
Este documento contempla uma análise de experiências nacionais e internacionais de
participação mediada por Internet considerando os aspectos de inclusão, inovação e
deliberação, com a finalidade de gerar uma matriz de características para integração ao portal
Participa.br. As iniciativas analisadas foram agrupadas em 5 capítulos temáticos, somando 18
iniciativas e mais de 30 ferramentas de participação digital. O capítulo final apresenta
recomendações metodológicas à luz das experiências estudadas.

\section{Introdução}

Página 10

\pp{1} ok.

\pp{2} ok. Unipresente deve ser Omnipresente ou Ubíqua.

\pp{3} ok.

\pp{4} Legal da definição do Macintosh.

\pp{5} Porquê esta visão de participação social se opõe a esta capacidade do indivíduo influenciar os processos? É quanto à visão do Macintosh?

\pp{6} {\bf \color{red} Motivação para o SM essa inspiração através do empírico nas redes sociais}.

\pp{7} Legal a definição de ``gift economics'' como processos nos quais atores cooperam em projetos maiores que as corporações.

\pp{8} Ok.

\noindent Página 11

\pp{9} Dada a função da OGP, é pertinente observar quais os posicionamentos e recursos que exibem quanto à web semântica.

\pp{10} Legal a necessidade da ferramenta de participação de delimitar \emph{input} e \emph{output}.

\pp{11} Muito interessante. Esquematicamente: políticas públicas equilibra: necessidades da comunidade, recursos disponíveis e capacidade de execução. Participação pela internet: há hiato entre capacidade de coletar opinião e sistematização para ação. Congregando colaboração massiva, qualidade e resultado prático, cita: Wikipédia, o Software Livre e o Crowdfunding.

\pp{12} Interessante que nos gov abertos essas iniciativas tiveram pouco impacto.

\noindent Página 12

\pp{13} ok.

\pp{14} Legal da suavização da separação digital e presencial.

\pp{15} ok.

\pp{16} Interessante: inovação, inclusão e protagonismo; deliberação, promoção, construção de políticas públicas.

\pp{17} foco no executivo e na fiscalização de serviço público. Legal do longo histórico com o Legislativo.

\pp{18} ok. Cap 2 -> Democracia digital ligada aa PR nos EUA, Chile, Bolívia e País de Gales.

\noindent Página 13

\pp{19} Cap 3 -> visualização e deliberação de Orçamento Público.
Legal da simulação e visualização pela popularização dos datasets. Seria bom ver quais as iniciativas do OKF caso não esteja neste cap.

\pp{20} Cap4 -> projs/entidades com relevância na democracia digital.

\pp{21} Cap5 -> monitoramento e fiscalização de obras e ações do governo.

\pp{22} Cap6 -> Gabinete Digital do RS. Três maiores consultas nacionais ou internacionais?

\pp{23} Cap7 -> Recomendações de metodologias e ferramentas.

\pp{24} Legal dos três vértices: elaboração, fiscalização e monitoramento.

\noindent Página 14 

\pp{25} Ok.

\section{Experiências de democracia digital ligadas
diretamente a presidência da república}
Página 15
\subsection{Chile: Gobierno Abierto}
Página 16

Figura 1: ok.

\pp{1} ok.

\pp{2} ok.

\pp{3} ok.

\noindent Página 17

Figura 2: ok

\noindent Página 18

\pp{4} ok

\pp{5} ok

\pp{6} ok. Interessante essas consultas que já passaram estarem acessíveis junto a conteúdo jornalístico e multimídia.

\noindent Página 19

Figura 4: ok.

\pp{7} ok.

\noindent Página 20

\subsubsection{Resultados alcançados}

\pp{8} ok.

\subsection{Bolívia: Urna de Cristal}

\indent Figura 5: ok.

\noindent Página 21

\pp{1} ok.

\subsubsection{Metodologia e arquitetura de escolha}

\pp{2} Legal da proposta ou pergunta e então divulgação nas redes sociais.

\pp{3} ok.

\subsubsection{Resultados alcançados}

\pp{4} Interessante o envolvimento das equipes nas redes sociais. {\bf \color{red} Talvez valha eu procurar melhor como fazem isso.}

\noindent Página 22

Figura 6: ok.

\subsection{EUA: Iniciativas de diálogo com o Governo}

\pp{1} ok.

\pp{2} ok.

\pp{3} ok.

\noindent Página 23

\subsubsection{Open for Question}

\pp{4} Legal do envio de questões. Interessante a ambiguidade do texto, que deixou a sugestão de fazer uma relação de pessoas cadastradas para receberem a pergunta.

Figura 7: ok.

Metodologia

\pp{5} bacana a ideia de resposta em video pelo presidente.

\subsubsection{AskObama on Twitter}

\pp{6} super da equipe do Twitter junto.

\noindent Página 24

Figura 8: ok.

\subsubsection{Reddit}

\pp{7} ok.

\noindent Página 25

Figura 9: ok.

\subsubsection{We the People}

\pp{8} Interessante de petição estar integrado ao site da casa branca. {\bf \color{red} Talvez pensar em algo assim para o Participa.br, fornecendo ao usuário meios sofisticados de ativar suas redes sociais.}

Metodologia

\pp{9} ok.

\pp{10} ok.

\pp{11} Legal das respostas quando a petição tem mais de 100 mil assinaturas.

\noindent Página 26

Resultados

\pp{12} ok.

Figura 10: ok.

\subsection{País de Gales: E-petitions}

\pp{13} ok.

\noindent Página 27

Figura 11: ok.

\noindent Página 28

\section{Experiências de visualização e deliberação de Orçamento Público}

\noindent Página 29

\subsection{Orçamento Participativo (OP)}

\pp{1} ok dos três períodos do OP.

\pp{2} Dúbia a construção? Assumindo aqui que a segunda fase é situada entre 1989 e 1992 enquando a primeira fase vai de 1960 até 1998. (?)

\pp{3} OP é criação do PT? (creio que não, mas o texto está me dando a entender isso)

\pp{4} Interessante a mescla de OP com outras iniciativas de democracia digital.

\pp{5} ok.

\pp{6} Bem curiosa essa ausência de OP federal e pouca estadual.

\pp{7} ok.

\noindent Página 30

\subsection{Orçamento Participativo Digital – Belo Horizonte}

\pp{8} ok.

Figura 12: ok.

\pp{9} ok.

\subsubsection{Metodologia do OP Digital}

\pp{10} ok.

\pp{11} ok, bem legal da votação via ligações telefonicas e via app de celular.

\noindent Página 31

\pp{12} ok

Figura 13: ok.

\pp{13} ok.

\subsubsection{Resultados}

\pp{14} ok.

\noindent Página 32

\pp{15} ok.

\subsection{Nova Iorque: Participatory Budgeting in New York City}

\pp{16} ok.

\subsubsection{Metodologia e arquitetura de escolha}

\pp{17} ok.

\noindent Página 33

Figura 14: ok.
Figura 15: ok.

\noindent Página 34

\subsubsection{Orçamento Participativo em outras cidades norte-americanas}

\pp{18} ok.

\pp{19} ok.

\subsection{Liverpool: Budget Simulator}

\pp{20} que tendência?

Figura 16: ok.

\subsection{Stabilize the U.S. Debt}
\noindent Página 35
\pp{21} ok.
\pp{22} ok.
\pp{23} {\bf \color{red} Gamificação}.
\pp{24} ok.

Figura 17: ok.

\noindent Página 36

\subsection{Where Does My Money Go}
\pp{25} ok.
\pp{26} ok.
\pp{27} ok.
\pp{28} ok.
\pp{29} ok.

Figura 18: ok.

\noindent Página 37

\subsubsection{Metodologia utilizada}
\pp{30} ótimo das techs livres.

Figura 19: ok.

\subsection{Aplicações no Brasil}
\subsubsection{Para onde foi meu dinheiro}
ok tudo.
\subsubsection{Orçamento ao seu alcance}
Bem legal sobre o orçamento federal, até porque não há OP federal.

\section{Projetos de democracia digital oriundos da sociedade civil}

\noindent Página 42

\subsection{PortoAlegre.cc}
\pp{1} ok.
\pp{2} ok. Quais conceitos de inteligência social?
\pp{3} ok.
\subsubsection{Metodologia e arquitetura de escolha}
\pp{4} ok.
\pp{5} ok.
\pp{6} Super legal dos pilares: cultura de cidadania, ética do cuidado, corresponsabilização cidadã e engajamento cívico.
\pp{7} 

\noindent Página 43

Figura 23: ok.

\pp{8} ok.
\pp{9} Interessante a confluência de pautas levantadas pela sociedade com agendas públicas.
\pp{10} 

\noindent Página 44

Figura 24: ok.

\subsubsection{Fases de uma causa}
\pp{11} ok.
\pp{12} ok.
\pp{13} ok.
\pp{14} ok.

\subsubsection{Resultados}
\pp{15} ok.
\noindent Página 45
\subsection{MySociet}
\pp{16} ok da UKCOD.
\pp{17} só de transportes?
\pp{18} ok. Legal que trabalham sob encomenda, talvez fazer contato em algum momento ou trazer para uma consultoria presencial.
\pp{19} ok, lembra o lm e orgs de gsoc.
\pp{20} ok.

\subsubsection{Alaveteli}
\pp{21} legal. Em que países são ou foram usados?
\noindent Página 46
\pp{22} ok.
\pp{23} ok.
Figura 25: ok.
\pp{24} ok.
\pp{25} ok.
\noindent Página 47
\subsubsection{FixMyStreet}
\pp{26} ok.
\pp{27} Figura 26: ok.
\pp{28} ok.
\noindent Página 48
\pp{29} ok.
\subsubsection{Eu prometo}
\pp{30} ok. Mandar p alguns amigos.
Figura 27: ok.
\pp{31} ok.
\noindent Página 49
\subsection{SeeClickFix}
\pp{32} ok.
\pp{33} ok.
\pp{34} ok.
Figura 28: ok.
\noindent Página 50
\subsection{All Our Ideas}
\pp{35} ok.
\pp{36} ok.
\pp{37} bem legal. Belo parágrafo sobre All Our Ideas e muito do que a participação social trata no momento.
\subsubsection{Metodologia empregada}
\pp{38} ok.
\noindent Página 51
Figura 29: ok.
\pp{39} ok. Dá vontade de analisar os dados de uma consulta com o Allourideas, também de ver como funciona, e caracterizar estatísticamente. Talvez também propor dinâmicas em rede para melhor coleta de ideias e votos.
\pp{40} ok.
\noindent Página 52
\pp{41} ok. Seria interessante um parâmetro para esse "bastante fiel".
\pp{42} ok.
\pp{43} ok.
\subsection{Cidade Democrática}
\pp{44} ok.
\pp{45} ok. Propor no Cidade Democrática de todo estudante de curso superior público dê 1h de aula gratuita por semana em comunidades menos favorecidas.
\pp{46} ok. Muito legal do sonho p Xingu.
\noindent Página 53
Figura 30: ok.
\noindent Página 54
\section{Ferramentas de monitoramento de obras e políticas públicas}
\noindent Página 55
\pp{1} ok. {\bf \color{red} referência boa para fazer ponte com os avanços recentes e previstos com as redes sociais.}
\pp{2} ok.
\pp{3} ok
\subsection{OnTrack}
\pp{4} ok.
\pp{5} ok.
Figura 31: ok.
Figura 32: ok. Seria bastante interessante saber se há alguma iniciativa que implementa ou visa implementar o OnTrack no Brasil e se podemos usar instâncias já instaladas.
\subsection{De Olho nas Obras}
\pp{6} ok.
\subsection{RapidSMS}
\pp{7} bem legal. {\bf \color{red} Considerando p usar c AA.}
\noindent Página 57
\pp{8} Muito bom. Seria bastante interessante saber se há alguma iniciativa que implementa ou visa implementar o OnTrack no Brasil e se podemos usar instâncias já instaladas.
Figura 33: ok.
\subsection{Colab}
\pp{9} bem interessante. {\bf \color{red} ver sobre as ferramentas e entrar em contato com equipe}.
\noindent Página 58
\section{Estudo de caso: Gabinete Digital do Rio Grande do Sul}
\noindent Página 59
\pp{1} ok.
\pp{2} ok.
\pp{3} ok.
\pp{4} ok.
\pp{5} ok.
\pp{6} ok. {\bf \color{red} procurar e navegar minimamente pelo repo}.
\pp{7} ok.
\noindent Página 60
\subsection{Ferramentas de participação do Gabinete Digital RS}
\subsection{Governador Pergunta}
\pp{8} ok.
\pp{9} muito legal. {\bf \color{red} Temos acesso aos dados desta(s) consulta(s)?}
\pp{10} ok. {\bf \color{red} Bom que esta grande parcela da população reforça a pertinência do uso destes dados para benchmarks e pesquisas científicas.}
\pp{11} Interessante.
\subsubsection{Primeira edição: saúde pública}
\pp{12} muito legal.
\noindent Página 61
Figura 34: ok
\subsubsection{Segunda edição: Segurança no Trânsito}
\pp{13} ok.
\noindent Página 62
Figura 35: ok
\subsubsection{Terceira Edição: Reforma política}
\pp{14} ok.
\pp{15} ok.
\pp{16} ok.
\pp{17} ok. {\bf \color{red} Há opção no PairWise para equivalência entre as ideias? (assim o sistema pode estabelecer propostas equivalentes).}
\pp{18} ok.
\noindent Página 63
\pp{19} ok. Seria interessante fazer plots:
\begin{itemize}
    \item tempo de consulta \emph{versus} número de votos. 
    \item tempo de consulta \emph{versus} número de ideias.
    \item número de votos \emph{versus} número de ideias.
\end{itemize}
\subsubsection{Passos e condições para as consultas}
\pp{20} ok.
\pp{21} ok, muito legal.
\pp{22} ok, muito legal. {\bf \color{red} há ou foi discutida uma página similar para acompanhamento das propostas do arenaNETmundial?}
\noindent Página 64
Figura 36: ok.
\pp{23} ok.
\pp{24} ok.
\noindent Página 65
\pp{25} ok.
\subsubsection{Metodologia e arquitetura de escolha empregadas no Governo Pergunta}
\pp{26} ok.
\pp{27} ok.
\pp{28} ok, muito legal desse código. {\bf \color{red} procurar o repo e navegar minimamente.}
\pp{29} ok. Essa implementação evitou a necessidade de instalar várias instâncias do AllOurIdeas (pairwise?), é isso?
\noindent Página 66
Figura 37: ok.
\subsubsection{Propostas sem mediação}
\pp{30} ok.
\pp{31} ok.
\pp{32} ok.
\noindent Página 67
\pp{33} ok. {\bf \color{red} Mesmo positivo neutralizar grupos de interesse?}.
\pp{34} ok.
\pp{35} Interessante saber que as propostas polêmicas não tem força neste contexto. {\bf \color{red} elaborar formas de equilibrar esse processo?}
Tabela 1: ok.
\pp{36} ok.
\noindent Página 68
\subsubsection{Mobilização e Vans de Participação}
\pp{37} ok.
Figura 38: ok.
\pp{38} ok.
\pp{39} ok.
\noindent Página 69
\pp{40} ok.
\pp{41} ok.
\pp{42} ok.
\subsection{Governador Responde}
\pp{43} ok.
\pp{44} ok.
\pp{45} ok.
\noindent Página 70
\pp{46} ok
Tabela 2: ok.
\pp{47} ok.
\pp{48} ok. Uma pena estar desativada. {\bf \color{red} Seria legal ter acesso a estes dados}
\noindent Página 71
Tabela 3: ok.
Figura 39: ok.
\noindent Página 72
\subsection{Governo Escuta}
\pp{49} ok.
\pp{50} ok.
Tabela 4: ok.
\pp{51} ok.
\pp{52} ok.
\pp{53} ok.
\noindent Página 73
Figura 40: ok.
\subsubsection{Diálogo com 500 mil pessoas}
\pp{54} ok.
\pp{55} {\bf \color{red} Interessante que o número de pessoas nas ruas foi aprox. igual ao de pessoas assistindo do Governador escuta. Interessante que das 500 mil visitações, aprox. 4\% assistiram ao video, que é a parcela esperada de hubs.}
\pp{56} ok.
\subsection{Agenda Colaborativa}
\pp{57} ok.
\noindent Página 74
\pp{58} ok.
Figura 41: ok.
\pp{59} ok
\noindent Página 75
\pp{60} ok.
Tabela 5: ok.
\pp{61} ok.
\pp{62} ok, bem legal estas oficinas.
\pp{63} ok.
\subsection{Monitoramento de obras públicas}
\pp{64} ok.
\pp{65} ok.
Figura: 42: ok.
\noindent Página 77
\subsubsection{Metodologia}
\pp{66} ok.
\pp{67} ok.
\pp{68} ok.
\pp{69} ok.
\subsection{Prêmios alcançados}
\pp{70} ok.
\noindent Página 78
\subsection{Sistema Estadual de Paritcipação Cidadã}
\pp{71} ok.
\pp{72} ok. Legal do site da participação.

\noindent Página 79
\section{Proposta de integração e metodologias para o Participa.br}
\noindent Página 80
\pp{1} ok.
\pp{2} ok.
\pp{3} ok.
\pp{4} ok.
\subsection{Instâncias, mecanismos e ferramentas de participação}
\pp{5} ok.
\pp{6} ok.
\pp{7} ok.
\noindent Página 81
Figura 43: ok. Bem legal.
\pp{8} ok.
\pp{9} ok.
\pp{10} ok. {\bf \color{red} dicernimento útil: reflexivo, não imperativo.}
\pp{11} ok.
\noindent Página 82
Figura 44: ok.
\subsection{Fases das políticas públicas e a participação social}
\pp{12} ok.
\pp{13} ok.
\pp{14} ok. {\bf \color{red} Interessante essa relação entre bottom-up, democracia participativa e legitimidade de política pública.}
\pp{15} {\bf \color{red} Instrumental esta divisão do sebrae.}
\noindent Página 83
\pp{16} ok.
\subsection{Metodologias para mecanismos de participação}
\pp{17} ok.
\pp{18} ok.
\noindent Figura 45: ok.
\pp{19} ok.
\pp{20} ok. {\bf \color{red} Precendente para a foco na categorização dos agentes participativos: citada como um desafio pelo consultor.}
\noindent Página 84
\pp{21} ok.
\subsection{Arquitetura de escolha}
\pp{22} ok.
\pp{23} ok.
\pp{24} ok.
\pp{25} ok. {\bf \color{red} Talvez passar um período lendo sobre nudges \url{http://www.inudgeyou.com}.}
\noindent Página 85
\pp{26} ok. Talvez visitar o livro.
\pp{27} ok. {\bf \color{red} Como pode refletir positivamente: através de uma cartilha dos métodos mais comuns, para instrumentalizar o participante a reconhecer caso haja manipulação indevida.}
\pp{28} ok.
\pp{29} ok.
\pp{30} ok.
\noindent Página 86
\pp{31} ok.
\pp{34} ok.
\subsection{Mecanismos e metodologias recomendados}
\subsubsection{Constituição da agenda}
\pp{35} ok.
\pp{36} ok.
\pp{37} ok.
\pp{38} ok.
\noindent Página 87
\pp{39} ok.
\pp{40} ok.
\pp{41} ok.
\pp{42} ok.
Figura 46: ok.
\noindent Página 88
\subsubsection{Consultas e audiências públicas}
\pp{43} ok.
\pp{44} ok.
\pp{45} ok.
\pp{46} ok. {\bf \color{red} Apontamento para enquetes. Útil caso sejam propostas enquetes que atentem para a diversidade de papéis nas redes sociais.}
\pp{47} ok.
\pp{48} ok.
\pp{49} ok.
\noindent Página 89
Figura 47: ok.
\pp{50} ok.
\subsubsection{Conferências e gestão para conselhos}
\pp{51} ok.
\pp{52} ok.
\pp{53} ok. Tem também as conferências livres de educação, embora n incorporadas ao CONAE como esperado.
\pp{54} ok.
\noindent Página 90
\pp{55} ok.
Figura 48: ok.
\subsubsection{Decisões orçamentárias}
\pp{56} ok.
\pp{57} ok.
\pp{58} ok. {\bf \color{red} ver a ferramenta Delib}.
\noindent Página 91
\subsubsection{Monitoramento de políticas públicas}
\pp{59} ok.
\pp{60} ok.
\pp{61} ok.
\pp{62} ok.
\pp{63} ok.
\pp{64} ok. {\bf \color{red} Na definição usada pelo consultor, ``nós'' são os muito conectados, e os ``hubs'' são os atravessadores. Esta nomenclatura conflita com a da ciência das redes na física, na matemática aplicada e na computação, em que ``hubs'' são os muito conectados e ``nós'' é sinônimo de ``vértices'' (usualmente agentes em redes sociais ou elementos sendo relacionados nestas e outras redes). É útil procurar as fontes do consultor para complementar o ponto de vista usado pela consultoria de dados linkados, redes complexas e processamento de linguagem natural.}
\pp{65} ok.
\noindent Página 92
\pp{66} ok. {\bf Precedente para o uso de critérios de redes para seleção de agentes (p.ex. delegados): o consultor aponta a pertinência destas métricas para complementar os votos.}
\section{Conclusões preliminares}
\pp{1} ok.
\pp{2} ok.
\pp{3} ok.

\chapter{Metodologias de participação, ciclo de políticas públicas e  produção de metadados: integração às trilhas de participação do portal Participa.br}
Este documento contempla uma série de propostas de metodologias de participação mediadas por internet, considerando o ciclo de políticas públicas e sua consequente produção de metadados. O intuito é auxiliar na identificação dos processos decisórios e de participação a partir da plataforma do Participa.br. A proposta também aponta para formas de integração dos processos participativos às chamadas trilhas de participação do portal.
\noindent Página 7
\section{Introdução}
\pp{1} ok.
\pp{2} ok.
\pp{3} ok.
\pp{4} ok.
\pp{5} ok.
\pp{6} ok.
\noindent Página 8
\pp{7} ok. Legal que houve integração com os servidores.
\pp{8} ok.
\noindent Página 9
\section{Ciclo de políticas públicas}
\pp{9} {\bf \color{red} bem legal o triangulo: ciclo de políticas públicas, sistema de participação e trilha participativa.}
\pp{10} ok.
\pp{11} ok.
\pp{12} ok.
\pp{13} que accountability?
\pp{14} ok.
\pp{15} {\bf \color{red} legal as tendências a profissionalização e rarefação dos agentes participativos.}
\noindent Página 10
\pp{16} ok.
Figura 1: Conselhos são prioritariamente para monitoramento e fiscalização? E ouvidoria?
\pp{17} ok.
\pp{18} ok. Interessante dos conselhos e ouvidorias.
\noindent Página 11
\pp{19} ok. {\bf \color{red} ``fugir do setorialismo sem tornar as pautas individualizadas''}
\pp{20} ok.
\pp{21} ok.
\pp{22} ok.
\pp{23} legal dos focos de atenção pelos filtros participativos.
\pp{24} ``consolidar agenda para movimentos sociais''.
\noindent Página 12
\pp{25} dualidade do participa.br: instância de comunidade formal ou por interesse, instância para aflorar demandas individuais pelos processos.
\pp{26} ok. {\bf \color{red} indicador de inclusão na participação social.}
\pp{27} ok.
\pp{28} ok. {\bf \color{red} Declaração de registro histórico com a consulta do arenaNETmundial: 280 mil votos, 200 participações; uma das maiores consultas públicas feitas no Brasil, certamente a maior consulta digital. (internacional?)}
\pp{29} {\bf \color{red} participação digital: novos métodos e melhoras para os existentes}.
\subsection{Trilhas de participação}
\pp{30} ok.
\noindent Página 13
\pp{31} ok.
\pp{32} ok.
\pp{33} ok.
\noindent Página 14
Figura 2: ok.
\noindent Página 15
\pp{34} ok.
\noindent Página 16
Figura 3: ok.
\pp{35} ok.
\pp{36} ok. {\bf \color{red} Pensando em trilhas p o servidor}
\pp{37} ok.
\pp{38} ok.
\pp{39} ok. {\bf \color{red} Apontanda demanda por sistematização dos conteúdos de participação}
\noindent Página 17
\pp{40} ok.
\pp{41} {\bf \color{red} motivação para a sistematização dos conteúdos: evitar sobreposição de pautas e demandas}.
\pp{42} ``balcanização da participação''.
\pp{43} ok.
\pp{44} {\bf \color{red} triângulo: Noosfero incorpora ferramentas participativas; informacionalmente, integra-se com os ciclos participativos e das políticas públicas; metodologicamente, deve ser um canal de uso das ferramentas para as comunidades.}
\noindent Página 18
\subsection{Recomendações para integração das trilhas aos ciclos}
\pp{45} ok.
Figura 4: ok.
\pp{46} ok.
\noindent Página 19
\section{Integração do Participa.br com as mesas de diálogo e monitoramento}
\pp{47} ``interface para mesas de diálogo e monitoramento.''
\pp{48} ok.
\pp{49} ok.
\pp{50} ok. Bem legal das mesas de diálogo (negociação) e de monitoramento.
\pp{51} ok. {\bf \color{red} Demanda de um dashboards de indicadores, para mesas de monitoramento.}
\section{Elementos motivadores}
\pp{52} ok. {\bf \color{red} governança compartilhada através da transparência, prestação de contas e siálogo social}
\pp{53} {\bf \color{red} objs da SNAS: canais participativos não hierarquizados, flexíveis, abertos e que dialógue com as redes.}
\pp{54} O setor está em fase avançada de implementação de um sistema de indicadores, painel de monitoramento e controle de demandas que pode ser integrado ao Partipa.br. {\bf \color{red} onde está este painel? há especificações deste sistema de indicadores?}
\pp{55} {\bf \color{red} ``...auxiliado por processos de dados abertos, provido por uma interface robusta no que tange a aplicação de recursos federais nos diversos níveis, acessível ao cidadão.''}
\pp{56} ok.
\subsection{Recomendações metodológicas}
\pp{57} ``trilha de participação e monitoramento''.
\subsubsection{Estabelecer critérios e requisitos mínimos para a formação de comunidades}
\pp{58} {\bf \color{red} estabelecer claramente que o servidor possui uma trilha de monitoramento que tem como objetivo monitorar o Participa.br. A ``abertura da abertura''.}
\pp{59} ok.
\pp{60} ok.
\subsubsection{Estabelecimento de agenda de participação}
\pp{61} {\bf \color{red} pensar em séries temporais para agendas, além disso, somente os vinculados à SNAS?}
\pp{62} ok.
\noindent Página 22
\pp{63} ok.
\subsubsection{Estabelecer processos dialógicos com prazo definido de retornos}
\pp{64} {\bf \color{red} apontando necessidade de priorizar devolutivas do governo na agenda para melhorar engajamento}.
\pp{65} ok.
\pp{66} ok. {\bf \color{red} procurar os produtos desta consultoria feita para o Departamento de Diálogos Sociais.}
\noindent Página 23
Figura 24: ok. Bastante interessante o ciclo.
\noindent Página 24
\pp{67} {\bf \color{red} proposto sistema de monitoramento para uso interno da SNAS, enquanto o Participa.br p contato com a sociedade.} (isso?)
\pp{68} ok.
\pp{69} {\bf \color{red} apontada pertinência de auxílio na devolutiva do Estado através dos rastros digitais e análise pública deles}.
\subsubsection{Utilização de dados abertos como forma de alimentar indicadores}
\pp{70} {\bf \color{red} Argumento para dados linkados.}
\noindent Página 25
\section{Sugestão para criação de metadados}
\pp{1} ok.
\pp{2} ok.
\pp{3} {\bf \color{red} como assim } ``complementa com mais três além de adicionar outros campos. ''?
\pp{4} ok. (com tabela sem numeração)
\noindent Página 26
\pp{5} legal essa de dc+mpeg 7.
\pp{6} ok.
\subsection{O que classificar?}
\pp{7} ok.
\pp{8} ok. {\bf \color{red} Proposta de metadados para os resultados, trilhas e comunidades}
\subsection{O que é obrigatório e opcional}
\pp{9} {\bf \color{red} como assim?} ``Os campos não devem ter variação quanto à categoria e a tipo dos dados a serem preenchidos''.
\noindent Página 27
\subsection{Vocabulário}
\pp{10} ok.
\subsection{Construindo uma proposta de metadados}
\pp{11} importante {\bf \color{red} apontada pertinência de uma atividade com desenvolvedores, consultores coordenadores e outros envolvidos no projeto, para atividade volta p vocabulário}
\pp{12} legal dos campos para integração com políticas públicas.
Tabela: {\bf \color{red} REFERENCIA: proposta de campos para trilhas participativas.} Integrar em versão próxima da triplificação dos dados do participa.
\noindent Página 28
\pp{13} {\bf pensar na compatibilização da triplificação feita, proposta nesta tabela, e microdados feitos pela Dani.}
\noindent Página 29
\subsection{Integração de metadados, trilhas, ciclos e processos decisórios}
\pp{14} ok.
\pp{15} ok. {\bf \color{red} ocorre-me de fazer um servidor p ação: trabalhar na opa e extração de info dos dados do Participa}
\pp{16} ok.
\pp{17} ok. {\bf \color{red} que grupo de foco?}
\noindent Página 30
\section{Conclusões preliminares}
\pp{1} ok.
\pp{2} ok.
\pp{3} ok. {\bf \color{red} aqui tá ``uma das maiores consultas'' para a do arenaNETmundial}.
\pp{4} ok.
\pp{5} ok. {\bf \color{red} apologia ao sistma de monitoramento encaminhado à DITEC}.


\part{Produtos de Grazielle Machado Fernandes}
\chapter{Plano de trabalho contendo estratégias de relacionamento e mobilização e mapeamento de intelocutores do ``Projeto PARTICIPA''}
\noindent Página 1
\section{Participa.br - Portal Federal de Participação Social}
\pp{1} ok.
\pp{2} ok.
\pp{3} ok.
\pp{4} ok. {\bf \color{red} Necessidade do sistema monitoramento}
\noindent Página 2
Figura: ok. {\bf \color{red} ciclo fundamental encontrado.}
\section{Monitoramento e mapeamento da redes}
\pp{5} ok.
\pp{6} ok.
\noindent Página 3
\section{Identificação de Influenciadores X Gestores Públicos}
\pp{7} {\bf \color{red} previsto SM e contato com agentes detectados e considerados estratégicos}
\pp{8} essa abertura de dados e análise vinga como ideia pois está ancorada na realidade, portanto é útil e instrumental. {\bf \color{red} este momento aponta para o monitoramento interno.}
Figura: ok. {\bf \color{red} mais um ciclo fundamental.}
\section{Mapeamento e abordagem de interlocutores}
\pp{9} ok.
\pp{10} {\bf \color{red} Trecho descreve trilhas criadas via colaboração de agentes civis (escuta/monitoramento nas redes sociais) e gov (monitoramento interno)}.
\noindent Página 4
\pp{11} ok e previsto como responsabilidade da equipe do Participa.br.
\pp{12} ok. {\bf \color{red} estas regras são campos já presentes nas trilhas? Colocar na OPA}
\pp{13} ok.
\section{Ações previstas para realização até dezembro de 2014}
\pp{14} ok.
\pp{15} ok.
\pp{16} ok.
\pp{17} ok. {\bf \color{red} para a difusão do Participa.br e termo \#ParticipaBR, escolher agentes para fazerem as 3 rodadas de difusão nas redes deles, ou comunicação única através de atravessadores e mais próximos.}
\section{Ações a serem executadas para atigir os objetivos propostos}
\pp{18} ok.
\noindent Página 5
\pp{19} ok. {\bf \color{red} muito jóia esta proposta de mobilizar os blogueiros.}
\pp{20} ok.
\pp{21} ok.
\pp{22} ok.
\pp{23} ok.
\pp{24} ok.
\pp{25} ok. {\bf\color{red}já rolou este ``todo mundo participa''}?
\pp{26} ok.
\noindent Página 6
\pp{27} ok.
\pp{28} ok. {\bf \color{red} ponte para o govern-art}.
\pp{29} ok.
\section{Dinâmina da blogagem coletiva}
\pp{1} ok.
\pp{2} ok.
\pp{3} ok.
\pp{4} ok.
\noindent Página 7
\pp{5} ok.
\section{Dinâmica de campanha Todo Mundo Participa}
\pp{1} ok.
\pp{2} ok.
\pp{3} ok.
\pp{4} ok.
\pp{5} ok.
\pp{6} ok.
\pp{7} ok.
\section{Anexos}
\chapter{Produto 2}
\section{Introdução}
\pp{1} ok.
\pp{2} ok.
\pp{3} ok.
\pp{4} ok.
\pp{5} ok.
\pp{6} ok.
\section{O que é o Participa.br}
\pp{1} ok.
\noindent Página 2
\pp{2} ok.
\section{Identificação de mobilização de públicos}
\pp{1} ok.
\pp{2} ok.
\pp{3} ok.
\pp{4} ok.
\pp{5} ok.
\noindent Página 3
Figura: ok.
\subsection{Público interno (agentes e órgãos públicos)}
\pp{6} ok.
\pp{7} ok.
\subsubsection{Comunidades com processos de participação em curso}
\noindent Página 4
\pp{8} ok.
\pp{9} ok.
\pp{10} ok.
\pp{11} ok.
\pp{12} ok.
\pp{13} ok.
\pp{14} ok.
\pp{15} ok.
\subsection{Público externo (sociedade civil)}
\pp{16} ok. {\bf \color{red} manifesta busca pelos que não dialogam com o estado}
\noindent Página 5
\section{Instâncias e mecanismos de participação}
\pp{1} ok.
\pp{2} ok.
\pp{3} ok.
\pp{4} ok.
\pp{5} ok. {\bf \color{red} manifesta intensão de massificar a participação social}.
\noindent Página 6
Figura: ok.
\subsection{Exemplos de práticas que podem ser incorporadas ao Participa.br}
\subsubsection{Gabinete Digital RS}
\pp{6} ok.
\noindent Página 7
\pp{7} ok.
\subsubsection{Gabinete Digital de Caruaru}
\pp{8} ok.
\pp{9} ok.
\subsubsection{Webcidadania Xingu - Cidade Democrática}
\pp{10} {\bf \color{red} visitar a iniciativa para ver as inspirações e as propostas encaminhadas}
\pp{11} ok.
\noindent Página 8
\section{Mobilização social x Engajamento de agentes públicos}
\pp{12} ok. {\bf \color{red} ``diretrizes e prioridades de políticas públicas são os principais objetivos das conferências''}.
\pp{13} ok. {\bf pensando em retomar as ontologias dos mecanismos participativos em si.}
\pp{14} ok.
\subsection{Metodologia e cronograma de trabalho}
\pp{15} ok.
\pp{16} ok.
\pp{17} {\bf \color{red} este levantamento ocorreu? Foi colocado na plataforma?}
\noindent Página 9
\pp{18} ok.
\section{Relacionamento, o que diferencia o Participa.br}
\pp{1} ok.
\pp{2} ok.
\pp{3} ok.
\pp{4} ok.
\pp{5} ok. {\bf o SM pode ajudar principalmente no escutar e no mensurar}.
\noindent Página 10
Figura: ok.
\section{Anexos}
ok.
\chapter{Produto 3 - Documento contendo estratégias de mobilização de servidores e órgãos públicos em conjunto com a base social mobilizada em torno do Portal da Participação Social}
\section{Introdução}
\pp{1} ok.
\pp{2} ok.
\pp{3} ok.
\pp{4} ok.
\pp{5} ok.
\pp{6} ok.
\section{Como o Participa.br pode ampliar as formas de comunicação}
\pp{7} ok.
\noindent Página 2
\pp{8} ok.
\section{Mobilizando a base social em torno do Portal da Participação Social}
\pp{9} ok. {\bf \color{red} inserção nas diversas redes: dados linkados em todas estas instâncias}
\pp{10} ok.
\pp{} ok.
\pp{} ok.
\pp{} ok.
\subsection{ArenaNETmundial e a maior consulta pública realizada na internet}
\subsubsection{Consulta Pública}
\pp{} ok.
\noindent Página 3
\pp{} ok.
\pp{} ok.
\pp{} ok.
\pp{} ok.
\pp{} ok.
\subsubsection{ArenaNETmundial ParticipaBR}
\pp{} ok.
\noindent Página 4
\pp{} ok.
\pp{} ok.
\pp{} ok.
\pp{} ok.
\pp{} ok.
{\bf Cobertura Colaborativa}
\pp{} ok.
\pp{} ok.
\pp{} ok.
{\bf Parceria com universidades}
\noindent Página 5
\pp{} ok.
\pp{} ok.
\pp{} ok.
\pp{} ok.
\pp{} ok.
\pp{} ok.
\pp{} ok.
\noindent Página 6
Figura: ok.
\subsection{Hangout}
\pp{} ok.
\pp{} ok.
\noindent Página 7
\section{Mobilizando servidores e órgãos públicos}
\pp{} ok.
\pp{} ok.
\subsection{Reunião com gestores}
\pp{} ok.
\pp{} ok.
\subsection{Treinamento}
\pp{} ok.
\pp{} ok.
\pp{} ok.
\subsection{Interação nas redes sociais}
\pp{} ok.
\noindent Página 8
\pp{} ok.
\subsection{Criação de comunidades}
\pp{} ok.
\pp{} ok.
\subsection{Trilhas de participação}
\pp{} ok.
\pp{} ok.
\section{Comunidades com gestão estabelecida por órgão da administração}
\pp{} ok.
\noindent Página 9
\pp{} ok.
\pp{} ok.
\pp{} ok.
\pp{} ok.
\pp{} ok.
\section{Anexos}
ok.

\part{Produtos de Joênio Marques da Costa}
\chapter{Documento com guia de codificação "coding guidelines" para o desenvolvimento do código do portal objetivando o reaproveitamento de código e o fomento à formação de comunidades em torno dos módulos, bem como tutoriais para implementação local das Soluções.}
\section{Apresentação}
\pp{} ok.
\pp{} ok.
\section{Comunidade Noosfero}
\pp{} ok.
\pp{} ok.
\pp{} ok.
\pp{} ok.
\pp{} ok.
\pp{} ok.
\pp{} ok.
\pp{} ok.
Figura 1: ok. Gostaria do link do oloh do noosfero.
\section{Desenvolvimento do Noosfero}
\pp{} ok.
\pp{} ok.
\noindent Página 8
\pp{} ok.
\pp{} ok.
Figura 2: ok.
\pp{} ok.
Figura 3: ok.
\pp{} ok.
\pp{} ok.
\noindent Página 9
\pp{} ok.
\subsection{Instalação, Execução, Desenvolvimento}
\pp{} ok.
\pp{} ok.
\pp{} ok. {\bf embora tenha q testar o schroot (nunca usei)}.
\section{Orientações para contribuição}
\pp{} ok.
\pp{} ok.
\noindent Página 10
\pp{} ok.
\pp{} ok.
\subsection{Guia de codificação}
\pp{} ok.
\subsubsection{Princípios}
\pp{} ok.
\pp{} ok.
\pp{} ok.
\pp{} ok.
\pp{} ok.
\pp{} ok.
\noindent Página 11
\pp{} ok.
\pp{} ok.
\subsubsection{Models}
\pp{} ok.
\noindent Página 12
\pp{} ok.
\subsubsection{Views}
\pp{} ok, embora eu não lembre o que seja um Partial.
\pp{} ok.
\subsubsection{Testes}
\pp{} ok.
\pp{} ok.
\pp{} ok.
\pp{} ok.
\pp{} ok.
\noindent Página 13
\subsubsection{Traduções}
\pp{} ok.
\pp{} ok.
\pp{} ok.
\subsubsection{JavaScript}
\pp{} ok.
\pp{} ok.
\noindent Página 14
\pp{} ok.
\subsubsection{CSS}
\pp{} ok.
\pp{} ok.
\noindent Página 15
\pp{} ok.
\pp{} ok.
\pp{} ok.
\subsection{Submissão de código}
\pp{} ok.
\pp{} ok.
\pp{} ok.
\noindent Página 16
\pp{} ok.
\pp{} ok.
\pp{} ok.
Figura 4: ok.
\noindent Página 17
\pp{} ok.
\pp{} ok.
\pp{} ok.
\pp{} ok.
\pp{} ok.
\pp{} ok.
\section{Convenções de codificação}
\pp{} ok.
\noindent Página 18
\pp{} ok.
\pp{} ok.
\pp{} ok.
\pp{} ok.
\pp{} ok.
\pp{} ok.
\noindent Página 19
\pp{} ok.
\section{Plugins Noosfero}
\pp{} ok.
\subsection{Ativação de plugins}
\pp{} ok.
\pp{} ok.
\pp{} ok.
\pp{} ok.
\noindent Página 20
\pp{} ok.
\subsection{Estrutura Básica}
\pp{} ok.
\pp{} ok.
\pp{} ok.
\subsection{Visão do desenvolvedor do Noosfero}
\pp{} ok.
\subsubsection{Definindo hotspots}
\pp{} ok.
\noindent Página 21
\pp{} ok.
\pp{} ok.
\pp{} ok.
\pp{} ok.
\pp{} ok.
\noindent Página 22
\pp{} ok.
\pp{} ok.
\pp{} ok.
\noindent Página 23
\pp{} ok.
\pp{} ok.
\pp{} ok.
\pp{} ok.
\pp{} ok.
\subsection{Visão do desenvolvedor de Plugins}
\pp{} ok.
\subsubsection{Como criar um Plugin Noosfero}
\pp{} ok.
\pp{} ok.
\noindent Página 24
\pp{} ok.
\subsubsection{Definição}
\pp{} ok.
\pp{} ok.
\pp{} ok.
\noindent Página 25
\subsubsection{Plugins e extensão do core}
\pp{} ok.
\pp{} ok.
\pp{} ok.
\pp{} ok.
\noindent Página 26
\pp{} ok.
\pp{} ok.
\pp{} ok.
Extendendo classes do core
\pp{} ok.
\pp{} ok.
\pp{} ok.
\noindent Página 27
\pp{} ok.
\pp{} ok.
\pp{} ok.
\pp{} ok.
\pp{} ok.
\subsubsection{Tabelas e registros}
\noindent Página 28
\pp{} ok.
\noindent Página 29
\pp{} ok.
\pp{} ok.
\pp{} ok.
\pp{} ok. {\bf \color{red} ``lib/mezuro\_plugin/project.rb'' dentro da pasta do plugin, certo?}
\noindent Página 30
\pp{} ok.
\subsubsection{Controllers e rotas}
\pp{} ok.
\pp{} ok.
\pp{} ok.
\pp{} ok.
\noindent Página 31
\pp{} ok.
\pp{} ok.
\pp{} ok.
\pp{} ok.
\pp{} ok.
\pp{} ok.
\noindent Página 32
\pp{} ok.
\pp{} ok.
\pp{} ok.
\pp{} ok.
\pp{} ok.
\subsubsection{Views}
\pp{} ok.
\pp{} ok.
\noindent Página 33
\pp{} ok.
\pp{} ok.
\pp{} ok.
\pp{} ok.
\pp{} ok.
\pp{} ok.
\noindent Página 34
\pp{} ok.
\subsubsection{Arquivos públicos}
\pp{} ok.
\pp{} ok.
\pp{} ok.
\pp{} ok.
\subsubsection{Testes}
\pp{} ok.
\pp{} ok.
\noindent Página 35
\pp{} ok.
\subsubsection{Dependencias and Instalação}
\pp{} ok.
\pp{} ok.
\pp{} ok.
\pp{} ok.
\section{Comunidade Participa.br}
\pp{} ok.
\pp{} {\bf \color{red} visitar o DisplayContextContent e ver se cabe ali os dados linkados, medidas de texto, de redes e usuários e infos de acesso à API e maiores análises}. Outra: {\bf \color{red} adicionar funcionalidades no pgSearchPlugin.} Outro: {\bf \color{red} visitar o statistics plugin e ver o que cabe ali de pln e rc}.
\noindent Página 36
\pp{} ok.
\section{Fomento à formação de comunidade de desenvolvimento}
\pp{} {\bf \color{red} houve algum destes eventos ``mão na massa'' para fomentar comunidade de desenvolvedores?}
\section{Considerações finais}
\pp{} ok.
\pp{} ok.
\pp{} ok.
\chapter{Documento com análise de arquiteturas de sistemas de identidade distribuída, estratégia de implantação considerando os sites parceiros e contendo propostas de códigos}
\PP{6}
\section{Apresentação}
\pp{} ok.
\pp{} ok.
\section{SSO - Single Sign-on}
\subsection{O que é SSO?}
\pp{} ok.
\pp{} ok.
\pp{} ok. \VV{SM pode precisar de administração robusta se for servidor de SSO.}
\PP{7}
\subsection{SSO no Noosfero}
\pp{} ok.
\pp{} ok.
\pp{} ok.
\subsection{Qual problema SSO resolve?}
\pp{} ok.
\pp{} ok.
\pp{} \VV{legal do web messaging}. ok.
\subsection{Como SSO funciona?}
\pp{} ok.
\PP{8}
\pp{} ok.
Figura 1: ok.
\subsection{Quais soluções de SSO existem?}
\pp{} ok.
\subsubsection{Accounts \& SSO}
\pp{} ok.
\PP{9}
\subsubsection{Central Authentication Service (CAS)}
\pp{} ok.
\pp{} ok.
\pp{} ok.
\pp{} ok.
\pp{} ok
\pp{} ok.
\PP{10}
Figura 2: ok.
\subsubsection{Distributed Access Control System (DACS)}
\pp{} ok.
\pp{} ok.
\pp{} ok.
\PP{11}
\pp{} ok.
\subsubsection{Enterprise Sign On Engine}
\pp{} ok.
\pp{} ok.
\pp{} ok.
\pp{} ok.
\subsubsection{FreeIPA}
\pp{} ok.
\pp{} ok.
\pp{} ok.
\pp{} ok.
\subsubsection{IBM Enterprise Identity Mapping}
\pp{} ok.
\pp{} ok.
\pp{} ok.
\PP{12}
\subsubsection{JBoss SSO}
\pp{} \VV{impressão ou algumas das soluções para login unificado também se propõem a integrar logins diferentes, de terceiros, em outros padrões?}
\pp{} ok.
\pp{} ok.
\pp{} ok.
\subsubsection{JOSSO}
\pp{} ok.
\pp{} ok.
\pp{} ok.
\subsubsection{Kerberos}
\pp{} ok.
\pp{} ok.
\pp{} ok.
\PP{13}
Figura 3: ok.
\subsubsection{OpenAM}
\pp{} ok.
\pp{} ok.
\pp{} ok.
\pp{} ok.
\subsubsection{Pubcookie}
\pp{} ok.
\PP{14}
Figura: ok.
\pp{} ok.
\pp{} ok.
\subsubsection{SAML}
\pp{} ok.
\pp{} ok.
\pp{} ok
\PP{15}.
Figura 5
\pp{} ok.
\pp{} ok.
\subsubsection{Shibboleth}
\pp{} ok.
\pp{} ok.
\pp{} ok.
\PP{16}
\subsubsection{ZXID}
\pp{} ok.
\pp{} ok.
\pp{} ok.
\section{IdP - Identity Provider}
\subsection{O que é IdP}
\pp{} ok.
\pp{} ok.
\pp{} ok.
\pp{} ok.
\pp{} ok.
\pp{} ok.
\PP{17}
\subsection{IdP no Noosfero}
\pp{} ok.
\subsection{Qual problema IdP resolve?}
\pp{} ok.
\subsection{Quais soluções de IdP existem?}
\subsubsection{Mozilla Persona}
\pp{} ok.
\pp{} ok.
\pp{} ok.
\pp{} ok.
\pp{} ok. \VV{No caso da participação social, é melhor que toda a atividade seja rastreável, não?}
\PP{18}
\subsubsection{OAuth}
\pp{} ok.
\pp{} ok.
\pp{} ok.
\pp{} ok.
\pp{} ok.
\subsubsection{OpenID}
\pp{} ok.
\pp{} ok.
\pp{} ok.
\pp{} ok.
\pp{} ok.
\subsubsection{OpenID Connect}
\pp{} ok.
\PP{19}
\pp{} ok.
\pp{} ok.
\section{Outras iniciativas}
\subsection{OpenAthens - Reino Unido}
\pp{} ok.
\pp{} ok.
\subsection{Microsoft account}
\pp{} ok.
\subsection{Liberty Alliance}
\pp{} ok.
\PP{20}
\pp{} ok.
\section{Discussão}
\subsection{Iniciativas (Governo e Comunidade)}
\subsubsection{Login Cidadão}
\pp{} ok.
\pp{} ok.
\pp{} ok.
\pp{} ok.
\subsubsection{Id da Cultura}
\pp{} ok.
\pp{} ok.
\PP{21}
\subsection{Proposta para o Participa.BR}
\pp{} ok. {\bf \color{red} citados o Participatório, o Cidade Democrática e o Cultura Educa. Destes, só não tenho dados do cultura educa para triplificar.}
\pp{} ok.
\pp{} ok.
\pp{} ok.
\pp{} ok.
\pp{} ok.
\pp{} ok.
\PP{22}
\pp{} ok.
\pp{} ok.
\section{Considerações finais}
\pp{} ok.
\pp{} ok.
\chapter{Proposta de ferramentas de participação social:
Ferramenta de elaboração e
discussão de propostas e ferramenta de votação em pares para o Participa.br}

\pp{} 
\pp{} 
\pp{} 
\pp{} 
\pp{} 
\pp{} 
\pp{} 
\pp{} 
\pp{} 
\pp{} 
\pp{} 
\pp{} 
\pp{} 
\pp{} 
\pp{} 
\pp{} 
\pp{} 
\pp{} 
\pp{} 
\pp{} 
\pp{} 
\pp{} 
\pp{} 
\pp{} 
\pp{} 
\pp{} 
\pp{} 
\pp{} 
\pp{} 
\pp{} 
\pp{} 
\pp{} 
\pp{} 
\pp{} 
\pp{} 
\pp{} 
\pp{} 
\pp{} 
\pp{} 
\pp{} 
\pp{} 
\pp{} 
\pp{} 
\pp{} 
\pp{} 
\pp{} 
\pp{} 
\pp{} 
\pp{} 
\pp{} 
\pp{} 
\pp{} 
\pp{} 
\pp{} 
\pp{} 
\pp{} 
\pp{} 



\part{Notas aleatórias}

\chapter{Atribuição de papéis}
Os intermediários podem ser chamados a avaliarem a produção da comunidade, pois não são víscerais como os hubs nem lateral como os periféricos.

Periféricos podem ser chamados para darem substantivos, conceitos, coisas.
Hubs convocados para qualificar estas coisas. Intermediários para formarem texto(s).
\chapter{Simulação de processos}

Difusão, contágio, acessibilidade e fofoca nas redes reais do Participa.br e de outras instâncias.

\chapter{Criação de benchmarks participativos}
Com medidas de experiências participativas (quantidade de participantes, médias, interações, etc)

\chapter{Triplificação dos dados participativos}
Recomendar p equipe de dev do noosfero que gerem URLs para acessar comentários específicos e outros recursos em conformidade com as URIs usadas na triplificação dos dados do Participa.br.

Triplificar os dados do Cidade Democrática e do Participatório.

Triplificar os dados da consulta do arenaNETmundial.

Triplificar os dados das experiências do RS.

Triplificar Tweets participativos meus, do Malini do Pimentel e de outros.


\newpage
\bibliography{bibliografia}
\newpage
%\listoffigures
\input{listadeabreviaturas.tex}
\newpage
\printindex
\newpage
%\input{listadeanexos.tex}
\appendix
\end{document}
