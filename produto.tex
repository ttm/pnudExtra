\documentclass[12pt]{report}
\usepackage[usenames,dvipsnames]{color}
\usepackage{listings}
\usepackage{graphicx}
\usepackage{fancyhdr}
\usepackage{framed}
\usepackage[T1]{fontenc}
\usepackage[toc,page]{appendix}
\usepackage[utf8]{inputenc}
\usepackage[brazil]{babel}
\usepackage{fancyvrb}
\usepackage[hmargin=2cm,vmargin=2cm]{geometry}
\usepackage{lastpage}
\usepackage{pdfpages}
\usepackage{makeidx}
\usepackage{hyperref}
\pagestyle{fancy}
\usepackage{enumitem}
% cabecalho e rodapé
\setlength{\headheight}{120pt}
\setlength{\textheight}{550pt}
\renewcommand{\headrulewidth}{0pt}
\lhead{\includegraphics[scale=0.03]{brasao.png}}
\chead{\includegraphics[scale=0.5]{logo-brasil-sem-pobreza2.png}}
\rhead{\includegraphics[scale=0.5]{logo-pnud.png}}
\cfoot{\textbf{\ProjectCode\ - Inovando a democracia participativa}}
\rfoot{\thepage}

\hyphenation{par-ti-ci-pa-ção}
\bibliographystyle{ieeetr}

% definições sobre o autor e o produto
\newcommand{\MyName}{Renato Fabbri}
\newcommand{\MySurnameForename}{Fabbri, Renato}
\newcommand{\SupervisorName}{Gabriella Vieira Oliveira Gonçalves}
\newcommand{\MyEmail}{renato.fabbri@gmail.com}
\newcommand{\ContractNumber}{2013/000566}
\newcommand{\ContractYear}{2014}
\newcommand{\ProjectCode}{Projeto BRA/12/018}
\newcommand{\NomeSecretaria}{Secretaria-Geral da Presidência da República}
%Q\newcommand{\SiglaSecretaria}{SG/PR}
\newcommand{\SiglaSecretaria}{Secretaria: SNAS }
\newcommand{\ProductNumber}{Extra}
\newcommand{\ProductTitle}{Anotações sobre a leitura de produtos dos outros consultores do Projeto BRA/12/018}
\newcommand{\ProductSubtitle}{confluências, aplicações, dúvidas, sugestões, correções}
\newcommand{\ProductDescription}{"Ferramentas assistidas de categorização de conteúdo: Com Processamento de Linguagem Natural e de Redes Complexas, adaptadas para o ambiente do portal de participação."
}

\newcommand{\ProductValue}{R\$ 0,00 (zero reais e zero centavos)}
\newcommand{\ObjetoContratacao}{
Aporte de conhecimentos e tecnologias para especificação de vocabulário e ferramentas assistidas que utilizam processamento de linguagem natural e análise de redes complexas para o conteúdo do portal da participação social.
}
\newcommand{\DataEntrega}{Agosto de 2014}
\newcommand{\PalavrasChave}{reconhecimento de padrões, redes complexas, processamento de linguagem natural, participação social}
\newcommand{\pp}[1]{

\textbf{Parágrafo #1:}
}
\newcommand{\p}[0]{

\textbf{Parágrafo:}
}
\newcommand{\PP}[1]{

\noindent Página #1

}
\newcommand{\VV}[1]{{\bf \color{red} #1}}

\newcommand{\mysection}[2]{
    \setcounter{section}{#1}
    \addcontentsline{toc}{section}{#2}
}

% lista de abreviações
\makeindex

\begin{document}

\input{folhaderosto.tex}
\input{folhadeaprovacao.tex}
\input{folhadeidentificacao.tex}
\tableofcontents
\newpage


\begin{abstract}
Este documento registra a reflexão sobre produtos dos outros consultores do mesmo projeto (BRA/12/018).
Cada consultor entrega alguns ``produtos'' para os órgãos interessados. No caso, são documentos escritos,
que relatam atividades, pesquisas, propostas, enfim, o que for pertinente para o trabalho.
Na leitura destes documentos, são registradas anotações pertinentes à consultoria de contrato 2013/000566, sob responsabilidade
do autor deste produto extra. Cada parte do documento corresponde a um consultor, cada capítulo corresponde
a um documento/produto, as seções e subseções correspondem à estrutura original de cada documento considerado.\\

{\bf Palavras-chave:} \PalavrasChave.
\end{abstract}
\newpage
\part{Produtos de Fabricio Solagna}
\chapter{Análise de experiências nacionais e internacionais de participação mediada por Internet}
Este documento contempla uma análise de experiências nacionais e internacionais de
participação mediada por Internet considerando os aspectos de inclusão, inovação e
deliberação, com a finalidade de gerar uma matriz de características para integração ao portal
Participa.br. As iniciativas analisadas foram agrupadas em 5 capítulos temáticos, somando 18
iniciativas e mais de 30 ferramentas de participação digital. O capítulo final apresenta
recomendações metodológicas à luz das experiências estudadas.

\section{Introdução}

Página 10

\pp{1} ok.

\pp{2} ok. Unipresente deve ser Omnipresente ou Ubíqua.

\pp{3} ok.

\pp{4} Legal da definição do Macintosh.

\pp{5} Porquê esta visão de participação social se opõe a esta capacidade do indivíduo influenciar os processos? É quanto à visão do Macintosh?

\pp{6} {\bf \color{red} Motivação para o SM essa inspiração através do empírico nas redes sociais}.

\pp{7} Legal a definição de ``gift economics'' como processos nos quais atores cooperam em projetos maiores que as corporações.

\pp{8} Ok.

\noindent Página 11

\pp{9} Dada a função da OGP, é pertinente observar quais os posicionamentos e recursos que exibem quanto à web semântica.

\pp{10} Legal a necessidade da ferramenta de participação de delimitar \emph{input} e \emph{output}.

\pp{11} Muito interessante. Esquematicamente: políticas públicas equilibra: necessidades da comunidade, recursos disponíveis e capacidade de execução. Participação pela internet: há hiato entre capacidade de coletar opinião e sistematização para ação. Congregando colaboração massiva, qualidade e resultado prático, cita: Wikipédia, o Software Livre e o Crowdfunding.

\pp{12} Interessante que nos gov abertos essas iniciativas tiveram pouco impacto.

\noindent Página 12

\pp{13} ok.

\pp{14} Legal da suavização da separação digital e presencial.

\pp{15} ok.

\pp{16} Interessante: inovação, inclusão e protagonismo; deliberação, promoção, construção de políticas públicas.

\pp{17} foco no executivo e na fiscalização de serviço público. Legal do longo histórico com o Legislativo.

\pp{18} ok. Cap 2 -> Democracia digital ligada aa PR nos EUA, Chile, Bolívia e País de Gales.

\noindent Página 13

\pp{19} Cap 3 -> visualização e deliberação de Orçamento Público.
Legal da simulação e visualização pela popularização dos datasets. Seria bom ver quais as iniciativas do OKF caso não esteja neste cap.

\pp{20} Cap4 -> projs/entidades com relevância na democracia digital.

\pp{21} Cap5 -> monitoramento e fiscalização de obras e ações do governo.

\pp{22} Cap6 -> Gabinete Digital do RS. Três maiores consultas nacionais ou internacionais?

\pp{23} Cap7 -> Recomendações de metodologias e ferramentas.

\pp{24} Legal dos três vértices: elaboração, fiscalização e monitoramento.

\noindent Página 14 

\pp{25} Ok.

\section{Experiências de democracia digital ligadas
diretamente a presidência da república}
Página 15
\subsection{Chile: Gobierno Abierto}
Página 16

Figura 1: ok.

\pp{1} ok.

\pp{2} ok.

\pp{3} ok.

\noindent Página 17

Figura 2: ok

\noindent Página 18

\pp{4} ok

\pp{5} ok

\pp{6} ok. Interessante essas consultas que já passaram estarem acessíveis junto a conteúdo jornalístico e multimídia.

\noindent Página 19

Figura 4: ok.

\pp{7} ok.

\noindent Página 20

\subsubsection{Resultados alcançados}

\pp{8} ok.

\subsection{Bolívia: Urna de Cristal}

\indent Figura 5: ok.

\noindent Página 21

\pp{1} ok.

\subsubsection{Metodologia e arquitetura de escolha}

\pp{2} Legal da proposta ou pergunta e então divulgação nas redes sociais.

\pp{3} ok.

\subsubsection{Resultados alcançados}

\pp{4} Interessante o envolvimento das equipes nas redes sociais. {\bf \color{red} Talvez valha eu procurar melhor como fazem isso.}

\noindent Página 22

Figura 6: ok.

\subsection{EUA: Iniciativas de diálogo com o Governo}

\pp{1} ok.

\pp{2} ok.

\pp{3} ok.

\noindent Página 23

\subsubsection{Open for Question}

\pp{4} Legal do envio de questões. Interessante a ambiguidade do texto, que deixou a sugestão de fazer uma relação de pessoas cadastradas para receberem a pergunta.

Figura 7: ok.

Metodologia

\pp{5} bacana a ideia de resposta em video pelo presidente.

\subsubsection{AskObama on Twitter}

\pp{6} super da equipe do Twitter junto.

\noindent Página 24

Figura 8: ok.

\subsubsection{Reddit}

\pp{7} ok.

\noindent Página 25

Figura 9: ok.

\subsubsection{We the People}

\pp{8} Interessante de petição estar integrado ao site da casa branca. {\bf \color{red} Talvez pensar em algo assim para o Participa.br, fornecendo ao usuário meios sofisticados de ativar suas redes sociais.}

Metodologia

\pp{9} ok.

\pp{10} ok.

\pp{11} Legal das respostas quando a petição tem mais de 100 mil assinaturas.

\noindent Página 26

Resultados

\pp{12} ok.

Figura 10: ok.

\subsection{País de Gales: E-petitions}

\pp{13} ok.

\noindent Página 27

Figura 11: ok.

\noindent Página 28

\section{Experiências de visualização e deliberação de Orçamento Público}

\noindent Página 29

\subsection{Orçamento Participativo (OP)}

\pp{1} ok dos três períodos do OP.

\pp{2} Dúbia a construção? Assumindo aqui que a segunda fase é situada entre 1989 e 1992 enquando a primeira fase vai de 1960 até 1998. (?)

\pp{3} OP é criação do PT? (creio que não, mas o texto está me dando a entender isso)

\pp{4} Interessante a mescla de OP com outras iniciativas de democracia digital.

\pp{5} ok.

\pp{6} Bem curiosa essa ausência de OP federal e pouca estadual.

\pp{7} ok.

\noindent Página 30

\subsection{Orçamento Participativo Digital – Belo Horizonte}

\pp{8} ok.

Figura 12: ok.

\pp{9} ok.

\subsubsection{Metodologia do OP Digital}

\pp{10} ok.

\pp{11} ok, bem legal da votação via ligações telefonicas e via app de celular.

\noindent Página 31

\pp{12} ok

Figura 13: ok.

\pp{13} ok.

\subsubsection{Resultados}

\pp{14} ok.

\noindent Página 32

\pp{15} ok.

\subsection{Nova Iorque: Participatory Budgeting in New York City}

\pp{16} ok.

\subsubsection{Metodologia e arquitetura de escolha}

\pp{17} ok.

\noindent Página 33

Figura 14: ok.
Figura 15: ok.

\noindent Página 34

\subsubsection{Orçamento Participativo em outras cidades norte-americanas}

\pp{18} ok.

\pp{19} ok.

\subsection{Liverpool: Budget Simulator}

\pp{20} que tendência?

Figura 16: ok.

\subsection{Stabilize the U.S. Debt}
\noindent Página 35
\pp{21} ok.
\pp{22} ok.
\pp{23} {\bf \color{red} Gamificação}.
\pp{24} ok.

Figura 17: ok.

\noindent Página 36

\subsection{Where Does My Money Go}
\pp{25} ok.
\pp{26} ok.
\pp{27} ok.
\pp{28} ok.
\pp{29} ok.

Figura 18: ok.

\noindent Página 37

\subsubsection{Metodologia utilizada}
\pp{30} ótimo das techs livres.

Figura 19: ok.

\subsection{Aplicações no Brasil}
\subsubsection{Para onde foi meu dinheiro}
ok tudo.
\subsubsection{Orçamento ao seu alcance}
Bem legal sobre o orçamento federal, até porque não há OP federal.

\section{Projetos de democracia digital oriundos da sociedade civil}

\noindent Página 42

\subsection{PortoAlegre.cc}
\pp{1} ok.
\pp{2} ok. Quais conceitos de inteligência social?
\pp{3} ok.
\subsubsection{Metodologia e arquitetura de escolha}
\pp{4} ok.
\pp{5} ok.
\pp{6} Super legal dos pilares: cultura de cidadania, ética do cuidado, corresponsabilização cidadã e engajamento cívico.
\pp{7} 

\noindent Página 43

Figura 23: ok.

\pp{8} ok.
\pp{9} Interessante a confluência de pautas levantadas pela sociedade com agendas públicas.
\pp{10} 

\noindent Página 44

Figura 24: ok.

\subsubsection{Fases de uma causa}
\pp{11} ok.
\pp{12} ok.
\pp{13} ok.
\pp{14} ok.

\subsubsection{Resultados}
\pp{15} ok.
\noindent Página 45
\subsection{MySociet}
\pp{16} ok da UKCOD.
\pp{17} só de transportes?
\pp{18} ok. Legal que trabalham sob encomenda, talvez fazer contato em algum momento ou trazer para uma consultoria presencial.
\pp{19} ok, lembra o lm e orgs de gsoc.
\pp{20} ok.

\subsubsection{Alaveteli}
\pp{21} legal. Em que países são ou foram usados?
\noindent Página 46
\pp{22} ok.
\pp{23} ok.
Figura 25: ok.
\pp{24} ok.
\pp{25} ok.
\noindent Página 47
\subsubsection{FixMyStreet}
\pp{26} ok.
\pp{27} Figura 26: ok.
\pp{28} ok.
\noindent Página 48
\pp{29} ok.
\subsubsection{Eu prometo}
\pp{30} ok. Mandar p alguns amigos.
Figura 27: ok.
\pp{31} ok.
\noindent Página 49
\subsection{SeeClickFix}
\pp{32} ok.
\pp{33} ok.
\pp{34} ok.
Figura 28: ok.
\noindent Página 50
\subsection{All Our Ideas}
\pp{35} ok.
\pp{36} ok.
\pp{37} bem legal. Belo parágrafo sobre All Our Ideas e muito do que a participação social trata no momento.
\subsubsection{Metodologia empregada}
\pp{38} ok.
\noindent Página 51
Figura 29: ok.
\pp{39} ok. Dá vontade de analisar os dados de uma consulta com o Allourideas, também de ver como funciona, e caracterizar estatísticamente. Talvez também propor dinâmicas em rede para melhor coleta de ideias e votos.
\pp{40} ok.
\noindent Página 52
\pp{41} ok. Seria interessante um parâmetro para esse "bastante fiel".
\pp{42} ok.
\pp{43} ok.
\subsection{Cidade Democrática}
\pp{44} ok.
\pp{45} ok. Propor no Cidade Democrática de todo estudante de curso superior público dê 1h de aula gratuita por semana em comunidades menos favorecidas.
\pp{46} ok. Muito legal do sonho p Xingu.
\noindent Página 53
Figura 30: ok.
\noindent Página 54
\section{Ferramentas de monitoramento de obras e políticas públicas}
\noindent Página 55
\pp{1} ok. {\bf \color{red} referência boa para fazer ponte com os avanços recentes e previstos com as redes sociais.}
\pp{2} ok.
\pp{3} ok
\subsection{OnTrack}
\pp{4} ok.
\pp{5} ok.
Figura 31: ok.
Figura 32: ok. Seria bastante interessante saber se há alguma iniciativa que implementa ou visa implementar o OnTrack no Brasil e se podemos usar instâncias já instaladas.
\subsection{De Olho nas Obras}
\pp{6} ok.
\subsection{RapidSMS}
\pp{7} bem legal. {\bf \color{red} Considerando p usar c AA.}
\noindent Página 57
\pp{8} Muito bom. Seria bastante interessante saber se há alguma iniciativa que implementa ou visa implementar o OnTrack no Brasil e se podemos usar instâncias já instaladas.
Figura 33: ok.
\subsection{Colab}
\pp{9} bem interessante. {\bf \color{red} ver sobre as ferramentas e entrar em contato com equipe}.
\noindent Página 58
\section{Estudo de caso: Gabinete Digital do Rio Grande do Sul}
\noindent Página 59
\pp{1} ok.
\pp{2} ok.
\pp{3} ok.
\pp{4} ok.
\pp{5} ok.
\pp{6} ok. {\bf \color{red} procurar e navegar minimamente pelo repo}.
\pp{7} ok.
\noindent Página 60
\subsection{Ferramentas de participação do Gabinete Digital RS}
\subsection{Governador Pergunta}
\pp{8} ok.
\pp{9} muito legal. {\bf \color{red} Temos acesso aos dados desta(s) consulta(s)?}
\pp{10} ok. {\bf \color{red} Bom que esta grande parcela da população reforça a pertinência do uso destes dados para benchmarks e pesquisas científicas.}
\pp{11} Interessante.
\subsubsection{Primeira edição: saúde pública}
\pp{12} muito legal.
\noindent Página 61
Figura 34: ok
\subsubsection{Segunda edição: Segurança no Trânsito}
\pp{13} ok.
\noindent Página 62
Figura 35: ok
\subsubsection{Terceira Edição: Reforma política}
\pp{14} ok.
\pp{15} ok.
\pp{16} ok.
\pp{17} ok. {\bf \color{red} Há opção no PairWise para equivalência entre as ideias? (assim o sistema pode estabelecer propostas equivalentes).}
\pp{18} ok.
\noindent Página 63
\pp{19} ok. Seria interessante fazer plots:
\begin{itemize}
    \item tempo de consulta \emph{versus} número de votos. 
    \item tempo de consulta \emph{versus} número de ideias.
    \item número de votos \emph{versus} número de ideias.
\end{itemize}
\subsubsection{Passos e condições para as consultas}
\pp{20} ok.
\pp{21} ok, muito legal.
\pp{22} ok, muito legal. {\bf \color{red} há ou foi discutida uma página similar para acompanhamento das propostas do arenaNETmundial?}
\noindent Página 64
Figura 36: ok.
\pp{23} ok.
\pp{24} ok.
\noindent Página 65
\pp{25} ok.
\subsubsection{Metodologia e arquitetura de escolha empregadas no Governo Pergunta}
\pp{26} ok.
\pp{27} ok.
\pp{28} ok, muito legal desse código. {\bf \color{red} procurar o repo e navegar minimamente.}
\pp{29} ok. Essa implementação evitou a necessidade de instalar várias instâncias do AllOurIdeas (pairwise?), é isso?
\noindent Página 66
Figura 37: ok.
\subsubsection{Propostas sem mediação}
\pp{30} ok.
\pp{31} ok.
\pp{32} ok.
\noindent Página 67
\pp{33} ok. {\bf \color{red} Mesmo positivo neutralizar grupos de interesse?}.
\pp{34} ok.
\pp{35} Interessante saber que as propostas polêmicas não tem força neste contexto. {\bf \color{red} elaborar formas de equilibrar esse processo?}
Tabela 1: ok.
\pp{36} ok.
\noindent Página 68
\subsubsection{Mobilização e Vans de Participação}
\pp{37} ok.
Figura 38: ok.
\pp{38} ok.
\pp{39} ok.
\noindent Página 69
\pp{40} ok.
\pp{41} ok.
\pp{42} ok.
\subsection{Governador Responde}
\pp{43} ok.
\pp{44} ok.
\pp{45} ok.
\noindent Página 70
\pp{46} ok
Tabela 2: ok.
\pp{47} ok.
\pp{48} ok. Uma pena estar desativada. {\bf \color{red} Seria legal ter acesso a estes dados}
\noindent Página 71
Tabela 3: ok.
Figura 39: ok.
\noindent Página 72
\subsection{Governo Escuta}
\pp{49} ok.
\pp{50} ok.
Tabela 4: ok.
\pp{51} ok.
\pp{52} ok.
\pp{53} ok.
\noindent Página 73
Figura 40: ok.
\subsubsection{Diálogo com 500 mil pessoas}
\pp{54} ok.
\pp{55} {\bf \color{red} Interessante que o número de pessoas nas ruas foi aprox. igual ao de pessoas assistindo do Governador escuta. Interessante que das 500 mil visitações, aprox. 4\% assistiram ao video, que é a parcela esperada de hubs.}
\pp{56} ok.
\subsection{Agenda Colaborativa}
\pp{57} ok.
\noindent Página 74
\pp{58} ok.
Figura 41: ok.
\pp{59} ok
\noindent Página 75
\pp{60} ok.
Tabela 5: ok.
\pp{61} ok.
\pp{62} ok, bem legal estas oficinas.
\pp{63} ok.
\subsection{Monitoramento de obras públicas}
\pp{64} ok.
\pp{65} ok.
Figura: 42: ok.
\noindent Página 77
\subsubsection{Metodologia}
\pp{66} ok.
\pp{67} ok.
\pp{68} ok.
\pp{69} ok.
\subsection{Prêmios alcançados}
\pp{70} ok.
\noindent Página 78
\subsection{Sistema Estadual de Paritcipação Cidadã}
\pp{71} ok.
\pp{72} ok. Legal do site da participação.

\noindent Página 79
\section{Proposta de integração e metodologias para o Participa.br}
\noindent Página 80
\pp{1} ok.
\pp{2} ok.
\pp{3} ok.
\pp{4} ok.
\subsection{Instâncias, mecanismos e ferramentas de participação}
\pp{5} ok.
\pp{6} ok.
\pp{7} ok.
\noindent Página 81
Figura 43: ok. Bem legal.
\pp{8} ok.
\pp{9} ok.
\pp{10} ok. {\bf \color{red} dicernimento útil: reflexivo, não imperativo.}
\pp{11} ok.
\noindent Página 82
Figura 44: ok.
\subsection{Fases das políticas públicas e a participação social}
\pp{12} ok.
\pp{13} ok.
\pp{14} ok. {\bf \color{red} Interessante essa relação entre bottom-up, democracia participativa e legitimidade de política pública.}
\pp{15} {\bf \color{red} Instrumental esta divisão do sebrae.}
\noindent Página 83
\pp{16} ok.
\subsection{Metodologias para mecanismos de participação}
\pp{17} ok.
\pp{18} ok.
\noindent Figura 45: ok.
\pp{19} ok.
\pp{20} ok. {\bf \color{red} Precendente para a foco na categorização dos agentes participativos: citada como um desafio pelo consultor.}
\noindent Página 84
\pp{21} ok.
\subsection{Arquitetura de escolha}
\pp{22} ok.
\pp{23} ok.
\pp{24} ok.
\pp{25} ok. {\bf \color{red} Talvez passar um período lendo sobre nudges \url{http://www.inudgeyou.com}.}
\noindent Página 85
\pp{26} ok. Talvez visitar o livro.
\pp{27} ok. {\bf \color{red} Como pode refletir positivamente: através de uma cartilha dos métodos mais comuns, para instrumentalizar o participante a reconhecer caso haja manipulação indevida.}
\pp{28} ok.
\pp{29} ok.
\pp{30} ok.
\noindent Página 86
\pp{31} ok.
\pp{34} ok.
\subsection{Mecanismos e metodologias recomendados}
\subsubsection{Constituição da agenda}
\pp{35} ok.
\pp{36} ok.
\pp{37} ok.
\pp{38} ok.
\noindent Página 87
\pp{39} ok.
\pp{40} ok.
\pp{41} ok.
\pp{42} ok.
Figura 46: ok.
\noindent Página 88
\subsubsection{Consultas e audiências públicas}
\pp{43} ok.
\pp{44} ok.
\pp{45} ok.
\pp{46} ok. {\bf \color{red} Apontamento para enquetes. Útil caso sejam propostas enquetes que atentem para a diversidade de papéis nas redes sociais.}
\pp{47} ok.
\pp{48} ok.
\pp{49} ok.
\noindent Página 89
Figura 47: ok.
\pp{50} ok.
\subsubsection{Conferências e gestão para conselhos}
\pp{51} ok.
\pp{52} ok.
\pp{53} ok. Tem também as conferências livres de educação, embora n incorporadas ao CONAE como esperado.
\pp{54} ok.
\noindent Página 90
\pp{55} ok.
Figura 48: ok.
\subsubsection{Decisões orçamentárias}
\pp{56} ok.
\pp{57} ok.
\pp{58} ok. {\bf \color{red} ver a ferramenta Delib}.
\noindent Página 91
\subsubsection{Monitoramento de políticas públicas}
\pp{59} ok.
\pp{60} ok.
\pp{61} ok.
\pp{62} ok.
\pp{63} ok.
\pp{64} ok. {\bf \color{red} Na definição usada pelo consultor, ``nós'' são os muito conectados, e os ``hubs'' são os atravessadores. Esta nomenclatura conflita com a da ciência das redes na física, na matemática aplicada e na computação, em que ``hubs'' são os muito conectados e ``nós'' é sinônimo de ``vértices'' (usualmente agentes em redes sociais ou elementos sendo relacionados nestas e outras redes). É útil procurar as fontes do consultor para complementar o ponto de vista usado pela consultoria de dados linkados, redes complexas e processamento de linguagem natural.}
\pp{65} ok.
\noindent Página 92
\pp{66} ok. {\bf Precedente para o uso de critérios de redes para seleção de agentes (p.ex. delegados): o consultor aponta a pertinência destas métricas para complementar os votos.}
\section{Conclusões preliminares}
\pp{1} ok.
\pp{2} ok.
\pp{3} ok.

\chapter{Metodologias de participação, ciclo de políticas públicas e  produção de metadados: integração às trilhas de participação do portal Participa.br}
Este documento contempla uma série de propostas de metodologias de participação mediadas por internet, considerando o ciclo de políticas públicas e sua consequente produção de metadados. O intuito é auxiliar na identificação dos processos decisórios e de participação a partir da plataforma do Participa.br. A proposta também aponta para formas de integração dos processos participativos às chamadas trilhas de participação do portal.
\noindent Página 7
\section{Introdução}
\pp{1} ok.
\pp{2} ok.
\pp{3} ok.
\pp{4} ok.
\pp{5} ok.
\pp{6} ok.
\noindent Página 8
\pp{7} ok. Legal que houve integração com os servidores.
\pp{8} ok.
\noindent Página 9
\section{Ciclo de políticas públicas}
\pp{9} {\bf \color{red} bem legal o triangulo: ciclo de políticas públicas, sistema de participação e trilha participativa.}
\pp{10} ok.
\pp{11} ok.
\pp{12} ok.
\pp{13} que accountability?
\pp{14} ok.
\pp{15} {\bf \color{red} legal as tendências a profissionalização e rarefação dos agentes participativos.}
\noindent Página 10
\pp{16} ok.
Figura 1: Conselhos são prioritariamente para monitoramento e fiscalização? E ouvidoria?
\pp{17} ok.
\pp{18} ok. Interessante dos conselhos e ouvidorias.
\noindent Página 11
\pp{19} ok. {\bf \color{red} ``fugir do setorialismo sem tornar as pautas individualizadas''}
\pp{20} ok.
\pp{21} ok.
\pp{22} ok.
\pp{23} legal dos focos de atenção pelos filtros participativos.
\pp{24} ``consolidar agenda para movimentos sociais''.
\noindent Página 12
\pp{25} dualidade do participa.br: instância de comunidade formal ou por interesse, instância para aflorar demandas individuais pelos processos.
\pp{26} ok. {\bf \color{red} indicador de inclusão na participação social.}
\pp{27} ok.
\pp{28} ok. {\bf \color{red} Declaração de registro histórico com a consulta do arenaNETmundial: 280 mil votos, 200 participações; uma das maiores consultas públicas feitas no Brasil, certamente a maior consulta digital. (internacional?)}
\pp{29} {\bf \color{red} participação digital: novos métodos e melhoras para os existentes}.
\subsection{Trilhas de participação}
\pp{30} ok.
\noindent Página 13
\pp{31} ok.
\pp{32} ok.
\pp{33} ok.
\noindent Página 14
Figura 2: ok.
\noindent Página 15
\pp{34} ok.
\noindent Página 16
Figura 3: ok.
\pp{35} ok.
\pp{36} ok. {\bf \color{red} Pensando em trilhas p o servidor}
\pp{37} ok.
\pp{38} ok.
\pp{39} ok. {\bf \color{red} Apontanda demanda por sistematização dos conteúdos de participação}
\noindent Página 17
\pp{40} ok.
\pp{41} {\bf \color{red} motivação para a sistematização dos conteúdos: evitar sobreposição de pautas e demandas}.
\pp{42} ``balcanização da participação''.
\pp{43} ok.
\pp{44} {\bf \color{red} triângulo: Noosfero incorpora ferramentas participativas; informacionalmente, integra-se com os ciclos participativos e das políticas públicas; metodologicamente, deve ser um canal de uso das ferramentas para as comunidades.}
\noindent Página 18
\subsection{Recomendações para integração das trilhas aos ciclos}
\pp{45} ok.
Figura 4: ok.
\pp{46} ok.
\noindent Página 19
\section{Integração do Participa.br com as mesas de diálogo e monitoramento}
\pp{47} ``interface para mesas de diálogo e monitoramento.''
\pp{48} ok.
\pp{49} ok.
\pp{50} ok. Bem legal das mesas de diálogo (negociação) e de monitoramento.
\pp{51} ok. {\bf \color{red} Demanda de um dashboards de indicadores, para mesas de monitoramento.}
\section{Elementos motivadores}
\pp{52} ok. {\bf \color{red} governança compartilhada através da transparência, prestação de contas e siálogo social}
\pp{53} {\bf \color{red} objs da SNAS: canais participativos não hierarquizados, flexíveis, abertos e que dialógue com as redes.}
\pp{54} O setor está em fase avançada de implementação de um sistema de indicadores, painel de monitoramento e controle de demandas que pode ser integrado ao Partipa.br. {\bf \color{red} onde está este painel? há especificações deste sistema de indicadores?}
\pp{55} {\bf \color{red} ``...auxiliado por processos de dados abertos, provido por uma interface robusta no que tange a aplicação de recursos federais nos diversos níveis, acessível ao cidadão.''}
\pp{56} ok.
\subsection{Recomendações metodológicas}
\pp{57} ``trilha de participação e monitoramento''.
\subsubsection{Estabelecer critérios e requisitos mínimos para a formação de comunidades}
\pp{58} {\bf \color{red} estabelecer claramente que o servidor possui uma trilha de monitoramento que tem como objetivo monitorar o Participa.br. A ``abertura da abertura''.}
\pp{59} ok.
\pp{60} ok.
\subsubsection{Estabelecimento de agenda de participação}
\pp{61} {\bf \color{red} pensar em séries temporais para agendas, além disso, somente os vinculados à SNAS?}
\pp{62} ok.
\noindent Página 22
\pp{63} ok.
\subsubsection{Estabelecer processos dialógicos com prazo definido de retornos}
\pp{64} {\bf \color{red} apontando necessidade de priorizar devolutivas do governo na agenda para melhorar engajamento}.
\pp{65} ok.
\pp{66} ok. {\bf \color{red} procurar os produtos desta consultoria feita para o Departamento de Diálogos Sociais.}
\noindent Página 23
Figura 24: ok. Bastante interessante o ciclo.
\noindent Página 24
\pp{67} {\bf \color{red} proposto sistema de monitoramento para uso interno da SNAS, enquanto o Participa.br p contato com a sociedade.} (isso?)
\pp{68} ok.
\pp{69} {\bf \color{red} apontada pertinência de auxílio na devolutiva do Estado através dos rastros digitais e análise pública deles}.
\subsubsection{Utilização de dados abertos como forma de alimentar indicadores}
\pp{70} {\bf \color{red} Argumento para dados linkados.}
\noindent Página 25
\section{Sugestão para criação de metadados}
\pp{1} ok.
\pp{2} ok.
\pp{3} {\bf \color{red} como assim } ``complementa com mais três além de adicionar outros campos. ''?
\pp{4} ok. (com tabela sem numeração)
\noindent Página 26
\pp{5} legal essa de dc+mpeg 7.
\pp{6} ok.
\subsection{O que classificar?}
\pp{7} ok.
\pp{8} ok. {\bf \color{red} Proposta de metadados para os resultados, trilhas e comunidades}
\subsection{O que é obrigatório e opcional}
\pp{9} {\bf \color{red} como assim?} ``Os campos não devem ter variação quanto à categoria e a tipo dos dados a serem preenchidos''.
\noindent Página 27
\subsection{Vocabulário}
\pp{10} ok.
\subsection{Construindo uma proposta de metadados}
\pp{11} importante {\bf \color{red} apontada pertinência de uma atividade com desenvolvedores, consultores coordenadores e outros envolvidos no projeto, para atividade volta p vocabulário}
\pp{12} legal dos campos para integração com políticas públicas.
Tabela: {\bf \color{red} REFERENCIA: proposta de campos para trilhas participativas.} Integrar em versão próxima da triplificação dos dados do participa.
\noindent Página 28
\pp{13} {\bf pensar na compatibilização da triplificação feita, proposta nesta tabela, e microdados feitos pela Dani.}
\noindent Página 29
\subsection{Integração de metadados, trilhas, ciclos e processos decisórios}
\pp{14} ok.
\pp{15} ok. {\bf \color{red} ocorre-me de fazer um servidor p ação: trabalhar na opa e extração de info dos dados do Participa}
\pp{16} ok.
\pp{17} ok. {\bf \color{red} que grupo de foco?}
\noindent Página 30
\section{Conclusões preliminares}
\pp{1} ok.
\pp{2} ok.
\pp{3} ok. {\bf \color{red} aqui tá ``uma das maiores consultas'' para a do arenaNETmundial}.
\pp{4} ok.
\pp{5} ok. {\bf \color{red} apologia ao sistma de monitoramento encaminhado à DITEC}.
\chapter{Metodologias e ferramentas de participaçãodigital para conferências nacionais}
Este produto analisa o contexto das Conferências Nacionais e sua relação com as ferramentas
de participação existentes no Participa.br. A partir da literatura pesquisada e das experiências em
conferências, foram elaboradas sugestões metodológicas para o portal. O objetivo é ampliar a
inclusão de novos atores, alcançar maior eficiência e transparência nos processos conferenciais e
promover a integração com uma interface que ajude no monitoramento da efetivação das diretrizes
elaboradas pelas conferências. São abordadas metodologias de votação para eleição de delegados,
sistemas de priorização de proposta e, por fim, apresentada uma sugestão de protótipo para
melhorar a construção de documentos colaborativos, a fim de ampliar ainda mais o escopo dos
processo de consultas públicas.
\PP{6}
\section{Introdução}
\pp{} \VV{Arquiteturas de escolha}.
\pp{} inclusão de novos atores: \VV{mobilizar periféricos}.
\pp{} ok.
\pp{} 70\% das conferências brasileiras ocorreram na última décadas.
\pp{} Foco nas mobilizações nacionais. \VV{Apontada necessidade de ``expandir o debate para a rede'' no aproveitamento do digital}.
\PP{7}
\pp{} ok. \VV{Apontado apetite por temas recentes na participação social.}
\pp{} \VV{Observada escassez de inovação nos processos deliberativos através da internet. Nas conferências, somente a COMIGRAR.}
\pp{} ok.
\PP{8}
\pp{} ok.
\pp{} ok.
\pp{} ok.
\PP{9}
\section{Contexto das consultas}
\pp{} ok.
\pp{} ok.
\pp{} ok
\PP{10}
\pp{} ok.
\subsection{Desenho institucional}
\pp{} \VV{focos na efetividade dos resultados e na inclusão de atores.}
\pp{} ok.
\pp{} ok.
\pp{} ``demanda conferencial geralmente parte de setores
organizados que são acolhidos pelo Executivo''.
\PP{11}
\pp{} ok. Primeira etapa é a aprovação do regimento.
\pp{} ok. A segunda é a mobilização para discussões locais. Nestas, eleições para delegados para as municipais.
\pp{} ok.
\pp{} ok. \VV{Estatutos especificam forma de deliberação}
\PP{12}
\pp{} ok.
Figura: \VV{Etapas das conferências, de convocações à publicação dos resultados}.
\pp{} \VV{Reportado uso de voto cumulativo}
\pp{} ok.
\pp{} ok.
\pp{} ok.
\PP{13}
\pp{} ok. \VV{Apontada pertinência de dinâmicas de elaboração de documentos.}
\pp{} ok. \VV{Dinâmica de priorização de propostas: podem contar com percentagem mínima de intermediários e periféricos. Outra: exibe resultados também por setor (hub, intermediário ou periférico) e por comunidades detectadas.}
\pp{} \VV{De antemão, pode-se disponibilizar um sistema de votação aberto, no qual outras pessoas validam o votante e o voto dele. Desta forma falcatruas e falhas de seguração podem saltar à vista das comunidades, especialmente com análises abertas.}
\pp{} \VV{está sendo implementado onde?}
\pp{} ok.
\PP{14}
\section{Experiêcias de consultas públicas online}
\pp{} ok.
\pp{} ok.
\subsection{Marco Civil da Internet}
\pp{} ok. \VV{Os dados destas estapas estão disponíveis?}
\pp{} ok.
\PP{15}
\pp{} ok.
\pp{} ok. \VV{Plugin Dialogue p comentários p parágrafo no WP}.
\pp{} ok.
\pp{} ok.
\PP{16}
\pp{} ok.
\pp{} ok.
\subsection{Plano e Compromisso Nacional de Participação Social}
\pp{} ok.
\pp{} ok.
\pp{} ok.
\PP{17}
\pp{} ok.
\pp{} ok.
\PP{18}
\subsection{Consulta Arena Net Mundial}
\pp{} ok.
\pp{} ok.
\pp{} ok.
\pp{} ok.
\pp{} ok.
\PP{19}
\pp{} ok.
\pp{} ok.
\pp{} ok. Legal das 3 perguntas.
\pp{} ok.
\PP{20}
\pp{} \VV{Avanços da implementação do participa do pairWise}
\pp{} ok.
\PP{21}
\section{Sugestões metodológicas para o Participa.br em relação às conferências}
\subsection{Comunidades e subcomunidades}
\pp{} ok.
\pp{} ok.
\pp{} ok.
\pp{} ok.
\pp{} ok.
\PP{22}
\pp{} ok.
\pp{} \VV{Já tem subcomunidade no Participa.br. O item se refere à criação dos critérios de aprovação?}
\pp{} ok.
\subsection{Eleição de delegados}
\pp{} ok.
\PP{23}
\pp{} \VV{Com os dados linkados no endpoint, pode-se fazer processos deste tipo com a simples convenção de tags em uma postagem. Tanto a contagem na mão pode ser mais adequada, quanto scripts podem processar as mensagens com os marcadores de voto para delegados e outros fins.}
\pp{} ok.
\pp{} Item ``a.'' pode ser uma postagem na comunidade. Ou ser definida em uma etapa de trilha. Assim como item ``b.''. Pode servir de exemplo para o funcionamento da interface.
\pp{} \VV{é interessante que os votos possam ser mudados por um período de tempo até estabilizar? Aliás, podem começar aleatórios como incentivo aos membros da comunidade para ir mudar.}
\PP{24}
\pp{} \VV{Belíssimo link de status com papel de delegado. Abre para contabilização de participações online e status/papel por rastro de uso da plataforma.} TTM
\pp{} ok.
\pp{} ok.
\pp{} ok.
\pp{} ok.
\pp{} ok. Não pode ser atribuido pela equipe do participa o token?
\PP{25}
\subsection{Priorização de propostas}
\pp{} ok.
\pp{} ok.
\PP{26}
\pp{} ok. \VV{Lembra muito as ideias do Silva macambira e o conferência permanente que fizemos}.
\pp{} \VV{Os dispositivos móveis podem abrir o browser mesmo. Front em svg ou meteor em iframe}.
\subsection{Acompanhamento da efetividade}
\pp{} ok.
\pp{} ok.
\pp{} ok.
\PP{27}
\pp{} \VV{Como está a construção deste painel para mesas de diálogo e conferências?}
\pp{} ok.
\pp{} ok.
\pp{} ok.
\pp{} ok. \VV{um dos ganhos com as plataformas digitais eh agregar cidadaos que podem contribuir com o processo embora não (necessariamente) organizados}
\PP{28}
\pp{} ok.
\PP{29}
\section{Protótipo para consultas estruturadas e abertas}
\pp{} ok.
\pp{} ok.
\pp{} ok.
\subsection{Consulta temática ou por eixos}
\pp{} ok.
\PP{30}
\pp{} ok.
\pp{} ok.
\pp{} ok.
\PP{31}
\pp{} ok.
\pp{} ok.
\pp{} ok.
\PP{32}
\pp{} ok.
\PP{33}
\subsection{Proposta de protótipo para regulamentação do Marco Civil da Internet}
\pp{} ok.
\pp{} ok.
\pp{} ok.
\pp{} ok.
\pp{} ok.
\PP{36}
\subsection{Social Karma para participação social}
\pp{} ok.
\pp{} ok.
\pp{} ok.
\pp{} ok. \VV{Há um termo de concessão dos dados para os usuários do Participa.br?}
\pp{} ok.
\pp{} ok.
\pp{} ok.
\PP{37}
\section{Conclusão}
\pp{} ok.
\pp{} ok.
\pp{} ok.
\pp{} ok.
\PP{38}
\pp{} ok.

\part{Produtos de Grazielle Machado Fernandes}
\chapter{Plano de trabalho contendo estratégias de relacionamento e mobilização e mapeamento de intelocutores do ``Projeto PARTICIPA''}
\noindent Página 1
\section{Participa.br - Portal Federal de Participação Social}
\pp{1} ok.
\pp{2} ok.
\pp{3} ok.
\pp{4} ok. {\bf \color{red} Necessidade do sistema monitoramento}
\noindent Página 2
Figura: ok. {\bf \color{red} ciclo fundamental encontrado.}
\section{Monitoramento e mapeamento da redes}
\pp{5} ok.
\pp{6} ok.
\noindent Página 3
\section{Identificação de Influenciadores X Gestores Públicos}
\pp{7} {\bf \color{red} previsto SM e contato com agentes detectados e considerados estratégicos}
\pp{8} essa abertura de dados e análise vinga como ideia pois está ancorada na realidade, portanto é útil e instrumental. {\bf \color{red} este momento aponta para o monitoramento interno.}
Figura: ok. {\bf \color{red} mais um ciclo fundamental.}
\section{Mapeamento e abordagem de interlocutores}
\pp{9} ok.
\pp{10} {\bf \color{red} Trecho descreve trilhas criadas via colaboração de agentes civis (escuta/monitoramento nas redes sociais) e gov (monitoramento interno)}.
\noindent Página 4
\pp{11} ok e previsto como responsabilidade da equipe do Participa.br.
\pp{12} ok. {\bf \color{red} estas regras são campos já presentes nas trilhas? Colocar na OPA}
\pp{13} ok.
\section{Ações previstas para realização até dezembro de 2014}
\pp{14} ok.
\pp{15} ok.
\pp{16} ok.
\pp{17} ok. {\bf \color{red} para a difusão do Participa.br e termo \#ParticipaBR, escolher agentes para fazerem as 3 rodadas de difusão nas redes deles, ou comunicação única através de atravessadores e mais próximos.}
\section{Ações a serem executadas para atigir os objetivos propostos}
\pp{18} ok.
\noindent Página 5
\pp{19} ok. {\bf \color{red} muito jóia esta proposta de mobilizar os blogueiros.}
\pp{20} ok.
\pp{21} ok.
\pp{22} ok.
\pp{23} ok.
\pp{24} ok.
\pp{25} ok. {\bf\color{red}já rolou este ``todo mundo participa''}?
\pp{26} ok.
\noindent Página 6
\pp{27} ok.
\pp{28} ok. {\bf \color{red} ponte para o govern-art}.
\pp{29} ok.
\section{Dinâmina da blogagem coletiva}
\pp{1} ok.
\pp{2} ok.
\pp{3} ok.
\pp{4} ok.
\noindent Página 7
\pp{5} ok.
\section{Dinâmica de campanha Todo Mundo Participa}
\pp{1} ok.
\pp{2} ok.
\pp{3} ok.
\pp{4} ok.
\pp{5} ok.
\pp{6} ok.
\pp{7} ok.
\section{Anexos}
\chapter{Produto 2}
\section{Introdução}
\pp{1} ok.
\pp{2} ok.
\pp{3} ok.
\pp{4} ok.
\pp{5} ok.
\pp{6} ok.
\section{O que é o Participa.br}
\pp{1} ok.
\noindent Página 2
\pp{2} ok.
\section{Identificação de mobilização de públicos}
\pp{1} ok.
\pp{2} ok.
\pp{3} ok.
\pp{4} ok.
\pp{5} ok.
\noindent Página 3
Figura: ok.
\subsection{Público interno (agentes e órgãos públicos)}
\pp{6} ok.
\pp{7} ok.
\subsubsection{Comunidades com processos de participação em curso}
\noindent Página 4
\pp{8} ok.
\pp{9} ok.
\pp{10} ok.
\pp{11} ok.
\pp{12} ok.
\pp{13} ok.
\pp{14} ok.
\pp{15} ok.
\subsection{Público externo (sociedade civil)}
\pp{16} ok. {\bf \color{red} manifesta busca pelos que não dialogam com o estado}
\noindent Página 5
\section{Instâncias e mecanismos de participação}
\pp{1} ok.
\pp{2} ok.
\pp{3} ok.
\pp{4} ok.
\pp{5} ok. {\bf \color{red} manifesta intensão de massificar a participação social}.
\noindent Página 6
Figura: ok.
\subsection{Exemplos de práticas que podem ser incorporadas ao Participa.br}
\subsubsection{Gabinete Digital RS}
\pp{6} ok.
\noindent Página 7
\pp{7} ok.
\subsubsection{Gabinete Digital de Caruaru}
\pp{8} ok.
\pp{9} ok.
\subsubsection{Webcidadania Xingu - Cidade Democrática}
\pp{10} {\bf \color{red} visitar a iniciativa para ver as inspirações e as propostas encaminhadas}
\pp{11} ok.
\noindent Página 8
\section{Mobilização social x Engajamento de agentes públicos}
\pp{12} ok. {\bf \color{red} ``diretrizes e prioridades de políticas públicas são os principais objetivos das conferências''}.
\pp{13} ok. {\bf pensando em retomar as ontologias dos mecanismos participativos em si.}
\pp{14} ok.
\subsection{Metodologia e cronograma de trabalho}
\pp{15} ok.
\pp{16} ok.
\pp{17} {\bf \color{red} este levantamento ocorreu? Foi colocado na plataforma?}
\noindent Página 9
\pp{18} ok.
\section{Relacionamento, o que diferencia o Participa.br}
\pp{1} ok.
\pp{2} ok.
\pp{3} ok.
\pp{4} ok.
\pp{5} ok. {\bf o SM pode ajudar principalmente no escutar e no mensurar}.
\noindent Página 10
Figura: ok.
\section{Anexos}
ok.
\chapter{Produto 3 - Documento contendo estratégias de mobilização de servidores e órgãos públicos em conjunto com a base social mobilizada em torno do Portal da Participação Social}
\section{Introdução}
\pp{1} ok.
\pp{2} ok.
\pp{3} ok.
\pp{4} ok.
\pp{5} ok.
\pp{6} ok.
\section{Como o Participa.br pode ampliar as formas de comunicação}
\pp{7} ok.
\noindent Página 2
\pp{8} ok.
\section{Mobilizando a base social em torno do Portal da Participação Social}
\pp{9} ok. {\bf \color{red} inserção nas diversas redes: dados linkados em todas estas instâncias}
\pp{10} ok.
\pp{} ok.
\pp{} ok.
\pp{} ok.
\subsection{ArenaNETmundial e a maior consulta pública realizada na internet}
\subsubsection{Consulta Pública}
\pp{} ok.
\noindent Página 3
\pp{} ok.
\pp{} ok.
\pp{} ok.
\pp{} ok.
\pp{} ok.
\subsubsection{ArenaNETmundial ParticipaBR}
\pp{} ok.
\noindent Página 4
\pp{} ok.
\pp{} ok.
\pp{} ok.
\pp{} ok.
\pp{} ok.
{\bf Cobertura Colaborativa}
\pp{} ok.
\pp{} ok.
\pp{} ok.
{\bf Parceria com universidades}
\noindent Página 5
\pp{} ok.
\pp{} ok.
\pp{} ok.
\pp{} ok.
\pp{} ok.
\pp{} ok.
\pp{} ok.
\noindent Página 6
Figura: ok.
\subsection{Hangout}
\pp{} ok.
\pp{} ok.
\noindent Página 7
\section{Mobilizando servidores e órgãos públicos}
\pp{} ok.
\pp{} ok.
\subsection{Reunião com gestores}
\pp{} ok.
\pp{} ok.
\subsection{Treinamento}
\pp{} ok.
\pp{} ok.
\pp{} ok.
\subsection{Interação nas redes sociais}
\pp{} ok.
\noindent Página 8
\pp{} ok.
\subsection{Criação de comunidades}
\pp{} ok.
\pp{} ok.
\subsection{Trilhas de participação}
\pp{} ok.
\pp{} ok.
\section{Comunidades com gestão estabelecida por órgão da administração}
\pp{} ok.
\noindent Página 9
\pp{} ok.
\pp{} ok.
\pp{} ok.
\pp{} ok.
\pp{} ok.
\section{Anexos}
ok.

\part{Produtos de Joênio Marques da Costa}
\chapter{Documento com guia de codificação "coding guidelines" para o desenvolvimento do código do portal objetivando o reaproveitamento de código e o fomento à formação de comunidades em torno dos módulos, bem como tutoriais para implementação local das Soluções.}
\section{Apresentação}
\pp{} ok.
\pp{} ok.
\section{Comunidade Noosfero}
\pp{} ok.
\pp{} ok.
\pp{} ok.
\pp{} ok.
\pp{} ok.
\pp{} ok.
\pp{} ok.
\pp{} ok.
Figura 1: ok. Gostaria do link do oloh do noosfero.
\section{Desenvolvimento do Noosfero}
\pp{} ok.
\pp{} ok.
\noindent Página 8
\pp{} ok.
\pp{} ok.
Figura 2: ok.
\pp{} ok.
Figura 3: ok.
\pp{} ok.
\pp{} ok.
\noindent Página 9
\pp{} ok.
\subsection{Instalação, Execução, Desenvolvimento}
\pp{} ok.
\pp{} ok.
\pp{} ok. {\bf embora tenha q testar o schroot (nunca usei)}.
\section{Orientações para contribuição}
\pp{} ok.
\pp{} ok.
\noindent Página 10
\pp{} ok.
\pp{} ok.
\subsection{Guia de codificação}
\pp{} ok.
\subsubsection{Princípios}
\pp{} ok.
\pp{} ok.
\pp{} ok.
\pp{} ok.
\pp{} ok.
\pp{} ok.
\noindent Página 11
\pp{} ok.
\pp{} ok.
\subsubsection{Models}
\pp{} ok.
\noindent Página 12
\pp{} ok.
\subsubsection{Views}
\pp{} ok, embora eu não lembre o que seja um Partial.
\pp{} ok.
\subsubsection{Testes}
\pp{} ok.
\pp{} ok.
\pp{} ok.
\pp{} ok.
\pp{} ok.
\noindent Página 13
\subsubsection{Traduções}
\pp{} ok.
\pp{} ok.
\pp{} ok.
\subsubsection{JavaScript}
\pp{} ok.
\pp{} ok.
\noindent Página 14
\pp{} ok.
\subsubsection{CSS}
\pp{} ok.
\pp{} ok.
\noindent Página 15
\pp{} ok.
\pp{} ok.
\pp{} ok.
\subsection{Submissão de código}
\pp{} ok.
\pp{} ok.
\pp{} ok.
\noindent Página 16
\pp{} ok.
\pp{} ok.
\pp{} ok.
Figura 4: ok.
\noindent Página 17
\pp{} ok.
\pp{} ok.
\pp{} ok.
\pp{} ok.
\pp{} ok.
\pp{} ok.
\section{Convenções de codificação}
\pp{} ok.
\noindent Página 18
\pp{} ok.
\pp{} ok.
\pp{} ok.
\pp{} ok.
\pp{} ok.
\pp{} ok.
\noindent Página 19
\pp{} ok.
\section{Plugins Noosfero}
\pp{} ok.
\subsection{Ativação de plugins}
\pp{} ok.
\pp{} ok.
\pp{} ok.
\pp{} ok.
\noindent Página 20
\pp{} ok.
\subsection{Estrutura Básica}
\pp{} ok.
\pp{} ok.
\pp{} ok.
\subsection{Visão do desenvolvedor do Noosfero}
\pp{} ok.
\subsubsection{Definindo hotspots}
\pp{} ok.
\noindent Página 21
\pp{} ok.
\pp{} ok.
\pp{} ok.
\pp{} ok.
\pp{} ok.
\noindent Página 22
\pp{} ok.
\pp{} ok.
\pp{} ok.
\noindent Página 23
\pp{} ok.
\pp{} ok.
\pp{} ok.
\pp{} ok.
\pp{} ok.
\subsection{Visão do desenvolvedor de Plugins}
\pp{} ok.
\subsubsection{Como criar um Plugin Noosfero}
\pp{} ok.
\pp{} ok.
\noindent Página 24
\pp{} ok.
\subsubsection{Definição}
\pp{} ok.
\pp{} ok.
\pp{} ok.
\noindent Página 25
\subsubsection{Plugins e extensão do core}
\pp{} ok.
\pp{} ok.
\pp{} ok.
\pp{} ok.
\noindent Página 26
\pp{} ok.
\pp{} ok.
\pp{} ok.
Extendendo classes do core
\pp{} ok.
\pp{} ok.
\pp{} ok.
\noindent Página 27
\pp{} ok.
\pp{} ok.
\pp{} ok.
\pp{} ok.
\pp{} ok.
\subsubsection{Tabelas e registros}
\noindent Página 28
\pp{} ok.
\noindent Página 29
\pp{} ok.
\pp{} ok.
\pp{} ok.
\pp{} ok. {\bf \color{red} ``lib/mezuro\_plugin/project.rb'' dentro da pasta do plugin, certo?}
\noindent Página 30
\pp{} ok.
\subsubsection{Controllers e rotas}
\pp{} ok.
\pp{} ok.
\pp{} ok.
\pp{} ok.
\noindent Página 31
\pp{} ok.
\pp{} ok.
\pp{} ok.
\pp{} ok.
\pp{} ok.
\pp{} ok.
\noindent Página 32
\pp{} ok.
\pp{} ok.
\pp{} ok.
\pp{} ok.
\pp{} ok.
\subsubsection{Views}
\pp{} ok.
\pp{} ok.
\noindent Página 33
\pp{} ok.
\pp{} ok.
\pp{} ok.
\pp{} ok.
\pp{} ok.
\pp{} ok.
\noindent Página 34
\pp{} ok.
\subsubsection{Arquivos públicos}
\pp{} ok.
\pp{} ok.
\pp{} ok.
\pp{} ok.
\subsubsection{Testes}
\pp{} ok.
\pp{} ok.
\noindent Página 35
\pp{} ok.
\subsubsection{Dependencias and Instalação}
\pp{} ok.
\pp{} ok.
\pp{} ok.
\pp{} ok.
\section{Comunidade Participa.br}
\pp{} ok.
\pp{} {\bf \color{red} visitar o DisplayContextContent e ver se cabe ali os dados linkados, medidas de texto, de redes e usuários e infos de acesso à API e maiores análises}. Outra: {\bf \color{red} adicionar funcionalidades no pgSearchPlugin.} Outro: {\bf \color{red} visitar o statistics plugin e ver o que cabe ali de pln e rc}.
\noindent Página 36
\pp{} ok.
\section{Fomento à formação de comunidade de desenvolvimento}
\pp{} {\bf \color{red} houve algum destes eventos ``mão na massa'' para fomentar comunidade de desenvolvedores?}
\section{Considerações finais}
\pp{} ok.
\pp{} ok.
\pp{} ok.
\chapter{Documento com análise de arquiteturas de sistemas de identidade distribuída, estratégia de implantação considerando os sites parceiros e contendo propostas de códigos}
\PP{6}
\section{Apresentação}
\pp{} ok.
\pp{} ok.
\section{SSO - Single Sign-on}
\subsection{O que é SSO?}
\pp{} ok.
\pp{} ok.
\pp{} ok. \VV{SM pode precisar de administração robusta se for servidor de SSO.}
\PP{7}
\subsection{SSO no Noosfero}
\pp{} ok.
\pp{} ok.
\pp{} ok.
\subsection{Qual problema SSO resolve?}
\pp{} ok.
\pp{} ok.
\pp{} \VV{legal do web messaging}. ok.
\subsection{Como SSO funciona?}
\pp{} ok.
\PP{8}
\pp{} ok.
Figura 1: ok.
\subsection{Quais soluções de SSO existem?}
\pp{} ok.
\subsubsection{Accounts \& SSO}
\pp{} ok.
\PP{9}
\subsubsection{Central Authentication Service (CAS)}
\pp{} ok.
\pp{} ok.
\pp{} ok.
\pp{} ok.
\pp{} ok
\pp{} ok.
\PP{10}
Figura 2: ok.
\subsubsection{Distributed Access Control System (DACS)}
\pp{} ok.
\pp{} ok.
\pp{} ok.
\PP{11}
\pp{} ok.
\subsubsection{Enterprise Sign On Engine}
\pp{} ok.
\pp{} ok.
\pp{} ok.
\pp{} ok.
\subsubsection{FreeIPA}
\pp{} ok.
\pp{} ok.
\pp{} ok.
\pp{} ok.
\subsubsection{IBM Enterprise Identity Mapping}
\pp{} ok.
\pp{} ok.
\pp{} ok.
\PP{12}
\subsubsection{JBoss SSO}
\pp{} \VV{impressão ou algumas das soluções para login unificado também se propõem a integrar logins diferentes, de terceiros, em outros padrões?}
\pp{} ok.
\pp{} ok.
\pp{} ok.
\subsubsection{JOSSO}
\pp{} ok.
\pp{} ok.
\pp{} ok.
\subsubsection{Kerberos}
\pp{} ok.
\pp{} ok.
\pp{} ok.
\PP{13}
Figura 3: ok.
\subsubsection{OpenAM}
\pp{} ok.
\pp{} ok.
\pp{} ok.
\pp{} ok.
\subsubsection{Pubcookie}
\pp{} ok.
\PP{14}
Figura: ok.
\pp{} ok.
\pp{} ok.
\subsubsection{SAML}
\pp{} ok.
\pp{} ok.
\pp{} ok
\PP{15}.
Figura 5
\pp{} ok.
\pp{} ok.
\subsubsection{Shibboleth}
\pp{} ok.
\pp{} ok.
\pp{} ok.
\PP{16}
\subsubsection{ZXID}
\pp{} ok.
\pp{} ok.
\pp{} ok.
\section{IdP - Identity Provider}
\subsection{O que é IdP}
\pp{} ok.
\pp{} ok.
\pp{} ok.
\pp{} ok.
\pp{} ok.
\pp{} ok.
\PP{17}
\subsection{IdP no Noosfero}
\pp{} ok.
\subsection{Qual problema IdP resolve?}
\pp{} ok.
\subsection{Quais soluções de IdP existem?}
\subsubsection{Mozilla Persona}
\pp{} ok.
\pp{} ok.
\pp{} ok.
\pp{} ok.
\pp{} ok. \VV{No caso da participação social, é melhor que toda a atividade seja rastreável, não?}
\PP{18}
\subsubsection{OAuth}
\pp{} ok.
\pp{} ok.
\pp{} ok.
\pp{} ok.
\pp{} ok.
\subsubsection{OpenID}
\pp{} ok.
\pp{} ok.
\pp{} ok.
\pp{} ok.
\pp{} ok.
\subsubsection{OpenID Connect}
\pp{} ok.
\PP{19}
\pp{} ok.
\pp{} ok.
\section{Outras iniciativas}
\subsection{OpenAthens - Reino Unido}
\pp{} ok.
\pp{} ok.
\subsection{Microsoft account}
\pp{} ok.
\subsection{Liberty Alliance}
\pp{} ok.
\PP{20}
\pp{} ok.
\section{Discussão}
\subsection{Iniciativas (Governo e Comunidade)}
\subsubsection{Login Cidadão}
\pp{} ok.
\pp{} ok.
\pp{} ok.
\pp{} ok.
\subsubsection{Id da Cultura}
\pp{} ok.
\pp{} ok.
\PP{21}
\subsection{Proposta para o Participa.BR}
\pp{} ok. {\bf \color{red} citados o Participatório, o Cidade Democrática e o Cultura Educa. Destes, só não tenho dados do cultura educa para triplificar.}
\pp{} ok.
\pp{} ok.
\pp{} ok.
\pp{} ok.
\pp{} ok.
\pp{} ok.
\PP{22}
\pp{} ok.
\pp{} ok.
\section{Considerações finais}
\pp{} ok.
\pp{} ok.
\chapter{Proposta de ferramentas de participação social:
Ferramenta de elaboração e
discussão de propostas e ferramenta de votação em pares para o Participa.br}
O Participa.br é a plataforma federal de participação social, ela fomenta o diálogo entre
o governo e sociedade civil.
É desenvolvida em softwares livres e utiliza o Noosfero como
infraestrutura básica em sua construção.
A plataforma está em constante evolução e este
documento traz a proposta de 2 novas ferramentas com diversas novas funcionalidades para
promover a participação do cidadão nas decisões e na elaboração de políticas públicas.
\PP{7}
\section{Introdução}
\pp{} ok. \VV{Pensando em possíveis aplicativos e etapas}.
\pp{} ok.
\subsection{O Participa.br}
\pp{} ok.
\pp{} ok.
\pp{} ok.
\subsection{O Noosfero}
\pp{} ok.
\pp{} ok.
\PP{8}
\section{Desenvolvimento}
\pp{} ok.
\pp{} ok.
\pp{} ok.
\subsection{Aplicativos para trilhas de participação}
\pp{} ok.
\pp{} ok.
\pp{} ok.
\subsubsection{Ferramenta para construção e debate de propostas}
\pp{} ok.
\pp{} ok.
\PP{9}
\pp{} ok.
\pp{} ok.
\pp{} ok. \VV{Nesta altura, cada proposta pode ser votada favoravel ou desfavoravelmente, pode ser etiquetada e comentada por parágrafo ou trecho, certo? Pode ser confrontada com outras para votação no esquema PairWise?}
\pp{} ok.
\pp{} ok.
\pp{} ok.
\pp{} ok.
\PP{10}
\subsubsection{Votação em Pares}
\pp{} ok.
\pp{} ok \VV{, embora não conflua 100\% com dispensar a autoria dos votos. Creio que isso deva ser analisado caso a caso}.
\pp{} ok.
\pp{} ok.
\pp{} ok.
\pp{} ok.
\PP{11}
\pp{} ok.
\pp{} ok.
\pp{} ok.
\pp{} ok.
\pp{} ok.
\pp{} ok.
\section{Conclusão}
\pp{} ok.
\pp{} ok.
\part{Fabiano Rangel Cidade}
\chapter{Documento com planejamento e metodologia para atividades de design participativo para o tema padrão do portal}
\PP{4}
\section{Introdução}
\pp{} ok.
\pp{} ok.
\pp{} ok.
\pp{} ok.
\pp{} ok.
\PP{5}
\section{Planejamento e metodologia para atividade de design colaborativo}
\pp{} ok.
\pp{} ok.
\pp{} ok.
\pp{} ok.
\pp{} ok.
\pp{} ok.
\PP{6}
\pp{} ok.
\pp{} ok.
\section{Técnica de Brainstorming}
\pp{} ok.
\pp{} ok.
\PP{7}
\pp{} ok.
\pp{} ok.
\pp{} ok.
\pp{} ok. \VV{Pode também possuir ambas as etapas estruturadas e não estruturadas, certo?}
\pp{} ok.
\pp{} ok.
\PP{8}
\section{Metodologia Card Sorting para construção de Arquitetura de Informação}
\pp{} ok.
\pp{} ok.
\pp{} ok.
\pp{} ok.
Figura: ok.
\PP{9}
\section{Oficina de design colaborativo}
\pp{} ok. \VV{Algum destes encontros rolaram?}
\pp{} ok.
\pp{} ok.
\section{Metodologia distribuida baseada em Hangout (vídeo chamadas)}
\pp{} ok.
\pp{} ok.
\pp{} ok.
\pp{} ok.
\section{Conclusão}
\pp{} ok.
\pp{} ok.
\pp{} ok.
\pp{} ok.
\pp{} ok.
\pp{} ok.
\section*{Anexos}
Bastante interessante o ímpeto de buscar direções no coletivo para design e funcionalidades.
\chapter{Relatório da oficina de cardsorting ou atividade similar planejada no produto 1, e documento contendo a arquitetura de informação detalhada para o tema padrão do portal}
\PP{4}
\section{Introdução sobre o Projeto}
ok.
\PP{5}
\section{Sobre o desafio de aplicativos e documentos}
\pp{} Conceitos úteis para situar e legitimar SM e demais ferramentas: \VV{``ecossistema de metodologias e ferramentas de participação mediada por internet''} e \VV{rede de inovacao aberta}. Usados para endossar evento promovido pela SGPR e SNAS.
\pp{} \VV{que fim levou os aplicativos inscritos?}
\pp{} ok.
\pp{} ok.
\PP{6}
\pp{} ok.
\pp{} \VV{usado sociedade-governo}, o que é o mais comum nos produtos.
\pp{} ok. \VV{Há uma relação acessível das inscrições? Posso (ou outra pessoa) solicitar esta informação?}. \VV{As trilhas temáticas são excelente exemplos de questões caras ao Participa.br}.
\pp{} ok.
\pp{} ok.
\PP{7}
\pp{} ok.
\pp{} ok.
\pp{} ok.
\pp{} ok.
\pp{} ok.
\pp{} ok.
\pp{} ok.
\pp{} ok.
\pp{} ok.
\pp{} ok.
\PP{8}
\pp{} ok.
\pp{} ok. \VV{Emails para pedir infos}.
\pp{} ok.
\pp{} ok.
\pp{} ok.
\PP{9}
\section{Mídia kit}
Figuras e telas de referência: ok.
\section{Formulário de inscrição}
\pp{} ok.
\section{Banners de divulgação}
\pp{} ok.
\chapter{Documento com as propostas de wireframes, telas e
userstories para o tema padrão do portal contendo definições, orientações e códigos.}
\section{Introdução sobre o Projeto}
ok.
\PP{5}
\section{Ícones de Participação Social}
\pp{} ok.
\pp{} ok.
\pp{} ok.
\pp{} ok.
\PP{6}
\pp{} ok.
\pp{} ok.
\PP{10}
\section{Organização da pasta de ítens e código css - tema Participabr}
\pp{} \VV{dado caminho para as figuras participativas todas na arvore do noosfero.}
\pp{} ok.
\PP{14}
\pp{} \VV{paletas de cores}
\PP{15}
\section{Telas de lançamento do projeto participa.br}
Figuras: ok
\PP{21}
\section{Fotos da oficina de apresentação do projeto}
Fotos: ok.
\PP{25}
\section{Conclusão}
\pp{} ok.
\pp{} ok.
\pp{} ok.
\chapter{Relatório da oficina de cardsorting ou atividade similar
sugerida na metodologia descrita no produto 1, e documento contendo a arquitetura de
informação detalhada para o aplicativo de consulta.}
\section{Introdução}
ok.
\PP{5}
\section{Oficina de Design do Portal de Participação Social - Atividade Card Sorting}
\pp{} ok.
\pp{} ok.
\pp{} ok.
\pp{} ok.
Figura: ok.
\PP{6}
\section{Oficina na Casa das Redes}
\pp{} ok.
\pp{} ok.
\pp{} ok.
\pp{} ok. \VV{apontada hashtag \#psocial}.
\pp{} ok. \VV{está disponível esta sistematização das infos coletadas?}
\PP{7}
\pp{} ok.
\section{Participantes}
\pp{} ok.
\PP{9}
\section{Contribuições das Oficinas}
\pp{} 
\PP{15} 
\pp{} \VV{grupo de Belém prevê grupos de estudos para o portal.}
\PP{18}
\section{Banner de divulgação da oficina remota}
\PP{19}
\section{Arquitetura de informação presenciais digitalizadas}
\pp{} \VV{Grupo 1 prevê parte de informação saindo na home}
\PP{20}
\pp{} \VV{Grupo 2 prevê toda uma parte sobre acesso a conhecimentos e informação, incluindo pesquiasas}.
\PP{21}
\pp{} \VV{Grupo 3 prevê Pesquisas nas Produções, também prevê Democracia Direta nos Instrumentos}.
\PP{22}
\pp{} \VV{Grupo 4 traz proposta de início lúdico para introduzir o participante.}
\PP{23}
\pp{} \VV{Grupo 5 preve API, Metadados e código aberto. Também um observatório em um \#psocial na rede}.
\PP{24}
\section{Conclusão}
\pp{} ok.
\pp{} ok. \VV{Figura com um resumo visual de todas as contribuições, preve formações, espaços de criação, redes sociais, relatoria e ferramentas.}
\chapter{Documento com os exemplos de wireframes, telas e
userstories para o aplicativo de consulta, orientações e códigos}
\section{Introdução sobre o projeto}
\pp{} ok.
\PP{5}
\section{Wireframes do Portal da Participação Social}
\pp{} \VV{aba de Ciência e Tecnologia parece ter se perdido}. 
\pp{} outras boas ideias como ``redações coletivas''.
\pp{} \VV{Abas de comunidades podem ganhar aba de análises ou indicações de links e métodos para isso}.
\PP{35}
\section{Layouts do Portal}
\pp{} ok.
\PP{62}
\section{Conclusão}
\pp{} \VV{Aponta para templates .gov.br em plone: \url{https://github.com/plonegovbr/brasil.gov.temas}.}
\pp{} ok.
\pp{} ok.
\pp{} ok.
\chapter{06 - Documento com os exemplos do kit mídias para o aplicativo de consulta contendo definições, orientações e códigos; e compilado das propostas de conteúdos a serem produzidos durante a utilização do aplicativo de consulta, contendo orientações e exemplos de postagens}.
\section{Introdução}
\pp{} ok.
\pp{} ok.
\pp{} ok.
\pp{} ok.
\pp{} ok.
Figuras: ok.
\PP{4}
\pp{} ok.
\pp{} ok.
\section{Desenvolvimento}
\pp{} ok.
\pp{} ok.
\pp{} ok.
\pp{} ok.
\pp{} ok.
\pp{} ok.
\pp{} \VV{Como foi rastreado estes 14 milhões de impactos?}
\PP{5}
\pp{} ok.
\pp{} ok.
\pp{} ok.
\pp{} ok.
\pp{} ok.
\pp{} ok.
\pp{} ok.
\PP{6}
\subsection{Documento com exemplo do kit mídia para o aplicativo de consulta}
\pp{} ok.
\pp{} ok.
\pp{} ok.
\pp{} ok.
\pp{} ok.
\PP{7}
\pp{} ok.
\pp{} ok.
\pp{} ok.
\pp{} ok.
\pp{} ok.
\pp{} ok.
\pp{} ok.
\pp{} ok.
\pp{} ok.
\pp{} ok.
\pp{} \VV{Somente usa Open Sans e Arial}.
\PP{8}
\pp{} ok.
\pp{} ok.
\PP{9}
\pp{} ok.
\pp{} ok.
\pp{} ok.
\pp{} ok.
Paleta de Cores
\pp{} ok.
\pp{} ok.
\PP{10}
\pp{} ok.
\pp{} ok.
Figura 4 cores: ok.
\pp{} ok.
\pp{} ok.
\PP{11}
\pp{} \VV{Cores primárias e secundárias}.
\PP{12}
\pp{} ok. \VV{Cores alteradas na intensidade, claridade e saturação}.
\PP{14}
\subsection{Definição do Aplicativo de Consulta Pública}
\pp{} ok.
\pp{} ok.
\pp{} ok.
\pp{} ok.
\pp{} ok.
\pp{} ok.
\subsection{Orientação do Aplicativo}
\pp{} ok.
\PP{15}
\pp{} ok.
\pp{} ok.
\pp{} ok.
\pp{} ok.
\pp{} ok.
\PP{16}
\pp{} ok.
\pp{} ok.
\pp{} ok.
\pp{} ok.
\pp{} ok.
\pp{} ok. \VV{Como o desenrolar da consulta cuidou de agrupar propostas? Rejeitando todas menos uma?}
\pp{} ok.
\PP{17}
\pp{} ok.
\subsection{Exemplo de Códigos}
\pp{} ok.
\subsection{Compilado das propostas de conteúdo durante a utilização do aplicativo de consulta}
\pp{} ok. bem legal. \VV{Há contagem de temas? (a cada proposta foi associado um tema, os temas se repetem com concentrações)}.
\PP{38}
\subsection{Orientação para votação}
\pp{} ok. \VV{Seria bom ter identidade dos votantes ou analisar o padrão dos que votaram. Caso haja IP, pode-se considerar um votante por IP.}
\pp{} ok.
\pp{} ok.
\PP{39}
\pp{} ok.
\pp{} ok.
\subsection{Exemplo de postagem}
\pp{} ok.
\pp{} ok.
\PP{40}
\section{Uso de kit de mídias para disseminação}
\pp{} ok.
\pp{} ok. \VV{legal o iframe de incorporação}.
\pp{} ok. Curioso que é na proporção 1000:1 de votos para propostas, como de views para likes do youtube.
\PP{67}
\section{Resultado da consulta}
\pp{} ok.
\pp{} ok.
\pp{} ok.
\pp{} ok.
\pp{} ok.
\PP{68}
\pp{} Interessante o cruzamento com o texto do NETmundial. Gostaria também de uma contagem de incidência dos temas e de um relacionamento proposta-itens elencados neste final.
\PP{70}
\section{Conclusão}
\pp{} ok.
\PP{71}
\pp{} ok.
\pp{} ok.
\chapter{Documento com o planejamento, metodologia e relato da realização das atividades de design participativo para temas de novas comunidades no portal contendo a consolidação do resultado das atividades, exemplos e códigos dos itens afetados por elas.}
\pp{} ok.
\pp{} ok.
\pp{} ok.
\pp{} ok.
\pp{} ok.
\section{Propostas de Logo e Resultado Final}
\pp{} \VV{Ótimas logos para animação, basta fazer sucessão delas (pessoas amarelo e verde com braços levantados, balõezinhos, etc)}.
\PP{7}
Figura de resultado final: \VV{Fazer controlezinho de efeito nessa bonita logo. Plugin de Noosfero?}
\PP{8}
\section{Novos layouts do portal}
ok.
\PP{14}
\section{Mídia Kit e Banner para evento Conexões Globais}
ok.
\section{Perfis em Redes Sociais}
\pp{} ok.
\section{Apresentação do Participa.br}
\PP{34}
\pp{} {\bf Ampliar, qualificar e potencializar a participação}
\PP{40}
\section{Apresentação Consulta e Arena Participa.NET: proposta do Participa.br para o NETmundial}
ok
\PP{50}
\section{Design novo para as votações em pares}
\PP{53}
\section{Código e comunidade NETmundial}
\pp{} ok.
\PP{54}
\pp{} ok.
\pp{} ok.
\pp{} ok.
\pp{} ok.
\PP{67} 
\section{Conclusão}
\pp{} ok.
\chapter{08 - Documento com proposta de aplicativos para agregação de
conteúdos das redes (com orientações, exemplos e códigos), incluindo planejamento,
metodologia e relato da realização das atividades de design participativo para novos aplicativos do
portal contendo a consolidação do resultado das atividades, exemplos e códigos.”}
\PP{3}
\section{Introdução}
\pp{} ok.
\pp{} ok.
\pp{} ok.
\pp{} ok.
\pp{} ok.
\PP{4}
\section{Evento ARENA NETmundial}
\pp{} ok.
\pp{} ok.
\pp{} ok.
\subsection{Metodologia da ARENANETmundial}
\pp{} ok.
\PP{5}
\pp{} ok.
\pp{} ok.
\pp{} ok.
\pp{} ok, mas faltante os nomes dos hubs.
\pp{} ok.
\PP{6}
\subsection{Apresentação visual do HUB}
Figura: ok.
\PP{7}
\subsection{Relato da atividade HUB}
\pp{} ok.
\pp{} ok.
\pp{} ok. \VV{As hashtags foram usadas por cada hub como quiseram, certo? Há registro destas hashtags?}
\pp{} ok.
\subsection{Exemplo e código Noosfero}
\pp{} ok.
\PP{12}
\section{Comunidade NETMundial}
\pp{} ok.
\pp{} ok.
\pp{} ok.
\pp{} ok.
\subsection{Metodologia da Consulta Pública}
\pp{} ok.
\PP{13}
\pp{} ok.
\pp{} ok.
\pp{} ok.
\pp{} ok.
\subsection{Apresentação da interface de consulta pública PairWise}
\pp{} ok.
\PP{14}
\subsection{Exemplos de código Noosfero}
ok.
\PP{21}
\section{Comunidade MROSC}
\pp{} ok.
\pp{} ok.
\subsection{Atividades na Arena da Participação Social}
\pp{} ok.
\PP{22}
\pp{} ok.
\pp{} ok.
\pp{} que fim levou estes aplicativos e a maratona?
\pp{} ok.
\subsection{Processo de criação de design colaborativo}
\pp{} ok.
\subsection{Exemplos de código Noosfero}
ok.
\PP{24}
\section{Comunidade Política Nacional de Participação Social (PNPS)}
\pp{} ok.
\PP{25}
\pp{} ok.
\pp{} ok.
\pp{} ok.
\pp{} ok.
\subsection{Metodologia de Consulta Pública}
\pp{} ok.
\pp{} ok.
\pp{} ok.
\pp{} ok.
\pp{} ok.
\PP{26}
\pp{} ok.
\subsection{Apresentação visual da Consulta Pública}
Figura: ok.
\subsection{Exemplo e códigos Noosfero}
\PP{27}
\section{Comunidade Compromisso Nacional pela Participação Social (CNPS)}
\pp{} ok.
\pp{} ok.
\pp{} ok.
\subsection{Metodologia de Consulta Pública}
\pp{} ok.
\PP{28}
\pp{} ok.
\pp{} ok.
\subsection{Apresentação visual da consulta pública}
Figura: ok.
\subsection{Exemplo e códigos Noosfero}
\PP{29}
\section{Conclusão}
\pp{} ok.
\pp{} ok.
\pp{} ok.
\pp{} ok.

\part{Ana Célia da Silva Costa}
\chapter{Plano de trabalho contendo o detalhamento das estratégias de
pesquisa, interlocutores e referências para construção dos produtos}
\PP{3}
\section{Apresentação}
\pp{} ok.
\pp{} ok.
\PP{4}
\section{Editoria do ParticipaBR}
\pp{} ok.
\pp{} ok. \VV{quem serão os representantes do servidor?}
\pp{} ``\VV{O fluxo de trabalho deve ser diretamente influenciado pelo Monitoramento, já que ele mostrará o feedback de nossas ações e de temas que estejam em pauta nas redes.}''
\PP{5}
\section{Planejamento do Monitoramento de Temas}
\pp{} ok.
\pp{} \VV{elencados Participa, Facebook, Twitter, Google Plus e Instagram.}
\pp{} ok.
\pp{} \VV{estabelecidas tags por região, função e grupo.}
\PP{6}
\pp{} ok.
\pp{} ok.
\pp{} \VV{Followerwonk analsa perfil de Twitter}
\pp{} \VV{Métricas básicas do Participa.br nas redes sociais}
\pp{} \VV{KPI do Participa.br}
\PP{7}
\pp{} \VV{Há estes relatórios diários?}
\section{Ações propostas para realização até dezembro de 2014}
\subsection{Objetivo: debater temas}
\pp{} ok.
\pp{} ok.
\pp{} ok.
\pp{} ok.
\PP{8}
\pp{} \VV{Previstos twitaços e blogagens coletivas.}
\pp{} ok.
\pp{} ok.
\pp{} ok.
\PP{9}
\pp{} ok.
\pp{} ok.
\pp{} ok.
\pp{} ok.
\pp{} ok.
\pp{} ok.
\pp{} ok. \VV{Sugerida corrente de pessoas enviando fotos delas com a tag \#participabr}
\pp{} ok.
\pp{} ok.
\pp{} ok.
\PP{10}
\pp{} \VV{Previsto SM para ajudar a perceber temas em evidência e participantes.}
\pp{} \VV{Algum podcast produzido}
\pp{} ok.
\pp{} ok.
\pp{} ok.
\subsection{Objetivo: relacionar}
\pp{} \VV{Prevista intensificação da relação dos principais agentes das redes com o governo}.
\pp{} ok.
\pp{} ok.
\pp{} ok. \VV{Proposto registro do relacionamento com os atores da rede, para montar histórico.}
\pp{} ok.
\pp{} ok.
\PP{11}
\pp{} \VV{Como está sendo vislumbrado este infográfico?}
\subsection{Objetivo: mobilizar}
\pp{} ok.
\pp{} ok.
\pp{} ok. \VV{Continua zero o page-ranking do participa?}
\pp{} ok.
\pp{} ok.
\pp{} ok.
\PP{12}
\section{Exercícios de conteúdo}
ok.
\chapter{02 - Documento com proposta detalhada da linha editorial do Portal da Participação Social bem como sua metodologia de implementação e manutenção}
\PP{3}
\section{Apresentação}
\pp{} ok.
\pp{} ok.
\pp{} ok.
\pp{} ok.
\pp{} ok.
\pp{} ok.
\PP{4}
\section{Linha editorial do Portal da Participação Social}
\pp{} ok.
\pp{} ok.
\pp{} ok.
\pp{} ok.
\pp{} ok.
\pp{} ok.
\pp{} ok.
\pp{} ok.
\PP{5}
\pp{} ok.
\pp{} ok.
\pp{} ok. \VV{Elencados pilares participativos}.
\PP{6}
\section{Metodologia}
\pp{} \VV{existe esse manual de redação? e o de boas práticas?}
\pp{} \VV{Manifesto interesse por observar articuladores-chave.}
\pp{} ok.
\pp{} ok. \VV{Há esta compilação de textos de fontes diversas sobre o Participa.br}.
\pp{} ok.
\pp{} ok.
\pp{} ok. \VV{Reforçada necessidade da aproximação da comunicação para que as comunidades tenham força.}
\PP{7}
Figura: ok
\PP{8}
\pp{} ok.
\pp{} ok.
\PP{9}
\section{Manutenção}
\pp{} ok. \VV{Apontada importância de destaque de postagens no portal, pensando em recuperar o servidor}.
\pp{} ok.
\pp{} ok.
\pp{} \VV{Essa reunião mensal de auto-avalição ocorre?}
\PP{10}
\section{Redes Sociais e Participação Social}
\pp{} ok.
\pp{} ok.
\pp{} ok.
\pp{} ok. \VV{Para o relacionamento digital, seria pertinente que o perfil do participa travasse diálogos/discussões nas redes, não?}
\pp{} ok.
\PP{11}
\pp{} ok. \VV{Apontada experiência que interage diariamente com a população}.
\PP{12}
\pp{} ok.
\pp{} ok.
\PP{13}
\pp{} ok.
\pp{} ok.
\PP{15}
\pp{} ok.
\pp{} ok.
\PP{16}
\pp{} ok.
\pp{} ok.
\PP{20}
\pp{} ok.
\chapter{Documento contendo boas práticas para disseminação de conteúdos nas plataformas de redes sociais e blogs considerando as especificidades da linha editorial do Portal da Participação Social}
\PP{2}
\section{Apresentação}
\pp{} ok.
\PP{3}
\pp{} ok.
\pp{} ok.
\section{Estratégia de conteúdo nas redes sociais}
\pp{} ok.
\PP{4}
Foto: interessante com parcelas da audiência global por continente.
\pp{} ok.
\pp{} ok.
\pp{} ok.
\pp{} ok.
\PP{5}
\pp{} ok.
\pp{} ok.
Figura ok.
\pp{} ok.
\pp{} ok.
\pp{} ok.
\PP{6}
Figura: \VV{A legenda diz que é \% da população, mas a figura aponta horas online. Assumi que são horas online por semana por pessoa}.
\pp{} ok.
\pp{} ok.
\pp{} ok.
\pp{} \VV{Previsto que o processo final tem como base resultados de softwares de monitoramento}
\pp{} ok.
\PP{7}
Figura: ok.
\pp{} \VV{Como assim segmentar os anúncios no FB?}
\pp{} ok.
\pp{} \VV{Proposto que o Participa.br seja tb referência no relacionamento internacional.}
\section{Estratégia de conteúdo nos blogs}
\pp{} ok.
\pp{} ok.
\pp{} ok.
\PP{8}
\pp{} ok.
\pp{} ok.
\pp{} ok.
\pp{} ok.
\section{Exemplo de ações do Portal da Participação em redes sociais e blogs: Case Arena NETmundial}
ok.

\part{Produtos de Paulo Meirelles}
\chapter{01 - Levantamento de informações e proposta de catalogação e
sistematização de conteúdos sobre participação social: Proposta para criação de uma biblioteca digital de participação social}
Este documento refere-se a um dos produtos relacionados no Termo de Referência BRA/12/018 sob
coordenação da Secretaria Geral da Presidência da República voltado para o desenvolvimento de meto-
dologias de articulação e gestão de políticas públicas para promoção da democracia participativa. Esse
relatório faz parte da estratégia de geração de conhecimento para levantamento, organização e cataloga-
ção de conteúdos de participação social disponibilizados em diversos suportes e fontes de informação.
Para esse objetivo, propõe-se aqui a construção de uma biblioteca digital distribuída, tomando o portal
Participa.br como um provedor de serviços e os canais de participação social como provedores de dados,
cuja interoperabilidade ocorre por meio do protocolo OAI-PMH. Além da identificação da proposição
citada, foram gerados os seguintes resultados preliminares: (i) uma abordagem teórica sobre participa-
ção social, tomando como base os pressupostos da Ciência da Informação, (ii) uma prova de conceito
na forma de uma biblioteca digital (centralizada) experimental, alimentada por uma amostra de docu-
mentos (iii) uma compilação sobre produtos de software livre e soluções tecnológicas para a construção
de bibliotecas digitais, e (iv) uma proposta de ampliação das funcionalidades do portal Participa.br para
torná-lo um espaço colaborativo de participação social com opção de monitoramento de informações
sobre políticas públicas.
\PP 8
\section{Introdução}
\p ok.
\p ok.
\p ok.
\p ok.
\p ok.
\p ok.
\p ok.
\PP 9
\section{Ciência da informação e informação sobre participação social}
\p ok.
\p ok.
\p ok.
\subsection{Dado e informação de participação social}
\p ok.
\p ok.
\PP 10
\p ok.
\p ok.
\subsection{Tipos de informação sobre participação social}
\p ok.
\p ok.
Tabela 1: ok.
\p ok.
\PP 11
Figura 1: ok.
\p ok.
\p ok.
\p ok.
\p ok.
\p ok.
\PP 12
\p ok.
Figura 2: ok.
\p ok.
\p ok.
\PP 13
Figura 3: ok.
\p ok.
\subsection{Alternativas de representação da realidade de participação social}
\p ok.
\p ok.
\p ok.
\PP 14
\p ok.
\p ok.
\subsection{Formatos para respresentação de dados de participação social}
\p ok.
\p ok.
\PP 15
\p ok.
\subsection{Padrões de metadados para catalogação e sistematização de dados}
\p ok.
\p ok.
\p ok.
\p ok.
\p ok.
\p ok.
\PP 16
\p ok.
Figura 4: ok.
\p ok.
\PP 17
Figura 5: ok.
\p ok.
\PP 18
Figura 6: ok.
\subsection{Estratégias de recuperação da informação sobre participação social}
\p ok.
\p ok.
\PP 19
\p ok.
\p ok.
\p ok.
\p ok.
\PP 20
Figura 7: ok.
\p ok.
\p ok.
\p ok.
\PP 21
\p ok.
\p ok.
\PP 22
Figura 8: ok.
\p ok.
\p ok.
\PP 23
\section{Uma biblioteca digital de participação social}
\p ok.
\p ok.
\p ok.
\subsection{Conceituando bibliotecas digitais}
\p ok.
\p ok.
\PP 24
\p ok.
\p ok.
Figura 9: ok.
\PP 25
Figura 10: ok.
\p ok.
\PP 26
\p ok.
\p ok.
\subsubsection{Arquitetura interna de bibliotecas digitais}
\p ok.
\p ok.
\PP 27
Figura 11: ok.
\p ok.
\p ok.
\PP 28
Figura 12: ok.
\p ok.
\subsubsection{Entregáveis em um projeto de biblioteca digital}
\p ok.
\p ok.
\PP 29
Figura 13: legal esta disposição dos entregáveis.
\p ok.
\subsubsection{Alternativas de softwares para construção de bibliotecas digitais}
\p ok.
\p ok.
\PP{30}
\p ok.
\p ok.
\p ok.
\p ok.
\PP{31}
\p ok.
Figura 14: ok.
\p ok.
\PP{32}
\p ok.
\p ok.
\p ok.
\p ok.
\p ok.
\PP{33}
\p ok.
\p ok.
\p ok.
\p ok.
\p ok.
\PP{34}
\p ok.
\p ok.
\p ok.
\subsection{Uma prova de conceito para biblioteca digital}
\p ok.
\p ok.
\subsubsection{Arquitetura da informação de participação social}
\p ok.
\PP{35}
\p ok.
\p ok.
\p ok.
\p ok.
\p \VV{Das ONGs, o siconv pode resolver para o catálogo pretendido.}
\PP{36}
\p ok.
\subsubsection{Metadados de participação social}
\p ok.
\PP{37}
Tabela 2: ok.
\p ok.
\PP{38}
Tabela 3: ok.

Tabela 4: ok.

Tabela 5: ok.
\PP{39}
\p ok.
\p ok.
\p ok.
\subsubsection{Arquitetura interna da biblioteca digital de participação social}
\p ok.
\p ok.
\PP{40}
Figura 15: ok.
\p ok.
\p ok.
\PP{41}
\p ok.
\subsubsection{Uma proposta de interface para a biblioteca digital}
\p ok.
\p ok.
\p ok.
\PP{42}
\p ok.
\p ok.
\p ok.
Figura 16: ok.
\p ok.
\p ok.
\PP{43}
Figura 17: ok.
\p ok.
Figura 18: ok.
\p ok.
\p ok.
\PP{44}
Figura 19: ok.
\p ok.
\PP{45}
\p ok.
Figura 20: ok.
\p ok.
\PP{46}
\p ok.
\p ok.
\p ok.
Figura 21: ok.
\PP{47}
\p ok.
Figura 22: ok.
\p ok.
\p ok.
\p ok.
\PP{48}
Figura 23: ok.
\p ok.
Figura 24: ok.
\PP{49}
\p ok.
Figura 25: ok.
\p ok.
Figura 26: ok.
\p ok.
\PP{50}
\p ok.
Figura 27: ok.
\p ok.
Figura 28: ok.
\p ok.
\PP{51}
Figura 29: ok.
\p ok.
Figura 30: ok.
\p ok.
Figura 31: ok.
\p ok.
\PP{52}
Figura 32: ok.
\p ok.
Figura 33: ok.
\p ok.
\p ok.
\p ok.
\PP{53}
Figura 34: ok.
\p ok.
\p ok.
Figura 35: ok.
\PP{54}
\section{Estendendo a biblioteca digital do Participa.br}
\p ok.
Figura 36: ok.
\p ok.
\PP{55}
\p ok.
\p ok.
\p ok.
\PP{56}
Figura 37: ok.
\p ok.
\p ok.
\p ok.
\p ok.
\PP{57}
\p ok.
\p ok.
\p ok.
\p ok.
\PP{58}
\section{Considerações finais}
\p ok.
\p ok.
\p ok.
\p ok.
\p ok.
\PP{59}
\p ok.
\p ok.
\section{Anexos}
ok.
\chapter{03 - Proposta de metodologia de organização da informação, formulários e demais ferramentas que fomentem a sistematização de metodologias de participação social de forma clara e reutilizável no portal de participação social}
Este documento refere-se ao terceiro produto relacionado no Termo de Referência BRA/12/018 sob a coordenação da Secretaria Geral da Presidência da República voltado para o desenvolvimento de metodologias de articulação e gestão de políticas públicas para promoção da democracia participativa. Nesse relatório é proposta uma redefinição da biblioteca digital de participação social relacional e aberta, considerando um alicerce de informações sobre os objetos de informação que contempla (i) o registro bibliográfico dos objetos de informação, (ii) o registro de autoridades ou pessoas vinculadas a esses objetos, e (iii) um arranjo de assuntos que sirvam como índices que facilitam o acesso a esses documentos. Adicionalmente, foi previsto um esquema de anotações como uma alternativa para que a comunidade de usuários da biblioteca possa contribuir com a identificação dos
conceitos inerentes à organização dos conteúdos de participação social. Esses
elementos são colocados como subsídios para sistematização de metodologias de
participação social e oferecem elementos e artefatos necessários para sistematizar e
organizar as informações de participação social focada em relacionamentos.

\PP 8
\section{Introdução}
\p ok.
\p ok.
\p ok.
\p ok.
\PP 9
\subsection{Problema de pesquisa}
\p ok.

Figura 1 - ok

\p ok.
\PP{10}
\p ok.
Figura 1.2 - ok.
\subsection{Objetivos}
\p ok.
\PP{11}
\subsection{Justificativa}
\p ok.
\p ok.
\p ok.
\PP{12}
\PP{13}
\section{Revisão de literatura}
\p ok.
\p ok.
\p ok.
\p ok.
\PP{14}
\p ok.
Figura 2.1 - ok.
\subsection{Dados ligados e abertos}
\p ok.
\PP{15}
\p ok.
\p ok.
Figura 2.2 - ok.
\p ok.
\PP{16}
\p ok.
Figura 2.3 - ok.
\p ok.
\p ok.
\p ok.
\PP{17}
Figura 2.4 - ok.
\p ok.
\p ok.
\p ok.
\PP{18}
Figura 2.5 - ok.
\p ok.
\subsection{Localizando significado para conteúdos do linked data}
\p ok.
\PP{19}
\p ok.
\subsection{Tendências e padrões de catalogação}
\p ok.
\PP{20}
\subsubsection{Modelos conceituais para catalogação}
ok. Figura interessante sobre as três camadas do FRBR.
\PP{22}
\subsection{RDA - Novo padrão de catalogação}
\p ok
\p ok




\part{Produtos de Daniela Soares Feitosa}
\chapter{01 - Documento técnico com complementação da documentação de instalação e uso da plataforma Noosfero contendo conceitos e tutoriais}
\PP 5
\section{Apresentação}
\p ok.
\p ok.
\p ok.
\section{Conceitos}
\p ok.
\p ok.
\subsection{Linguagem Ruby}
\p ok.
\p ok.
\PP 6
Figura 1: ok.
\p ok.
\p ok.
\PP 7
\subsection{Framework Rails}
\p ok.
\p ok.
\p ok.
\p ok.
\subsubsection{Arquitetura MVC}
\p ok.
\p ok.
\PP 8
Figura 2: ok.
\subsection{Varnish}
\p ok.
\p ok.
\subsection{Memcached}
\p ok.
\subsection{PostgreSQL}
\p ok.
\p ok.
\subsection{Licenças de software}
\p ok.
\PP 9
\p ok.
\subsubsection{GNU Affero General Public License}
\p ok.
\p ok.
\section{Documentação de instalação e utilização do Noosfero}
\p ok.
\subsection{Instalação}
\p ok.
\subsubsection{Obter o pacote Debian}
\p \VV{Há pacote para Ubuntu?}
\PP{10}
\subsubsection{Configuração do pacote Debian}
\p ok.
Figura 3: ok.

Figura 4: ok.
\PP{11}
Figura 5: ok.

Figura 6: ok.
\p ok.
\PP{12}
\subsubsection{Configuração inicial do ambiente Noosfero}
\p ok.
\p ok.
\subsection{Utilização}
\p ok.
\PP{13}
\p ok.
\PP{14}
\section{Considerações finais}
\p ok.
\p ok.
\p ok.
\chapter{02 - Documento com proposta para desenvolvimento do código do tema padrão,
painel de controle e administração do portal contendo exemplos e códigos.}
\section{Introdução}
\p ok.
\p ok.
\p ok.
\p ok.
\p ok.
\PP 6
\section{Apresentação}
\p ok.
\p ok.
\p ok.
\p ok.
\p ok.
\section{Tema padrão do Portal de Consulta Pública}
\p ok.
\p ok.
\p ok.
\subsection{Barra de identidade do Governo Federal na Internet}
\PP 7
\p ok.
\p ok.
\p ok.
\subsection{Cabeçado do Portal da Participação}
\p ok.
\p ok.
\p ok.
\PP 8
\subsection{Barra de Participação Social}
\p ok.
\p ok.
\p ok.
\p ok.
\p ok.
\subsection{Bloco de categorias com relevância}
\p ok.
\PP 9
\p ok. \VV{Caso estas categorias estejam em uso, pode ser pertinente que esteja no quarto produto meu uma proposta para definir a relevância}.
\p ok.
\subsection{Bloco de estatísticas}
\p ok. \VV{Contemplar também o bloco de estatísticas no produto quatro}.
\PP{10}
\p ok. \VV{Como anda essa personalização do bloco de estatísticas?}
\p ok.
\subsection{Bloco de notícia destaque}
\p ok.
\subsection{Bloco de trilhas}
\p ok.
\p ok.
\PP{11}
\p ok.
\PP{12}
\subsection{Blocos de notícias da capa}
\p ok.
\subsection{Bloco de membros}
\p ok.
\p ok.
\PP{13}
\subsection{Bloco de mapa}
\p ok. \VV{Este bloco de mapas está operante? Consegue mostrar somente as pessoas de uma comunidade ou que tenham participado de alguma trilha ou etapa?}
\p ok.
\p ok.
\PP{14}
\p ok. \VV{Porquê o mapa (iframe) não pode ser adicionado ao contexto de ambiente?}
\subsection{Bloco de comunidades}
\p ok.
\subsection{Bloco de tags}
\p ok.
\PP{15}
\subsection{Bloco de videos}
\p ok.
\p ok.
\p ok.
\PP{16}
\subsection{Folha de estilo - CSS}
\p ok.
\p ok.
\p ok.
\p ok.
\p ok.
\section{Considerações finais}
\p ok.
\p ok.
\p ok.
\section{Apendices}
ok.
\chapter{Documento com proposta para desenvolvimento do código do aplicativo de consulta pública, do código de integração dele com o portal e do painel de controle e administração, contendo exemplos e códigos}
\PP 6
\section{Apresentação}
\p ok.
\p ok.
\p ok.
\p ok.
\p \VV{Etiqueta e status dos comentários já operantes. Ver como acessar estas etiquetas para treinar os classificadores}
\p ok.
\PP 7
\section{Plugin de classificação de comentários}
\p ok. \VV{Buscar as diferenças entre etiquetas, status e suporte neste plugin}
\p ok.
\p ok.
\p ok.
\p ok.
\p ok.
\PP 8
\section{Etiquetas}
\p ok.
\subsection{Definição}
\p ok.
\p \VV{o Profile é de quem definiu a etiqueta ou de quem associou a etiqueta com o comentário?}
\p ok.
\p ok.
\subsection{Administração}
\p ok.
\p ok.
\PP 9
\p ok.
\subsection{Utilização}
\p ok. \VV{não lembro deste campo, está habilitado?}
\p ok. \VV{para os classificadores e para algumas práticas de monitoramento, seria pertinente que cada comentário pudesse ter mais de uma etiqueta.}
\PP{10}
\p ok.
\section{Status}
\p ok.
\PP{11}
\subsection{Definição}
\p ok.
\p ok.
\p ok.
\p ok.
\PP{12}
\subsection{Administração}
\p ok.
\p ok.
\PP{13}
\subsection{Utilização}
\p ok.
\p ok.
\p ok.
\section{Apoio}
\p ok.
\PP{14}
\subsection{Definição}
\p ok.
\p ok.
\PP{15}
\subsection{Utilização}
\p ok.
\section{Personalização por perfil}
\p ok.
\p ok.
\p ok.
\PP{16}
\section{Estatísticas das classificações}
\p ok. \VV{Gostaria de ver estas estatísticas geradas automaticamente com as classificações, posso?}
\subsection{Label}
\p ok.
\subsection{Status}
\p ok.
\subsection{Support}
\p ok.
\p ok.
\section{Exportação dos dados}
\p ok.
\p ok.
\PP{17}
\section{Considerações finais}
\p ok.
\p ok.
\p ok.
\section{Appendices}
ok.
\chapter{05 - Documento com proposta para desenvolvimento do código do tema das comunidades contendo exemplos e códigos.}
\PP 5
\section{Introdução}
\p ok.
\p ok.
\p ok.
\p ok.
\p ok.
\PP 6
\section{Apresentação}
\p ok.
\p ok.
\p ok.
\p ok.
\section{Personalização de comunidades no Participa.br}
\p ok. \VV{``No Noosfero,..., o foco é rede social para produção e publicação de conteúdo na Internet''}.
\p ok.
\p ok.
\p ok.
\subsection{Alteração do template}
\p ok.
\PP 7
\p ok.
\p ok.
\subsection{Organização de blocos}
\p ok.
\p ok.
\subsection{Edição de cabeçalho e rodapé}
\p ok.
\PP 8
\subsection{Alteração do tema}
\p ok.
\section{Proposta para desenvolvimento do código do tema das comunidades contendo exemplos e códigos}
\p ok.
\p ok.
\PP 9
\p ok.
\p ok.
\p ok.
\PP{10}
\subsection{Cabeçalho da comunidade com a identidade visual do Portal}
\p ok.
\p ok.
\p ok.
\p ok.
\PP{11}
\subsection{Cabeçalho e rodapé da comunidade}
\p ok.
\p ok.
\PP{12}
\subsection{Bloco de links}
\p ok.
\p ok.
\p ok.
\PP{13}
\subsection{Menu para edição da comunidade}
\p ok.
\subsection{Folha de estilo - CSS}
\p ok.
\p ok.
\p ok.
\p ok.
\section{Considerações finais}
\p ok.
\p ok.
\p ok.
\p ok.
\section{Apendices}
ok.
\chapter{06 - Aplicativos de agregação de conteúdos das redes}
O presente produto descreve uma proposta de código para agregar o conteúdo do Portal
de Participação social com textos publicados no Twitter e Facebook. A ideia fundamental é
que as pessoas que já utilizem suas ferramentas de redes sociais e criação de conteúdo também
possam contribuir com os debates propostos no Portal de Participação mesmo sem a necessidade de fazer registro no portal. O código foi implementado e incorporado na plataforma
utilizada no portal (Noosfero) e já utilizado em debates durante eventos.
\section{introdução}
\subsection{Contexto e importância da consultoria}
\p ok.
\p ok.
\p ok.
\p ok.
\p ok.
\PP 7
\p ok.
\subsection{Contexto e importância do produto}
\p ok.
\p ok.
\p ok.
\p ok.
\p ok.
\p ok.
\PP 8
\p ok.
\section{Desenvolvimento}
\p ok.
\p ok.
\p ok.
\p ok.
\p ok.
\p ok.
\p ok.
\p ok.
\PP 9
\p ok.
\p ok.
\p ok.
\PP{10}
\subsection{Agregador de comunidade - Hub}
\p ok.
\p ok.
\p ok. \VV{Usando stream API do Twitter, certo? E Graph API 1 do FB?}
\PP{11}
\p ok.
\PP{12}
\p ok. \VV{Estes tweets estão no BD do Participa.br? Podemos analisá-los?}
\p ok.
\subsection{Arquitetura do plugin}
\p ok.
\p ok.
\PP{13}
\p ok.
\p ok.
\PP{14}
\section{Conclusão}
\p ok.
\p ok.
\p ok. \VV{Só apreendi o funcionamento do plugin integrador com as redes sociais. Há outro?}
\chapter{Plugin para triplificação dos dados do Portal de Participação Social}
Esse documento descreve uma proposta de código para gerar a triplificação em formato
RDF dos dados do Portal de Participação Social e sua disponibização em um endereço fixo.
O script para triplificação foi desenvolvido como produto do consultor Renato Fabbri e priorizado para desenvolvimento nas reuniões do projeto. O código foi implementado como plugin
para ser incorporado na plataforma utilizada no portal (Noosfero) e, para isso, foram necessárias algumas alterações no script original.
\section{Introdução}
\subsection{Contexto e importância da consultoria}
\p ok.
\p ok.
\p ok.
\p ok.
\p ok.
\PP 7
\p ok.
\subsection{Contexto e importância do produto}
\p ok.
\p ok.
\p ok.
\p ok.
\p ok.
\p ok.
\section{Desenvolvimento}
\p ok.
\PP 8
\p ok.
\p ok.
\p ok.
\p ok.
\subsection{Alterações no script de triplificação}
\p ok.
\PP 9
\p ok.
\p ok.
\p ok.
\p ok.
\subsection{Arquivos necessários para a execução}
\p ok.
\subsection{Arquitetura do plugin}
\p ok.
\p ok.
\PP{10}
\p ok. \VV{ótimo de triplificar dados de todos os ambientes configurados. Talvez valha convencionarmos uma URI para cada ambiente.}
\p ok.
\p ok.
\p ok.
\PP{11}
\section{Conclusão}
\p ok. \VV{Para o Noosfero, pode ser ainda melhor se a triplificação de cada dados ocorrer no momento em que o dado é adicionado. Talvez isso possa ser feito com facilidade, se o Noosfero tiver um mecanismo central (classes, funções) de escrita de dados, o que é provável.}
\p \VV{Está planejada a revisão e aprofundamento desta triplificação para daqui 1 ou 2 meses}
\p ok.


\part{Notas aleatórias}

\chapter{Sistema de navegação e aproveitamento}
Dashboards/análises dos recursos e conjuntos/classes de recursos, recomendações, dicas de uso/utilidade de cada recurso ou método de recomendações. Ver possibilidade de navegação em RDF destes dashboards, via elda e via browsers de dados mesmo.
Elaborar visualização destes dashboards. Talvez até mais importante: elaborar renderização (genérica talvez) do dashboard em um som e uma imagem, ressaltando inclusão, arte e recursos audiovisuais.

\chapter{Criação de aplicativos para as trilhas e o Participa.br em geral}
Há foco em RC e PLN. Principalmente através de RC, entram todos os dados de contexto (profissão, idade, padrão de uso, etc), pois as redes são grafos com contexto.

Para PLN, pode-se focar em 3 possibilidades: 1) tamanhos de mensagens, frases e palavras; 2) incidência em listas de palavras (ou tokens); e 3) em estruturas sintáticas. Outras possibilidades são: semelhança com algum texto ou a produção do próximo usuário, por critérios semânticos ou bag of words; incidência de classes gramaticais, como plurais, feminino/masculino, voz passiva ou verbo no infinitivo; uso de letras em absoluto ou em posições diferentes.

Para a classificação geral, as entidades são: rede de amizades, rede de interação, indivíduo, postagens, comentários.

De escrita de documentos; de difusão de informação para amadurecimento da causa; de interação entre os participantes.

{\bf Nas comunidades, trilhas, perfis, etc, deixar junto ao contexto um link, redirecionando para uma análise da entidade sendo visitada. Para isso, um plugin do Noosfero pode ser feito, o qual recebe uma URL para a instância do analisador. Este analisador, ao receber a chamada visita, já com o tipo e id da entidade, apresenta uma página com a análise dos dados disponíveis no endpoint Jena. Retorna só o Json processado no server Flask. Retorna Json cru no endpoint com a query.} Talvez liberar também código de integração para usar num iframe dentro do Noosfero.

Georeferenciar participantes. As participações pelo IP também.

Disponibilizar no perfil ou funcionalidades de mobilização, sugestão de pessoas para que ela as contate para mobilização, com explicação dos critérios de rede e linguísticos usados. Também, mais por questões lúdicas, apontar pessoas com características similares: vocabulário, padrão de uso, de interação, data de criação de perfil, etc.

Seção ``Participa.br em números'' ou ``dimensionamento do Participa.br'' com número de comentários, postagens, interações, fotos, amizades, pessoas, etc.

Classificar usuários (e conteúdos) pela geolocalização (ao menos por estado e região).

\chapter{Participa.br no GSoC}
Integra o Participa.br em circuitos de SL e liberam bolsas de 10 mil por 3 meses para tarefas combinadas. Garantidamente, a comunidade do Participa.br vai levantar várias ideias para o Participa.br e o Noosfero: implementar tickets ou ideias no Noosfero, para o Participa ou para o SM. Triplificar dados do cidade democrática, aprofundar e implementar classificadores e plugins. Talvez juntar no Participa.br o noosfero/colivre, o lm, o crânio, o thacker, etc.

\chapter{Atribuição de papéis}
Os intermediários podem ser chamados a avaliarem a produção da comunidade, pois não são víscerais como os hubs nem lateral como os periféricos.

Periféricos podem ser chamados para darem substantivos, conceitos, coisas.
Hubs convocados para qualificar estas coisas. Intermediários para formarem texto(s).
\chapter{Simulação de processos}

Difusão, contágio, acessibilidade e fofoca nas redes reais do Participa.br e de outras instâncias.

\chapter{Arte Participativa}
Os materiais e análises participativas podem ser encaminhados para artistas de atuação nacional (e.g. Hermeto Pascoal) e internacional (e.g. Philip Glass), para que haja produção artística participativa e haja mais interesse sobre o tema.

Interfaces participativas podem incorporar musicalização de estruturas e dados participativos.

Podem haver concursos periódicos de criação com os dados do Participa.br.
\chapter{Criação de benchmarks participativos}
Com medidas de experiências participativas (quantidade de participantes, médias, interações, etc)

\chapter{Triplificação dos dados participativos}
Recomendar p equipe de dev do noosfero que gerem URLs para acessar comentários específicos e outros recursos em conformidade com as URIs usadas na triplificação dos dados do Participa.br.

Triplificar os dados do Cidade Democrática e do Participatório.

Triplificar os dados da consulta do arenaNETmundial.

Triplificar os dados das experiências do RS.

Triplificar Tweets participativos meus, do Malini do Pimentel e de outros.

Ver onde são acrescentados dados ao BD, ver se vale fazer hack ali para enviar/remover dados em ambos o BD Postgresql e triple store rdf.

\part{Cursos visitados nesta última etapa}
4 de redes complexas:
\begin{itemize}
    \item Social Network Analysis: \url{https://class.coursera.org/sna-003}
    \item Networked Life: \url{https://class.coursera.org/networks-002}
    \item Networks: Friends, Money, and Bytes (REGISTROS INACESSÍVEIS): \url{https://www.coursera.org/course/friendsmoneybytes}
    \item Social and Economic Networks: Models and Analysis: \url{https://class.coursera.org/networksonline-002}
    \item Networks Illustrated: Principles without Calculus: \url{https://class.coursera.org/ni-002/lecture}
\end{itemize}
2 de pln:
\begin{itemize}
    \item Natural Language Processing (Michael Collins): \url{https://class.coursera.org/nlangp-001}
    \item Natural Language Processing (Jurafsky e Manning): \url{https://www.coursera.org/course/nlp}
\end{itemize}
1 d hci:
\begin{itemize}
    \item \url{https://class.coursera.org/hciucsd-005}
\end{itemize}
1 d design
1 d ciência da comunicação:
\begin{itemize}
    \item https://class.coursera.org/commscience-004
\end{itemize}
1 d Reactive Programming?:
\begin{itemize}
    \item https://class.coursera.org/reactive-001
\end{itemize}
1 de estatística?:
\begin{itemize}
    \item https://class.coursera.org/statistics-001/lecture
    \item https://class.coursera.org/introstats-001
    \item https://class.coursera.org/statinference-008/lecture (4 semanas)
\end{itemize}
1 de dados linkados?:
\begin{itemize}
    \item https://class.coursera.org/bigdata-edu-001
    \item https://class.coursera.org/metadata-002
\end{itemize}
Big data?:
\begin{itemize}
    \item https://class.coursera.org/datasci-002 (fala de map reduction)
    \item https://class.coursera.org/bigdata-004/lecture (tb fala de map reduction, linked data e grafos)
\end{itemize}
\chapter{Cursos (Coursera)}
\section{Social Network Analisys}
\subsection{Week 1 - Intro}
Sonia software para redes em evolução.

Além das comunidades, achar as componentes conectadas (fortes e fracas e a componente gigante).

Há bons videos mostrando como se usa o Gephi.
\subsection{Week 2 - Random Graph Models}
Bons exemplos de percolação.
\subsection{Week 3 - Centrality}
Info nova: betweenness é correlacionado com inovação/criatividade do agente, também com a diversidade e qualidade de informação recebida.

Aplicações são interessantes: detectar especialistas, diminuir espalhamento de doenças.

Componentes bowtie: core, in (pode entrar no core, mas não sair) e out (pode sair do core, mas não entrar).

Pagerank não foi ideal para achar especialistas.

Há desvios da lei livre de escala para redes sociais por causa da limitação na quantidade de pessoas que coneguimos interagir/conhecer (o setor de alta conectividade pode ser aproximado por uma exponencial). Por haver uma tendência natural para a lei livre de escala, mas uma limitação da realidade na qual a rede existe, pode ser interessante pensar em formas de expandir essa capacidade comunicativa humana (agrupando pessoas em comunidades, gerando resumos, bots, metaindivíduos, etc).
\subsection{Week 4 - Community}
Comunidades são, em geral, reflexo de uma estrutura multimodal (e.g. lugares q a pessoa frequenta ou interesses que possui, os vértices são pessoas e estes lugares/instâncias de contato).

Cliques, K-Core, N-Cliques (closeness), p-cliques (proporção dos vértices).

Limpeza: eliminar arestas não recíprocas ou abaixo de um limiar de peso.

hierarquical clustering para comunidades. Betweenness. Modularidade. Cliques com sobreposição.

Infomap e mapequation para detecção de comunidades e interface online.

Para o usuário: explicitar quais caminhos dele para outras pessoas na comunidade, ou entre 2 pessoas quaisquer (shortest path), tanto no Participa quanto no twitter e no FB. Apontar habilidades (como o linkedin). Apontar atividade de amigos e das pessoas mais distantes da rede. Detecção de usuários que mais usaram palavras específicas (ou expansões feitas com sinônimos ou wordnet).
\subsection{Week 5 - Small World Networks}
Calcular, para o usuário, qual a distância máxima para outros vértices, e quem são eles. Assim como alguma divisão dos mais próximos e intermediários.

Calcular o índice de clusterização para o participante, apontando os triângulos formados e os pares separados. Relações embutidas e apontamento das relações potencialmente mais poderosas para difundir informação (relações intermediárias e fracas). Calcular motivos mais e menos incidentes das redes (perfil de motivos) e apontar quais de deles o participante integra.

Gephi tem plugin para streaming de redes.
\subsection{Week 6 - processes on networks}
simulação de difusão e outros processos. Collection action para simular adesão a protestos. Inovação. Resolução de problemas.

Simulação de contágio simples (por um agente) e complexo (mais agentes múltiplos).

Criar mecanismos de escolha de pessoas para influenciar, de forma a aumentar a repercussão na rede, com base nos modelos de contágio simples e complexo. Levar em consideração comunidades detectadas e cliques.

Mobilização e ações coletivas. Aprendizado não supervisionado para apreender padrões.

Análise de uso por gênero e horário.
\subsection{Week 7 - cool and unusual applications}
ok.
\subsection{Week 8 - network resilience}
Teste de resiliência (quantos hubs precisam ser removidos para partir a componente gigante ou aumentar o caminho mais curto médio).

Assortativity para a rede toda, para cada setor e para cada grau (plot 2d). Medidas de homofilia (genero, profissão, grau, atividade no site, idade, data de inscrição).

\section{Metadata: Organizing and Discovering Information}
\subsection{semana 1 - Organizing Information}

1-6 item e coleção - \VV{pensar em niveis de itens e coleção para os recursos do participa e outros: existe o dashboard da palavra ``árvore'' (item) e existe o dashboard das palavras em geral (coleção).}

Library of Congress Subject Headings \url{http://id.loc.gov/authorities/subjects.html} \VV{Para vocabulário: termos Political Participation e Social Participation} \url{http://id.loc.gov/authorities/subjects/sh85104370.html#concept} e \url{http://id.loc.gov/authorities/subjects/sh85104370.html#concept}. Também algo no modelo do Authority Headings pode ser feito para todo participante que se cadastrar no participa.br.

Tesauro e campos comuns.

Vocabulários controlados e não controlados.

``The map is not the territory'' - Alfred Korzybski

Metadado descritivo, estrutural ou administrativo (manutenção e direitos).
Niveis: Item ou coleção. Localidade: Embarcado ou conectado (embedded or linked). Audiência: humanos ou para leitura por máquina (human or machine-readable.

Informação como coisa, conhecimento (estado) ou como processo.
Pilares: informação, tecnologia, pessoas; HCI, HInformationI e Ciência da Informação. Dilema da existência da informação na ausência de um presenciador (como do gato de Schrödinger).


\subsection{semana 2 - Dublin Core}
Propósitos, histórico.

15 Core elements. Qualificação por refinamento de elemento e por ampliação do namespace. Comunidades de Dublin Core.
\subsection{semana 3 - }
\subsection{semana 4 - }
AAT, CDWA, CCO.
Usar CDWA e lite e AAT para classificar objetos de arte feitos com os dados participativos. TGM. TGN. Darwin Core. Premis para manutenção, provenance.
METS. Crosswalks.

Taxonomy warehouse.
\subsection{semana 5 - }
Wikidata para consultar campos considerados e estruturas, além de doação/catalogação dos dados com eles.

Fazer toda triplificação com a opa, com as URIs todas, e com schema.or, em uma tabela de crosswalk.

\subsection{semana 6 - }
Fazer uma tabela com sugestões para crosswalk, cada coluna uma ontologia (dbpedia, cyc, freebase, etc).

Estudar grafo geral do Linked Open Data.

\subsection{semana 7 - }
OAI-PMH.

Acionar o Open Archives Initiative.

Testar Elda e linked data browsers.


\VV{Ver o que tem da dbpedia na opa e triplificação do participa; ver como likar a dbpedia p classes d opa e ops e os dados da triplificação.}

Seguindo o que é feito com o AAT e outras instâncias, é pertinente estabelecermos colaborações interinstitucionais e internacionais para essas elaborações, mesmo que pela internet (sem deslocamento de ninguém).

\subsection{ semana 8 - }

\VV{Critérios simples para avaliar qualidade dos metadados e do compartilhamento}. A avaliação ajuda a situar o trabalho com o contexto e o propósito dele.

Compartilhamento e Flow Control.

\section{HCI}
\subsection{Semana 1 - Intro}
Interessante principalmente o histórico e testes de eficiência (survey, observação, teste comparativo, etc).
\subsection{Semana 2 - Needfinding}
ok. Bem interessante para coletar demandas e projetar o desenvolvimento.
\section{Communication Science}
Endereço: \url{https://class.coursera.org/commscience-004}.

7 semanas.
\subsection{Semana 1}
Visto videos da primeira semana, em especial com as três visões lineares (focada no efeito), de significados (Jacobson) e social (Newcomb).
\section{Developing Data Products}
Endereço: \url{https://class.coursera.org/devdataprod-004}

4 semanas.
\subsection{Semana 1}
Focado no R, Shiny, rCharts, Plotly, etc. Boas infos sobre GoogleVis.
\subsection{Semanas 2 e 4}
Não há videos da semana 3. As semanas 2 e 4 ficam ainda mais centradas na linguagem R: slides em R, pacotes, classes, etc.
\part{Livros lidos nesta última etapa}

\chapter{Linked Data: Evolving the Web into a Global Data Space}
\url{http://linkeddatabook.com/editions/1.0/#htoc84}
The World Wide Web has enabled the creation of a global information space comprising linked documents. As the Web becomes ever more enmeshed with our daily lives, there is a growing desire for direct access to raw data not currently available on the Web or bound up in hypertext documents. Linked Data provides a publishing paradigm in which not only documents, but also data, can be a first class citizen of the Web, thereby enabling the extension of the Web with a global data space based on open standards - the Web of Data. In this Synthesis lecture we provide readers with a detailed technical introduction to Linked Data. We begin by outlining the basic principles of Linked Data, including coverage of relevant aspects of Web architecture. The remainder of the text is based around two main themes - the publication and consumption of Linked Data. Drawing on a practical Linked Data scenario, we provide guidance and best practices on: architectural approaches to publishing Linked Data; choosing URIs and vocabularies to identify and describe resources; deciding what data to return in a description of a resource on the Web; methods and frameworks for automated linking of data sets; and testing and debugging approaches for Linked Data deployments. We give an overview of existing Linked Data applications and then examine the architectures that are used to consume Linked Data from the Web, alongside existing tools and frameworks that enable these. Readers can expect to gain a rich technical understanding of Linked Data fundamentals, as the basis for application development, research or further study.
\section{Preface}
\p ok.
\p ok.
\p ok.
\section{Introduction}
\subsection{The Data Deluge}
\p ok
\p ok.
\p ok.
\p ok.
\p ok.
\subsection{The Rationale for Linked Data}
\p ok.
\subsubsection{Structure Enables Sophisticated Processing}
\p ok.
\p ok.
\p ok. Microformats
\p ok. Programmable web, APIs.
\p ok.
\subsubsection{Hyperlinks Connect Distributed Data}
\p ok.
\p ok.
\p ok.
\subsection{From Data Islands to a Global Data Space}
\p ok.
\p ok.
\p ok.
\p ok.
\p ok. A essencia do linked data.
\p ok.
\p ok.
\p ok.
\p ok.
\subsection{Introducing Big Lynx Productions}
\p ok.
\p ok.
\p ok.
\section{Principles of Linked Data}
\p ok.
\p ok.
\p ok.
\p ok.
\p ok.
\subsection{The Principles in a Nutshell}
\p ok.
\p ok.
\p ok.
\p ok.
\p ok.
\p ok.
\p ok.
\p ok.
\subsection{Naming Things with URIs}
\p ok.
\p ok.
Figure 2.1: ok.
\p ok.
\p ok.
\subsection{Making URIs Dereferenceable}
\p ok.
\p ok.
\p URIs descrevem coisas, e é comum usar uma URI para a coisa e outra para o documento que a descreve.
\p ok.
\p Documento original de cool uris e tb hash uris: \url{http://www.w3.org/TR/cooluris/}.
\subsubsection{303 URIs}
\p ok. See other <uri do documento que descreve o recurso>.
\p ok.
\p ok.
\p ok.
\p ok.
\p ok.
\p ok.
\p ok.
\p ok.
\subsubsection{Hash URIs}
\p ok.
\p ok. Fragment indentifier.
\p ok. Hash URIs para os objetos/conceitos do mundo.
\p ok.
\p ok.
\p ok.
\p ok.
\p ok.
\p ok.
\subsubsection{Hash versus 303}
\p ok.
\p ok.
\p ok.
\p ok. \#this como metodo hibrido Hash+303.
\p ok.
\subsection{Providing Useful RDF Information}
\p ok.
\p ok.
\subsubsection{The RDF Data Model}
\p ok.
\p ok.
\p citado vocabulário, falando que em 4.4 tem mais info sobre eles e apontando seu uso como predicado (!!!).
\p em 1. sobre triplas literais tipadas e nao tipadas. Em 2., importante distinção entre links internos e externos.
\p ok. Global graph e o texto do lee.
\p ok. Qqr um adiciona ou busca uma URI. Qqr URI eh ponto de inicio de navegação neste espaço de dados.
\subsubsection{RDF Serialization Formats}
\p ok.
\p ok. rdf/xml mime type.
\p ok.
\p ok.
\p ok.
\p ok. legal n-triples e rdf/json.
\subsection{Including Links to other Things}
\p ok. ode aos links externos.
\p ok. Tipo de links externos
\p ok.
\subsubsection{Relationship Links}
\p ok.
\p ok.
\p ok.
\p ok.
\subsubsection{Identity Links}
\p ok. Mundanizada a ocorrencias de multimas URIs para o mesmo objeto real.
\p ok.
\p ok.
\p URI aliases.
\p ok.
\p ok.
\p ok.
\p ok. tratamento evolutivo das identidades.
\p ok. owl:sameAs. Conteúdos devem ser consideradas aclamações de diferentes grupos, não fatos em si.
\subsubsection{Vocabulary Links}
\p ok.
\p ok.
\p ok.
\p ok.
\p ok. Termos RDFS, OWL e SKOS para equivalencias e relacoes proximas a isso.
\p ok.
\p ok. Integração gradual com namespaces externos.
\subsection{Conclusions}
\p ok.
\p ok.
\p Procurar python e js data browsers. Interessante que ele usa vocabulario como nome de campo a ser preenchido.
\p ok.
\p ok.
\p ok. Lembrei de mencionar que nossos dados sao 5 estrelas.
\p ok.
\section{The Web of Data}
\p ok.
\p ok.
\subsection{Bootstrapping the Web of Data}
\p ok. LOD e bootstrapping a web semântica.
\p ok.
\p ok.
\p ok. Sobre como adicionar os dados que publicamos ao grafico global reconhecido pelo LOD. dados.gov.br eh CKAN.
\subsection{Topologu of the Web of Data}
\p ok. Número de triplas publicadas e links inter datasets.
\subsubsection{Cross-Domain Data}
\p ok. dbpedia.
\p ok. Freebase.
\p Umbel, Yago, OpenCyc.
\subsubsection{Geographic Data}
\p Geonames.
\p ok. LinkedGeoData.
\p EuroStat, World Factbook, US Census.
\subsubsection{Media Data}
\p ok. BBC /programmes.
\p ok. BBC /music
\p ok. bbc wildlife finder.
\p ok.
\p New York Times subject headings.
\p ok. Muito legal do \VV{Calais}
\subsubsection{Government Data}
\p ok. Australia, Nova Zelândia, UK e EUA.
\p ok.
\p ok.
\p ok.
\subsubsection{Libraries and Education}
\p ok.
\p OpenLibrary.
\p Artigos academicos.
\p Aspire.
\p ok.
\p OAI-ORE, Europeana.
\p ok.
\subsubsection{Life Sciences Data}
\p ok.
\subsubsection{Retail and Commerce}
\p ok.
\p GoodRelations ontology.
\p ProductDB.
\subsubsection{User Generated Content and Social Media}
\p FlickrWrappr. Semantic MediaWiki. Ontowiki.
\p ok. Open Graph Protocol (FB) em RDFa, e IMDB.
\p Drupal
\subsection{Conclusions}
\p ok.
\p ok.
\section{Linked Data Design Considerations}
\p ok.
\p ok. URIs, RDF e external links
\p ok.
\subsection{Using URIs as Names for Things}
\p ok.
\subsubsection{Minting HTTP URIs}
\p ok.
\p ok.
\p ok.
\subsubsection{Guidelines for Creating Cool URIs}
\p ok.
\p ok. 
\p ok. 
\p ok. 
\p ok. 
\p ok.
\p ok.
\p ok.
\p ok.
\p ok.
\p ok.
\p ok.
\subsubsection{Example URIs}
\p ok.
\p ok.
\p ok.
\p ok.
\p ok.
\p ok.
\subsection{Describing Things with RDF}
\p ok.
\p ok.
\subsubsection{Literal Triples and Outgoing Links}
\p ok.
\p ok.
\p rdfs:label, foaf:name, skos:prefLabel, dcterms:title. dcterms:description, rdfs:comment.
\subsubsection{Incoming Links}
\p ok.
\p ok. \VV{Achado motivacao/critério para declarar propriedades inversas: os recursos envolvidos "ganham" a aresta no sentido contrario}
\p ok. foaf:primaryTopic, foaf:isPrimaryTopicOf
\p ok.
\subsubsection{Triples that Describe Related Resources}
\p ok.
\p ok.
\p ok.
\p ok.
\subsubsection{Triples that Describe the Description}
\p ok.
\subsection{Publishing Data about Data}
\subsubsection{Describing a Data Set}
\p ok.
\p ok.
\p ok.
\p ok.
\p ok. Semantic Sitemaps
\p ok.
\p ok.
\p ok.
\p ok.
\p ok. voiD
\p ok.
\p ok.
\p ok.
\p ok.
\p ok.
\p ok.
\p ok.
\subsubsection{Provenance Metadata}
\p ok.
\p ok. Dublin Core.
\p ok.
\p ok. NG4J.
\subsubsection{Licenses, Waivers and Norms for Data}
\p ok.
\p ok.
\p ok.
\p Licenses vc. Waivers ok.
\p ok.
\p ok.
\p ok.
\p Applying Licenses o Copyrightable Material ok.
\p ok.
\p ok.
\p ok.
\p ok.
\p Non-copyrightable Material ok.
\p ok.
\p ok.
\p ok.
\subsection{Choosing and Using Vocabularies to Describe Data}
\p ok.
\subsubsection{SKOS, RDFS and OWL}
\p ok.
\p ok. RDFS++
\p ok. Semantic Web for the Working Ontologist
\subsubsection{RDFS Basics}
\p ok.
\p ok.
\p ok.
\p ok.
\p ok.
\p ok.
\p ok.
\p Annotations in RDFS ok.
\p ok. \VV{rdfs:label e rdfs:comment sao recomendados para todos os recursos.}
\p Relating Classes and Properties. ok.
\p ok.
\subsubsection{A Little OWL}
\p equivalent
\p inverseFunctionalProperty
\p inverseOf
\subsubsection{Reusing Existing Terms}
\p ok.
\p ok. SIOC, Good Relations Ontology, BIBO, OAI, Review Vocabulary
\p ok.
\p ok.
\subsubsection{Selecting Vocabularies}
\p ok.
\p ok.
\subsubsection{Defining Terms}
\p ok. \VV{Melhor definir de forma meio solta do que sobrecarregar com axiomas de onologia}
\p Neologism, Protege, NeOn.
\subsection{Making Links with RDF}
\subsubsection{Making Links within a Data Set}
\p ok.
\p Publishing Incoming and Outgoing Links ok.
\p ok.
\subsubsection{Making links with External Data Sources}
\p ok.
\p ok.
\p ok. Fazendo triplas conectando de Triplestores externas para o BD. Entregar estas triplas para as comunidades das triples tores externas.
\p Choosing External Linking Targets ok.
\p ok.
\p ok.
\p Choosing Predicates for Linking ok.
\p ok.
\p ok.
\subsubsection{Settiong RDF Links Manually}
\p ok.
\p ok.
\p ok. Sindice e Falcon para achar URIs.
\p ok.
\subsubsection{Auto-generating RDF Links}
\p ok.
\p ok.
\p Key-based Approaches ok.
\p ok.
\p Similarity-based Approaches ok
\p ok.
\p ok.
\p ok. Silk, LIMES
\p ok. MiMOM, idMash, ObjectCoref
\p ok.
\section{Recipes for Publishing Linked Data}
\p ok.
\subsection{Linked Data Publishing Patterns}
\p ok.
\p ok.
Figura 5.1: \VV{Pubby.}
\subsubsection{Patterns in a Nutshell}
\p ok.
\p From Queryable Structured Data to Linked Data ok. RDF Wrappers. ok.
\p ok.
\p From Static Structured Daya to Linked Data. \VV{Lista de conversores:} \url{http://www.w3.org/wiki/ConverterToRdf#SQL} e havia uma no \url{http://simile.mit.edu/wiki/RDFizers}.
\p \VV{Linked Data Interface} de uma triplestore adequada ou \VV{Web server} ``clássico''.
\p From Text Documents to Linked Data. ok.
\subsubsection{Additional Considerations}
\p ok.
\p Data Volume: How much data need to be served? ok. \VV{Considerada opção do arquivo rdf de texto estático}.
\p \VV{Sugerido de quebrar o rdf textual em arquivos de texto para cada entidade, evitando operacoes com arquivos muito grandes}.
\p Data Dynamism: how often does data change? ok.
\p ok.
\subsection{The Recipes}
\p ok.
\subsubsection{Serving Linked Data as Static RDF/XML Files}
\p ok.
\p \VV{Reforcada prioridade p rdf/xml, deixando turtle e outras como complementares}.
\p Hosting and Naming Static RDF Files. ok.
\p ok.
\p Server-Side Configuration: MIME Types. ok.
\p ok. \VV{AddType application/rdf+xml .rdf} no \VV{httpd.conf} ou \VV{.htaccess}. 
\p ok.
\p ok. \VV{AddType text/n3;charset=utf-8 .n3} e \VV{AddType text/turtle;charset=utf-8 .ttl}.
\p  Making RDF Discoverable from HTML. ok.
\p  ok.
\p  ok. \VV{Autodiscovery do RDF com: <link rel="alternate" type="application/rdf+xml" href="company.rdf">}.
\subsubsection{Serving Linked Data as RDF Embedded in HTML Files}
\p ok.
\p ok.
\p ok.
\p ok.
\p ok. RDFa Distiller and Parser \url{http://www.w3.org/2007/08/pyRdfa/}
\p ok.
\subsubsection{Serving RDF and HTML with Custom Server-Side Scripts}
\p ok.
\p ok.
\p ok.
\p ok.
\p ok.
\p ok.
\subsubsection{Serving Linked Data from Relational Databases}
\p ok. Confirmando o que fizemos no participa: mantem estrutura original e apresenta versao linkada.
\p ok.
\p ok.
\p ok. \VV{D2R} p o que faz o ontop.
\p ok.
\p ok.
\subsubsection{Serving Linked Data from RDF Triple Stores}
\p ok.
\p ok. \VV{Pubby para uma interface de dados linkados atraves do endpoint sparql}
\p \VV{Pubby conf:loadRDF} para carregar o rdf estatico.
\p  \VV{ARC} pra inteface semantica, Sparql e +++.
\subsubsection{Serving linked Data by Wrapping Existing Application or Web APIs}
\p ok. Programmable Web.
\p ok.
\p ok.
\subsection{Additional Approaches to Publishing Linked Data}
\p ok. Derreferenciamento, HTML/RDFa, Dump rdf+xml ou ttl, sparql endpoint.
\subsection{Testing and Debugging Linked Data}
\p ok. \VV{Eyeball, Validator} \url{http://www.w3.org/RDF/Validator/}
\p \VV{Vapour, RDF:Alerts, Sindice Inspector}!!!!
\p ok. \url{http://richard.cyganiak.de/blog/2007/02/debugging-semantic-web-sites-with-curl/} cURL
\p ok.  LiveHTTPHeaders and ModifyHeaders Firefox
\p ok. linked data browsers
\p ok. Tabulator \url{http://www.w3.org/2005/ajar/tab}
\p ok. Marbles \url{http://dws.informatik.uni-mannheim.de/}
\p ok. LOD Browser Switch \url{http://browse.semanticweb.org/}
\p ok.
\subsection{Linked Data Publishing Checklist}
\p ok.  voID
\p ok. LOD Cloud document \url{http://lod-cloud.net/state/}.
\section{Consuming Linked Data}
\p ok. Padronização e abertura.
\p ok.
\subsection{Deployed Linked Data Applications}
\p ok.
\p ok.
\subsubsection{Generic Applications}
\p search, browse.
\p  Linked Data Browsers. ok.
\p ok.
\p ok. Dico hyperdata \url{http://wifo5-03.informatik.uni-mannheim.de/bizer/ng4j/disco/} Tabulator,  Marbles
\p ok. LinkSailor com visualizacoes;
\p ok. LOD Browser Switch
\p Linked Data Search Engines. ok. Crowlers que abrem p busca depois.
\p Sig.ma, Falcons, SWSE. 
\p Ok. \VV{Levantar ao menos uma search engine para nos}
\p Sig.ma e possibilidades de escolha das fontes.
\p ok. VisiNav e buscas semanticas complexas.
\p Buscas para aplicacoes: Sindice (instancias), Swoogle e Watson (ontologias).
\p ok.
\p Repositório central Uberblic.
\p ok. RDFa e Google.
\p ok.
\subsubsection{Domain-specific Applications}
\p ok.
\p ok.
\p Paggr e Dayta.me para agregacao de conteudo da web e para enriquecimento de info sobre o calendario da pessoa, respectivamente.
\p Talis Aspire.
\p DbPedia modile e o enriquecimento do grafo global pelos usuarios.
\p Diseasome e NCBO Resource Index.
\p Researcher Map achando perfis FOAF.
\p Faviki e tags com base no grafo global.
\p Shortipedia e Semantic MediaWiki.
\subsection{Developing a Linked Data Mashup}
\p ok.
\p ok.
\p ok.
\subsubsection{Software Requirements}
\p LDspider para crawl. Jena TDB de triplestore.
\p ok.
\subsubsection{Accessing Linked Data URIs}
\p ok. Derefenciar e raspar.
\p ok. COmando p ldspider crowl e mandar p endpoint
\p ok.
\p ok. Não ficou claro como o -b restringe a profundidade da busca.
\subsubsection{Representing Data Locally susing Named Graphs}
\p ok. Finalmente named graphs, graph set
\p ok.
\p ok.
\p ok. Legal exemplo de graph names.
\subsection{Querying Local Data with SPARQL}
\p \VV{SPARQL Recommendation} \url{http://www.w3.org/TR/rdf-sparql-query/}
\p ok.
\p ok.
\p ok. Exemplo excelente com named graphs. \VV{Pensar em named graphs para as triplificacoes e para as estruturas auxiliares (BoWs e redes).}
\p ok.
\p ok.
\subsection{Architecture of Linked Data Applications}
\p ok.
\p ok. observar \VV{Indicacoes do padrao ``busca federada''} que é o que estamos usando. Tentar disponiblizar outras arquiteturas.
\p ok.
\p ok. então: \VV{busca federada para cada instancia participativa, crowlers para apresentar estes dados relacionados a outros}. Garantido jah o crowler do LOD. Idealmente também o derreferenciamento.
\p como assim \VV{search the Web for mappings}??? sameAs + heuristica para identity.
\p ok.
\subsubsection{Accessing the Web of Data}
\p Crowlers, LD Client Libraries, Tabulator AJAR. sameAs.org.
\p ok. Billion Triples Challenge (BTC) Dataset
\subsubsection{Vocabulary Mapping}
\p ok.
\p ok. SPARQL++, R2R, RIF.
\p ok.
\p ok.
\subsubsection{Identity Resolution}
\p ok.
\p ok. Silk Server.
\subsubsection{Provenance Tracking}
\p ok.
\subsubsection{Data Quality Assessment}
\p É parte necessário em crowls maiores um modulo para limpar e remover rdf spam.
\p ok.
\p ok. Rank, filter com WIQA framework, fuse com DERI Pipes e o KnoFuss.
\p ok. Metodos para aferir qualidade \url{http://sourceforge.net/projects/trdf/}
\p ok.
\p ok.
\subsubsection{Cachung Web Data Locally}
\p ok. 1 biblhao de triplas por maquina. Berlin SPARQL Benchmark.
\p ok. MapReduce, Hadoop.
\subsubsection{Using Web Data in the Application Context}
\p \url{http://www.w3.org/2001/sw/wiki/GephiSemanticWebImportPlugin}, \url{http://www.w3.org/2001/sw/wiki/Javascript}, \url{http://www.w3.org/2001/sw/wiki/Python}
\p \VV{Information Workbrench} \url{http://iwb.fluidops.com/resource/dbpedia:Animal}
\subsection{Effort Distribution between Publishers, Consumers and Third Parties}
\p ok.
\p ok.
\p ok.
\p ok.
\p ok.
\p ok.
\p ok.
\p ok. dataspace.
\p ok. Web de dados como uma construcao social.
\p ok.
\section{Summary and Outlook}
\p ``concerned about future-proofing their assets.''.
\p ok.
\p ok.
\p ok.
\p ok.
\chapter{SKOS Simple Knowledge Organization System Primer}
\url{http://www.w3.org/TR/skos-primer/}

SKOS—Simple Knowledge Organization System—provides a model for expressing the basic structure and content of concept schemes such as thesauri, classification schemes, subject heading lists, taxonomies, folksonomies, and other similar types of controlled vocabulary. As an application of the Resource Description Framework (RDF), SKOS allows concepts to be composed and published on the World Wide Web, linked with data on the Web and integrated into other concept schemes.

This document is a user guide for those who would like to represent their concept scheme using SKOS.

In basic SKOS, conceptual resources (concepts) are identified with URIs, labeled with strings in one or more natural languages, documented with various types of note, semantically related to each other in informal hierarchies and association networks, and aggregated into concept schemes.

In advanced SKOS, conceptual resources can be mapped across concept schemes and grouped into labeled or ordered collections. Relationships can be specified between concept labels. Finally, the SKOS vocabulary itself can be extended to suit the needs of particular communities of practice or combined with other modeling vocabularies.

This document is a companion to the SKOS Reference, which provides the normative reference on SKOS.

\section{Introduction}
\p ok.
\p ok.
\p ok.
\subsection{About this Primer}
\p ok.
\p ok. SKOS-UCR.
\p ok.
\p ok.
\p About Examples in this Primer. ok.
\p ok.
\p ok.
\section{SKOS Essentals}
\p ok.
\p ok. Estranho ainda informal hierarchies e curioso sobre os ``concept schemes''.
\subsection{Concepts}
\p ok.
\p ok. \VV{Recipes} \url{http://www.w3.org/TR/swbp-vocab-pub/#recipe5aA} com PURL.
\subsection{Labels}
\p ok.
\p ok.
\subsubsection{Preferred Lexical Labels}
\p ok.
\p ok.
\p ok. Um prefLabel por lingua, max.
\p ok. Recomenda-se n usar o mesmo prefLabel para 2 conceitos diferentes, embora o modelo SKOS não imponha isso.
\subsubsection{Alternative Lexical Labels}
\p ok.
\p ok. \VV{synonyms, Near-synonyms, abbreviations and acronyms}
\p ok.
\p ok.
\subsubsection{Hidden Lexical Labels}
\p ok. \VV{Usar hiddenLabel para as escritas com grafia e acentuação errada}.
\subsection{Semantic Relationships}
\p ok. broader e narrower
\p ok. related
\subsubsection{Broader/Narrower Relationships}
\p ok.
\p ok.
\p muito interessante: \VV{skos: broader e narrower nao sao transitivas pois sao usados esquemas em que nao sao transitivos}
\p Em geral, a literatura aponta de vocabulários aponta que relacoes hierarquivas sao irreflexivas. Porém: rdfs:subClassOf é reflexiva.
\p ok.
\subsubsection{Associative Relationships}
\p ok.
\p skos:related é simétrica mas não transitiva.
\p associacao (related) e cadeia de hierarquia (broader, narrower) sao excludentes. Ou seja, se A related B, nao pode haver cadeia narrower ou broader entre A e B.
\subsection{Documentary Notes}
\p ok.
\p note, scopeNote, definition, example, historyNote. \VV{Incluir no historyNote se o conceito consta na constituição de 88 e no decreto PNPS}.
\p ok.
\p ok.
\p ok.
\p ok.
\p ok.
\p ok.
\p ok.
\p ok.
\p ok.
\p ok.
\subsection{Concept Schemes}
\p ok.
\p ok.
\p ok.
\p ok.
\p ok.
\p ok.
\p ok.
\section{Networking Knowledge Organization Systems on he Semantic Web}
\p ok.
\subsection{Mapping Concept Schemes}
\p ok.
\p ok. Conceptual mappings.
\p exactMatch, closeMatch, broadMatch, narrowMatch, relatedMatch.
\p ok.
\p ok.
\p ok.
\p ok. exactMatch eh transitivo e closeMatch não.
\p ok.
\p ok.
\p ok.
\p ok.
\subsection{Re-using and Extending Concept Schemes}
\p uma URI pode estar em mais de um vocabulario.
\p ok.
\p ok.
\p ok.
\p ok. importancia do derreferencialmento.
\p ok.
\p ok.
\p ok. imports e OWL full.
\p ok.
\subsection{Subject Indexing and SKOS}
\p ok.
\p ok.
\p ok. skos concept como objeto de uma relacao dct:subject
\section{Advances SKOS: When KOSs are not Simple Anymore}
\p ok.
\p ok.
\p ok.
\subsection{Collections of Concepts}
\p ok.
\p ok.
\p ok.
\p ok.
\p ok.
\p ok.
\p ok.
\p ok.
\p ok.
\subsection{Advanced Documentation Features}
\p ok.
\p ok.
\p ok.
\p ok.
\subsection{Relationships between Labels}
\p ok.
\p ok.
\p ok.
\p ok.
\p ok.
\p ok.
\subsection{Coordinating Concepts}
\p ok.
\p ok.
\p ok.
\p ok.
\p ok.
\p ok.
\subsection{Transitive Hierarchies}
\p ok.
\p ok.
\p ok.
\p ok.
\p ok.
\p ok.
\p ok.
\p ok. Mas estranho o skos:broaderTransitive ser mais geral que skos:broader.
\subsection{Notations}
\p ok.
\p ok.
\p ok.
\p ok.
\p ok.
\subsection{On Specializing the SKOS Model}
\p ok.
\p ok.
\p ok.
\p ok.
\p ok.
\p ok. \VV{nao usar skos no sujeito}
\p ok.
\section{Combining SKOS with other Modeling Approaches}
\p ok. É o nosso caso quanto aa ontologia OWL e vocabulário SKOS.
\subsection{Use of Labels Outside of SKOS}
\p ok.
\p ok.
\p ok.
\p ok. Talvez usar prefLabel nas classes para linkar vocabulario e ontologia.
\subsection{SKOS Concepts and OWL Classes}
\p ok.
\p ok.
\p ok.
\p ok.
\p ok.
\p ok. ex:PaintingClass ex:correspondingConcept ex:PaintingConcept
\p ok.
\p ok.
\subsection{SKOS, RDF Datasets and Information Containment}
\p ok.
\p ok.
\p ok.
\p ok.
\chapter{The Principles of LiquidFeedback (2014)}
\PP 5
\mysection{0}{Foreword}
\p ok.
\p ok.
\PP 6
\p ok.
\p ok.
\PP{7}
\p ok.
\p ok.
\PP{13}
\section{Introduction}
\p ok. LF é um soft?
\p ok.
\p ok.
\p ok.
\PP{14}
\p ok.
\p ok.
\subsection{Democracy vs. Republic and a new approach}
\p ok.
\p ok.
\p ok.
\PP{15}
\p ok. Parece apontar para repúblicas como sendo não democrática.
\p ok.
\p ok.
\p ok.
\p ok.
\p ok.
\p ok.
\p ok.
\PP{16}
\p ok.
\p ok.
\subsection{Project LiquidFeedback}
\p ok.
\p ok.
\p ok.
\p ok.
\p ok.
\PP{17}
\p
\p
\p
\p
\p
\p
\p
\p
\p
\p
\p
\p
\p
\p
\p
\p
\p
\p
\p
\p
\p
\p
\p
\p
\p
\p
\p
\p
\p
\p
\p
\p
\p
\chapter{Artigos}
\section{}
\url{http://www.w3.org/DesignIssues/LinkedData.html}
\section{Tracking RDF Graph Provenance using RDF Molecules}
\url{http://ebiquity.umbc.edu/_file_directory_/papers/178.pdf}

\newpage
\bibliography{bibliografia}
\newpage
%\listoffigures
\input{listadeabreviaturas.tex}
\newpage
\printindex
\newpage
%\input{listadeanexos.tex}
\appendix
\end{document}
